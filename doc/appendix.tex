\chapter{Abbildungen}
\begin{sidewaysfigure}[ht]
	\centering
	\includegraphics[keepaspectratio, width=\textwidth, height=\textheight]{img/klassendiagramm_v01}
	\captionsetup{format=hang}
	\caption{Erster Entwurf des Klassendiagramms}
	\small Quelle: eigene Darstellung mittels \url{http://dia-installer.de/index.html.de}
	\label{fig:anhang_klassendiagramm01}
\end{sidewaysfigure}

\begin{sidewaysfigure}[ht]
	\centering
	\includegraphics[keepaspectratio, width=\textwidth, height=\textheight]{img/klassendiagramm_v02}
	\captionsetup{format=hang}
	\caption{Zweiter Entwurf des Klassendiagramms}
	\small Quelle: eigene Darstellung mittels \url{https://draw.io/} 
	\label{fig:anhang_klassendiagramm02}
\end{sidewaysfigure}

\begin{sidewaysfigure}[ht]
	\centering
	\includegraphics[keepaspectratio, width=\textwidth, height=\textheight]{img/sequenzdiagramm_reservation}
	\captionsetup{format=hang}
	\caption{Sequenzdiagramm eines Reservierungsvorganges}
	\small Quelle: eigene Darstellung mittels \url{https://draw.io/} 
	\label{fig:Anhang_seq_reservieren}
\end{sidewaysfigure}

\begin{sidewaysfigure}[ht]
	\centering
	\includegraphics[keepaspectratio, width=\textwidth, height=\textheight]{img/ER-Modell}
	\captionsetup{format=hang}
	\caption{\acs{ER-Modell} der Datenbank}
	\small Quelle: eigene Darstellung mittels \url{https://www.lucidchart.com/}
	\label{fig:Anhang_ER-Modell}
\end{sidewaysfigure}

\chapter{Screenshots aus dem Front-End}
\label{sec:screenshots_frontend}
Die folgenden Screenshots sind alle von der Seite eines Filmes, auf der sich weitere Details zu diesem finden.

\begin{figure}[ht]
	\centering
	\includegraphics[width=\textwidth]{img/screenshots/film02a}
	\captionsetup{format=hang}
	\caption{Filmdetails}
	\label{fig:film02a}
\end{figure}

\begin{figure}[ht]
	\centering
	\includegraphics[width=\textwidth]{img/screenshots/film03}
	\captionsetup{format=hang}
	\caption{Liste der Vorstellungen der nächsten 7 Tage}
	\label{fig:film03}
\end{figure}

\begin{figure}[!t]
	\centering
	\includegraphics[width=\textwidth]{img/screenshots/film04}
	\captionsetup{format=hang}
	\caption{Bewertungen zu einem Film}
	\label{fig:film04}
\end{figure}

\chapter{Quellcode}

\begin{lstlisting}[style=lstJava, caption={Erstellen eines Movie-\acs{DTO}s aus einer Movie-Entität mit Hilfe des eigen erstellten EntityToToHelper}, label={lst:EntityToToHelper_movie}]
public static MovieTo createMovieTo ( Movie entity, boolean withShow )
{
	if ( null != entity )
	{
		MovieTo movieTo = new MovieTo();
		movieTo.setId(entity.getId());
		movieTo.setDescription(entity.getDescription());
		movieTo.setFsk(entity.getFsk());
		movieTo.setDuration(entity.getDuration());
		movieTo.setName(entity.getName());
		movieTo.setGenres(createGenreTos(entity.getGenres()));
		movieTo.setRatings(createRatingTos(entity.getRatings()));
		if ( withShow )
		{
			movieTo.setShows(createShowTos(entity.getShows(), false));
		} // end withSow
		movieTo.setActors(createActorTos(entity.getActors()));
		return movieTo;
	}  // end if null
	return null;
}
\end{lstlisting}

\newpage

\begin{lstlisting}[style=lstJava, caption={Erstellen einer Movie-Entität aus einem Movie-\acs{DTO} mit Hilfe des eigen erstellten ToToEntityHelper}, label={lst:ToToEntityHelper_movie}]
public static Movie createMovieEntity ( MovieTo transferObject, boolean withShow )
{
	if ( null != transferObject )
	{
		Movie movie = new Movie();
		movie.setId(transferObject.getId());
		movie.setActors(createActorEntities(transferObject.getActors()));
		movie.setDescription(transferObject.getDescription());
		movie.setFsk(transferObject.getFsk());
		movie.setDuration(transferObject.getDuration());
		movie.setName(transferObject.getName());
		movie.setRatings(createRatingEntities(transferObject.getRatings()));
		if ( withShow )
		{
			movie.setShows(createShowEntities(transferObject.getShows(), false));
		}
		movie.setGenres(createGenreEntities(transferObject.getGenres()));
		return movie;
	}
	return null;
}
\end{lstlisting}

\newpage

\begin{lstlisting}[style=lstJava, caption={Überprüfung, ob eine Reservierung möglich ist}, label={lst:Angang_Prüfung_ob_Reservierung_möglich}, tabsize=2]
private boolean checkIfSeatsAreBookable ( List<SeatTo> seatsToProof, List<TicketTo> bookedTicketTos, List<BlockTo> blockTos, String sessiontoken ) {
	boolean bookable = true;
	
	for ( SeatTo s : seatsToProof ) {
		if ( bookable ) {
			for ( TicketTo t : bookedTicketTos ) {
				if ( s.getId() == t.getSeat().getId() ) {
					bookable = false;
					s.setOccupied(true);
					break;
				}
				else {
					s.setOccupied(false);
				}
			} // end for ticket

			// verification for blocked elements
			for ( BlockTo b : blockTos ) {
				if ( s.getId() == b.getSeat().getId() &&
				     !(b.getSessiontoken().equals(sessiontoken)) ) {
					bookable = false;
					s.setBlocked(true);
					break;
				}
				else {
					s.setBlocked(false);
				}
			} // end for block
		}	// end if bookable
	}	// end for seatTo

	return bookable;
}
\end{lstlisting}

\newpage

\begin{lstlisting}[style=lstJava, caption={Erstellen eines Reservierungs-\acs{DTO}s}, label={lst:Anhang_Erstellen_Reservierung}]
private ReservationTo createReservationForSeats ( List<SeatTo> seatTos, ShowTo showTo )
{
	List<TicketTo> toBook = new ArrayList<>();
	ReservationTo createdReservationTo = new ReservationTo();
		
	for ( SeatTo seatTo : seatTos )
	{
		seatTo.setOccupied(true);
	
		TicketTo ticketTo = new TicketTo();
		ticketTo.setSeat(seatTo);
		ticketTo.setShow(showTo);
		ticketTo.setReservation(createdReservationTo);
		if ( seatTo.isReducedPrice() )
		{
			ticketTo.setReducedPrice(true);
		}
		toBook.add(ticketTo);
	}
	createdReservationTo.setDateOfReservation(Utils.convertDateToString(new Date()));
	createdReservationTo.setTickets(toBook);
	return createdReservationTo;
}
\end{lstlisting}

\newpage

\begin{lstlisting}[style=lstJava, caption={Ausschnitt der Test-Klasse der EntityToToHelper-Klasse}, label={src:entitytotohelpertest}, tabsize=2]
public class EntityToToHelper_Test {
	EntityToToHelper EtoToHelper = new EntityToToHelper();

	@Before
	public void initialize ( ) {
		EmployeeTo testEmployeeTo = new EmployeeTo();
		testEmployeeTo.setId(1);
		testEmployeeTo.setDateofbirth(testDateString);
		testEmployeeTo.setEmail("mail");
		testEmployeeTo.setFirstname("firstname");
		testEmployeeTo.setLastname("lastname");

		Employee testEmployeeEntity = new Employee();
		testEmployeeEntity.setId(1);
		testEmployeeEntity.setDateofbirth(testDateDate);
		testEmployeeEntity.setEmail("mail");
		testEmployeeEntity.setFirstname("firstname");
		testEmployeeEntity.setLastname("lastname");
	}

	@Test
	public void testCreateEmployeeTo ( ) {
		assertEquals(null, EtoToHelper.createEmployeeTo(null));

		EmployeeTo compareEmployeeTo = EtoToHelper.createEmployeeTo(testEmployeeEntity);

		assertThat(testEmployeeTo.getId(), equalTo(compareEmployeeTo.getId()));
		assertThat(testEmployeeTo.getDateofbirth(), equalTo(compareEmployeeTo.getDateofbirth()));
		assertThat(testEmployeeTo.getEmail(), equalTo(compareEmployeeTo.getEmail()));
		assertThat(testEmployeeTo.getFirstname(), equalTo(compareEmployeeTo.getFirstname()));
		assertThat(testEmployeeTo.getLastname(), equalTo(compareEmployeeTo.getLastname()));
	}
}
\end{lstlisting}
