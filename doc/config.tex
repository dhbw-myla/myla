% !TEX root =  master.tex

% language, font, colors
\usepackage[ngerman]{babel}
\usepackage[utf8]{inputenc}
\usepackage[german=quotes]{csquotes} 	% correct quotes using \enquote{}
\usepackage[T1]{fontenc}
\usepackage{lmodern} % latin modern font
\usepackage[onehalfspacing]{setspace}
\usepackage{xcolor}

% hyperlinks
\PassOptionsToPackage{hyphens}{url}\usepackage[hidelinks=true]{hyperref}

% commands for author and title
\newcommand{\TitelDerArbeit}[1]{\def\DerTitelDerArbeit{#1}\hypersetup{pdftitle={#1}}}
\newcommand{\AutorDerArbeit}[1]{\def\DerAutorDerArbeit{#1}\hypersetup{pdfauthor={#1}}}
\newcommand{\Firma}[1]{\def\DerNameDerFirma{#1}}
\newcommand{\Kurs}[1]{\def\DieKursbezeichnung{#1}}

% create free field for chapter
\makeatletter
\newcommand{\freechapterauthor}[1]{%
	{\parindent0pt\vspace*{-25pt}%
		\linespread{1.1}\large\scshape#1%
		\par\nobreak\vspace*{35pt}}
	\@afterheading%
}
\makeatother

% create author for chapter
\makeatletter
\newcommand{\chapterauthor}[1]{%
	{\parindent0pt\vspace*{-25pt}%
		\linespread{1.1}\large\scshape Autor: #1%
		\par\nobreak\vspace*{35pt}}
	\@afterheading%
}
\makeatother

% create multiple author for chapter
\makeatletter
\newcommand{\chaptermulitpleauthor}[2]{%
	{\parindent0pt\vspace*{-25pt}%
		\linespread{1.1}\large\scshape Autoren: #1, #2%
		\par\nobreak\vspace*{35pt}}
	\@afterheading%
}
\makeatother

% create free for section
\makeatletter
\newcommand{\freeauthorsection}[1]{%
	{\parindent0pt\vspace*{0 pt}%
		\linespread{1.1}\scshape#1%
		\par\nobreak\vspace*{15pt}}
	\@afterheading%
}
\makeatother

% create author for section
\makeatletter
\newcommand{\authorsection}[1]{%
	{\parindent0pt\vspace*{0 pt}%
		\linespread{1.1}\scshape Autor: #1%
		\par\nobreak\vspace*{15pt}}
	\@afterheading%
}
\makeatother

% create multiple author for section
\makeatletter
\newcommand{\multipleauthorsection}[2]{%
	{\parindent0pt\vspace*{0 pt}%
		\linespread{1.1}\scshape Autoren: #1, #2%
		\par\nobreak\vspace*{15pt}}
	\@afterheading%
}
\makeatother

%\usepackage{microtype}% verbesserter Randausgleich
%\setlength\emergencystretch{1em}
% correct superscripts
\usepackage{fnpct}

\usepackage{footnote}
\usepackage{rotating}

% calc 
\usepackage{calc} % Used for extra space below footsepline

% bibliography settings
% author-year-style with footnotes (Chicago)
\usepackage[backend=biber, autocite=footnote, style=authoryear, dashed=false]{biblatex}
\AdaptNoteOpt\footcite\multfootcite
\AdaptNoteOpt\autocite\multautocite

\DefineBibliographyStrings{ngerman}{  % change u.a. to et al. (german only!)
	andothers = {{et\,al\adddot}},
}
% Command to output section title headings
\newcommand{\cvsect}[1]{% The only parameter is the section text
	\vspace{\baselineskip} % Whitespace before the section title
	\colorbox{black}{\textcolor{white}{\MakeUppercase{\textbf{#1}}}}\\% Section title
}

\usepackage{longtable} % Required for tables that span multiple pages
\setlength{\LTpre}{0pt} % Remove default whitespace before longtable
\setlength{\LTpost}{0pt} % Remove default whitespace after longtable

\setlength{\tabcolsep}{0pt} % No spacing between table columns

% Environment to hold a new list of entries
\newenvironment{entrylist}{
	\begin{longtable}[H]{l l}
	}{
	\end{longtable}
}

\newcommand{\entry}[2]{% First argument for the leftmost date(s) text, second is for the entry heading
	\vspace{-0.15cm}
	\parbox[t]{0.33\textwidth}{% 25.0% of the text width of the page
		#1 % Leftmost entry date(s) text
	}%
	&\parbox[t]{0.68\textwidth}{% 75.0% of the text width of the page
		{#2}% Entry heading text
	}\\\\}

\newcounter{barcount}

% Environment to hold a new bar chart
\newenvironment{barchart}[1]{ % The only parameter is the maximum bar width, in cm
	\newcommand{\barwidth}{0.35}
	\newcommand{\barsep}{0.55}
	
	% Command to add a bar to the bar chart
	\newcommand{\baritem}[2]{ % The first argument is the bar label and the second is the percentage the current bar should take up of the total width
		\pgfmathparse{##2}
		\let\perc\pgfmathresult
		
		\pgfmathparse{#1}
		\let\barsize\pgfmathresult
		
		\pgfmathparse{\barsize*##2/5}
		\let\barone\pgfmathresult
		
		\pgfmathparse{(\barwidth*\thebarcount)+(\barsep*\thebarcount)}
		\let\barx\pgfmathresult
		
		\filldraw[fill=none, draw=black!] (0,-\barx) rectangle (5,-\barx-\barwidth);
		\filldraw[fill=black!90, draw=black!90] (0,-\barx) rectangle (\barone,-\barx-\barwidth);
		
		\node [label=180:{\textcolor{black}{##1: ##2/5}}] at (0,-\barx-0.175) {};
		\addtocounter{barcount}{1}
	}
	\begin{tikzpicture}
	\setcounter{barcount}{0}
}{
	\end{tikzpicture}
	\vspace{0.3cm}
}

\usepackage{tikz} % Required for creating the plots
\usetikzlibrary{shapes, backgrounds}
\tikzset{x=1cm, y=1cm} % Default tikz units

%%% Uncomment the following lines to support hard URL breaks in bibliography 
%\apptocmd{\UrlBreaks}{\do\f\do\m}{}{}
%\setcounter{biburllcpenalty}{9000}% Kleinbuchstaben
%\setcounter{biburlucpenalty}{9000}% Großbuchstaben

\setlength{\bibparsep}{\parskip}	% add some space between biblatex entries in the bibliography
\addbibresource{bibliography.bib}	% add file bibliography.bib as biblatex resource

% footnotes (count footnotes over chapters)
\usepackage{chngcntr}
\counterwithout{footnote}{chapter}

% acronyms
\makeatletter
\usepackage[printonlyused]{acronym}
\@ifpackagelater{acronym}{2015/03/20}
  {%
    \renewcommand*{\aclabelfont}[1]{\textbf{\textsf{\acsfont{#1}}}}
  }%
  {%
  }%
\makeatother

% listings
\usepackage{listings}
\renewcommand{\lstlistingname}{Quelltext} 
\renewcommand{\lstlistlistingname}{Quelltextverzeichnis}

%% More configuration for listings is in configlistings.tex %%


% extra packages
\usepackage{graphicx}			% use various graphics formats
\usepackage{subfig}				% sub-figures, e.g. side-by-side pictures
\usepackage[german]{varioref}	% nicer references \vref
\usepackage{caption}			% better captions
\usepackage{booktabs}			% nicer tabs
\usepackage{array}
\usepackage{pdfpages}			% for signed ewerkl

% algorithms
\usepackage{algorithm}
\usepackage{algpseudocode}
\renewcommand{\listalgorithmname}{Algorithmenverzeichnis}
\floatname{algorithm}{Algorithmus}

% page header/footer
\RequirePackage[automark,headsepline,footsepline]{scrpage2}
\pagestyle{scrheadings}
\renewcommand*{\pnumfont}{\upshape\sffamily}
\renewcommand*{\headfont}{\upshape\sffamily}
\renewcommand*{\footfont}{\upshape\sffamily}
\renewcommand{\chaptermarkformat}{}
\RedeclareSectionCommand[beforeskip=0pt]{chapter}
\clearscrheadfoot

\ifoot[\rule{0pt}{\ht\strutbox+\dp\strutbox}DHBW Mannheim]{\rule{0pt}{\ht\strutbox+\dp\strutbox}DHBW Mannheim}
\ofoot[\rule{0pt}{\ht\strutbox+\dp\strutbox}\pagemark]{\rule{0pt}{\ht\strutbox+\dp\strutbox}\pagemark}

\ohead{\headmark}

% highlight notes as not yet finished
\newenvironment{note}{\color{gray}}{}