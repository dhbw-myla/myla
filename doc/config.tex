% !TEX root =  master.tex

% language, font, colors
\usepackage[ngerman]{babel}
\usepackage[utf8]{inputenc}
\usepackage[german=quotes]{csquotes} 	% correct quotes using \enquote{}
\usepackage[T1]{fontenc}
\usepackage{lmodern} % latin modern font
\usepackage[onehalfspacing]{setspace}
\usepackage{xcolor}

% hyperlinks
%\PassOptionsToPackage{hyphens}{url}
\usepackage[hidelinks=true]{hyperref}

% commands for author and title
\newcommand{\TitelDerArbeit}[1]{\def\DerTitelDerArbeit{#1}\hypersetup{pdftitle={#1}}}
\newcommand{\AutorDerArbeit}[1]{\def\DerAutorDerArbeit{#1}\hypersetup{pdfauthor={#1}}}
\newcommand{\Firma}[1]{\def\DerNameDerFirma{#1}}
\newcommand{\Kurs}[1]{\def\DieKursbezeichnung{#1}}


%\usepackage{microtype}% verbesserter Randausgleich
%\setlength\emergencystretch{1em}
% correct superscripts
\usepackage{fnpct}

\usepackage{footnote}
\usepackage{rotating}

% calc
\usepackage{calc} % Used for extra space below footsepline

% bibliography settings
% author-year-style with footnotes (Chicago)
\usepackage[backend=biber, autocite=footnote, style=authoryear, dashed=false]{biblatex}
\AdaptNoteOpt\footcite\multfootcite
\AdaptNoteOpt\autocite\multautocite

\DefineBibliographyStrings{ngerman}{  % change \ua to et al. (german only!)
	andothers = {{et\,al\adddot}},
}


% Command to output section title headings
\newcommand{\cvsect}[1]{% The only parameter is the section text
	\vspace{\baselineskip} % Whitespace before the section title
	\colorbox{black}{\textcolor{white}{\MakeUppercase{\textbf{#1}}}}\\% Section title
}

\newcounter{barcount}

% Environment to hold a new bar chart
\newenvironment{barchart}[1]{ % The only parameter is the maximum bar width, in cm
	\newcommand{\barwidth}{0.35}
	\newcommand{\barsep}{0.55}
	
	% Command to add a bar to the bar chart
	\newcommand{\baritem}[2]{ % The first argument is the bar label and the second is the percentage the current bar should take up of the total width
		\pgfmathparse{##2}
		\let\perc\pgfmathresult
		
		\pgfmathparse{#1}
		\let\barsize\pgfmathresult
		
		\pgfmathparse{\barsize*##2/5}
		\let\barone\pgfmathresult
		
		\pgfmathparse{(\barwidth*\thebarcount)+(\barsep*\thebarcount)}
		\let\barx\pgfmathresult
		
		\filldraw[fill=none, draw=black!] (0,-\barx) rectangle (5,-\barx-\barwidth);
		\filldraw[fill=black!90, draw=black!90] (0,-\barx) rectangle (\barone,-\barx-\barwidth);
		
		\node [label=180:{\textcolor{black}{##1: ##2/5}}] at (0,-\barx-0.175) {};
		\addtocounter{barcount}{1}
	}

	\newcommand{\baritemNL}[2]{ % The first argument is the bar label and the second is the percentage the current bar should take up of the total width
		\pgfmathparse{##2}
		\let\perc\pgfmathresult
		
		\pgfmathparse{#1}
		\let\barsize\pgfmathresult
		
		\pgfmathparse{\barsize*##2/5}
		\let\barone\pgfmathresult
		
		\pgfmathparse{(\barwidth*\thebarcount)+(\barsep*\thebarcount)}
		\let\barx\pgfmathresult
		
		\filldraw[fill=none, draw=black!] (0,-\barx) rectangle (5,-\barx-\barwidth);
		\filldraw[fill=black!90, draw=black!90] (0,-\barx) rectangle (\barone,-\barx-\barwidth);
		
		\node [label=180:{}] at (0,-\barx) {};
		\addtocounter{barcount}{1}
	}

	\begin{tikzpicture}
	\setcounter{barcount}{0}
}{
	\end{tikzpicture}
	\vspace{0.3cm}
}

\usepackage{tikz} % Required for creating the plots
\usetikzlibrary{shapes, backgrounds}
\tikzset{x=1cm, y=1cm} % Default tikz units

%%% Uncomment the following lines to support hard URL breaks in bibliography 
\apptocmd{\UrlBreaks}{\do\f\do\m}{}{}
\setcounter{biburllcpenalty}{9000}% Kleinbuchstaben
\setcounter{biburlucpenalty}{9000}% Großbuchstaben

\setlength{\bibparsep}{\parskip}	% add some space between biblatex entries in the bibliography
\addbibresource{bibliography-martin.bib}	% add file bibliography.bib as biblatex resource
\addbibresource{bibliography-rene.bib}	% add file bibliography.bib as biblatex resource
\addbibresource{bibliography-sascha.bib}	% add file bibliography.bib as biblatex resource
\addbibresource{bibliography-niko.bib}	% add file bibliography.bib as biblatex resource
\addbibresource{bibliography-julian.bib}	% add file bibliography.bib as biblatex resource
\addbibresource{bibliography-erik.bib}	% add file bibliography.bib as biblatex resource
% footnotes (count footnotes over chapters)
\usepackage{chngcntr}
\counterwithout{footnote}{chapter}

% acronyms
\makeatletter
\usepackage[printonlyused]{acronym}
\@ifpackagelater{acronym}{2015/03/20}
  {%
    \renewcommand*{\aclabelfont}[1]{\textbf{\textsf{\acsfont{#1}}}}
  }%
  {%
  }%
\makeatother

% listings
\usepackage{listings}
\renewcommand{\lstlistingname}{Quelltext}
\renewcommand{\lstlistlistingname}{Quelltextverzeichnis}

%% More configuration for listings is in configlistings.tex %%


% extra packages
\usepackage{graphicx}			% use various graphics formats
\usepackage[german]{varioref}	% nicer references \vref
\usepackage{caption}			% better captions
\usepackage{booktabs}			% nicer tabs
\usepackage{array}
\usepackage{pdfpages}			% for signed ewerkl
%\usepackage{import}				% better import of section in several files
\usepackage{subfigure}

% table stuff
\newcolumntype{L}[1]{>{\raggedright\let\newline\\\arraybackslash\hspace{0pt}}m{#1}} % combines l and p
\newcolumntype{C}[1]{>{\centering\let\newline\\\arraybackslash\hspace{0pt}}m{#1}} % combines c and p
\newcolumntype{R}[1]{>{\raggedleft\let\newline\\\arraybackslash\hspace{0pt}}m{#1}} % combines r and p

% funktioniert aktuell nicht wie normal -> muss man nochmal drüber schauen.. oder ersetzen
\usepackage{tabularx}

% page header/footer
\RequirePackage{scrlfile}
\ReplacePackage{scrpage2}{scrlayer-scrpage}
\RequirePackage[automark,headsepline,footsepline]{scrpage2}
\pagestyle{scrheadings}
\renewcommand*{\pnumfont}{\upshape\sffamily}
\renewcommand*{\headfont}{\upshape\sffamily}
\renewcommand*{\footfont}{\upshape\sffamily}
\renewcommand{\chaptermarkformat}{}
\RedeclareSectionCommand[beforeskip=0pt]{chapter}
\clearscrheadfoot

\ifoot[\rule{0pt}{\ht\strutbox+\dp\strutbox}DHBW Mannheim]{\rule{0pt}{\ht\strutbox+\dp\strutbox}DHBW Mannheim}
\ofoot[\rule{0pt}{\ht\strutbox+\dp\strutbox}\pagemark]{\rule{0pt}{\ht\strutbox+\dp\strutbox}\pagemark}

\ohead{\headmark}

% highlight notes as not yet finished
\newenvironment{note}{\color{gray}}{}

\usepackage{todonotes}
\usepackage{xspace}% http://ctan.org/pkg/xspace
\usepackage{fontawesome5}
\usepackage{float} % !important for figure position

\usepackage{pifont} % for description