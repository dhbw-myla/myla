\section{Reservieren}
\label{sec:reservieren}

\subsection{Herausforderung}
\label{ssec:herausforderung_reservieren}

Eine weitere Herausforderung bestand darin, das Reservieren eines Sitzes im Back-End zu realisieren.
Hierfür mussten ebenso verschiedene Aspekte betrachtet werden.

\begin{enumerate}
	\label{enum:reservieren}
	\item \label{itm:zeitpunkt_reservieren}Das Reservieren darf nur, wie das Blocken, bis zu einem bestimmten Zeitpunkt möglich sein.
	\item \label{itm:geblockt_durch_benutzer} Falls der Sitzplatz durch den selben Benutzer geblockt wurde, darf dieser diesen buchen.
	Andernfalls darf es keine erfolgreiche Reservierung geben.
	\item \label{itm:mehr_eine_vorstellung_reservieren}Eine Mehrfachvergabe des selben Sitzplatzes für eine Vorstellung muss wie beim Blocken ausgeschlossen sein (vgl. Herausforderung \ref{enum:blocken} Punkt \ref{itm:mehr_eine_vorstellung}).
	\item \label{itm:mehr_mehrere_vorstellungen_reservieren}Die Belegung eines Sitzes für verschiedene Vorstellungen muss gewährleistet sein.
	\item \label{itm:ticket_reservieren} Der Sitzplatz darf nur reserviert werden, wenn es für ihn noch kein Ticket gibt und er derzeit auch nicht blockiert ist.
	\item \label{itm:front_end_reservieren} Eine Information mit den erfolgreich reservierten Sitzen muss an das Front-End gesendet werden.
	\item \label{itm:zeit_reservieren} Der Zeitpunkt der Reservierung darf durch das Front-End nicht manipulierbar sein.
\end{enumerate}

\subsection{Lösungsansatz}
\label{ssec:loesung_reservieren}

Um die geforderten o.g. Punkte umzusetzen, werden in der \acs{REST}-Schnittstelle \\\jinline |reservation/book| mehrere Überprüfungen durchgeführt.

\subsubsection*{Punkt \ref{itm:zeitpunkt_reservieren} -- Zeitpunkt}
\label{ssssec:Zeitpunkt_reservieren}
Siehe \nameref{ssec:loesung_blocken} \nameref{ssssec:Zeitpunkt} auf Seite \pageref{ssssec:Zeitpunkt}.

\subsubsection*{Punkt \ref{itm:geblockt_durch_benutzer} -- Geblockt durch Benutzer, Punkt \ref{itm:mehr_eine_vorstellung_reservieren}, \ref{itm:mehr_mehrere_vorstellungen_reservieren} -- Mehrfachvergabe, Punkt \ref{itm:ticket_reservieren} -- Reservieren}
\label{ssssec:geblockt_durch_benutzer}
Um einem Benutzer das Reservieren des möglicherweise geblockten Sitzplatzes für eine Vorstellung zu gewährleisten, wurde, wie bereits erwähnt, sich dazu entschlossen einen Sessiontoken als Identifikator mitzugeben.
Da dieser vom Front-End erzeugt wird, reicht dieses den Sessiontoken beim Reservierungsvorgang im \acs{JSON} mit.
Da der Sessiontoken auch hier nach erfolgreicher Buchung nicht übermittelt werden soll, gibt es auch für diesen Anwendungsfall zwei \acp{DTO}, die wie im \nameref{ssec:loesung_blocken} Blocken in \nameref{ssssec:Bezeichner} auf Seite \pageref{ssssec:Bezeichner} auch die \acs{JSON}-Annotation, für das Schreiben bzw. Nichtschreiben des Sessiontokens, besitzen (\textit{BookingToWithSessiontoken} und \textit{BookingTo}). 

Auch hier erfolgt durch eine ausgelagerte Methode \jinline |checkIfSeatsAreBookable| die Überprüfung, ob
\begin{enumerate}
	\item es bereits ein Ticket für den zu buchenden Sitzplatz gibt.
	\item der Sitzplatz bereits geblockt ist.
	\begin{enumerate}
		\item Wenn ja, passt der Sessiontoken der aktuellen Anfrage zu diesem? Dann darf die Buchung erfolgen.
		\item Falls nicht, darf die Reservierung nicht erfolgen.
	\end{enumerate}
\end{enumerate}

War dieser Methoden-Aufruf erfolgreich, wird eine weitere Methode aufgerufen, die ein Reservierungs-\acs{DTO} als Rückgabewert hat (\jinline |createReservationForSeats|). \\
Dieses ist mit dem aktuellen Datum und der aktuellen Uhrzeit sowie einer Liste mit Ticket-\acp{DTO} versehen.
Dem Ticket-\acs{DTO} wird der aktuelle Sitzplatz zugeteilt.
Ferner wird in jedem Ticket-\acs{DTO} individuell das Attribut \jinline |isReducedPrice| gesetzt, welches zuvor durch das Front-End zu dem jeweiligen Sitzplatz über das \acs{JSON} mitgeteilt wurde.
Durch dieses Setzen des Attributs hat man im Nachhinein die Möglichkeit, die korrekte Preisberechnung zu verifizieren.

Die Implementierung, ob die ausgewählten Sitzplätze verfügbar sind, befindet sich im Anhang \vref{lst:Angang_Prüfung_ob_Reservierung_möglich}.
Die Implementierung für das Erstellen des Reservierungs-\acp{DTO} befindet sich im Anhang \vref{lst:Anhang_Erstellen_Reservierung}.

\subsubsection*{Punkt \ref{itm:front_end_reservieren} -- Erfolgreich}
\label{ssssec:front_end_reservieren}
Wurde der Reservierungsvorgang erfolgreich durchgeführt, erhält das Front-End das Reservierungs-\acs{DTO} als \acs{JSON} wie in \nameref{ssec:loesung_blocken} im \nameref{ssssec:erfolgreich_blocken} auf Seite \pageref{ssssec:erfolgreich_blocken} beschrieben, zur Auswertung zurück (vgl. Herausforderungen Punkt \vref{itm:front_end_reservieren}).
Diese wertet das \acs{JSON} aus und generiert daraus einen \acs{QR-Code}, der im Browser angezeigt wird.
Ein Beispiel für die Umsetzung seitens des Front-Ends befindet sich in Kapitel \vref{fig:vorstellung04}.
%Dieses hat wie zuvor beschrieben das Attribut \jinline |sessiontoken| nicht gesetzt.
