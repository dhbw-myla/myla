% !TEX root =  master.tex
\section{\acf{DTO}}
\label{sec:dto}

\subsection{Begriff \acf{DTO}}
\label{ssec:Was_ist_dto}

Der Begriff \ac{DTO} ist ein Entwurfsmuster, welches aus dem Bereich der Softwareentwicklung stammt.
Es wird genutzt, um mehrere Objekte in einem Programmaufruf zusammenzufassen, weshalb es Anwendung in verteilten Systemen findet.\footnote{\url{https://www.georgbeier.de/tutorials-java-und-mehr/java8-spring-groovy-vaadin/svg-architecture/data-transfer-objects/}}

Ein \ac{DTO} entspricht eigentlich dem \ac{POJO}.
Es hat nahezu die selben Attribute, kann aber nach Belieben verändert werden.
So kann man sich mittels des \ac{DTO} nur einige Attribute anzeigen lassen und die Entität bleibt unberührt.

Die \acsp{DTO} werden in diesem Projekt dazu genutzt, um die Daten aus der Datenhaltungsschicht der Drei-Schichten-Architektur in die Logik- bzw. Fachkonzeptschicht zu transferieren, denn eine Entität verlässt niemals die Datenhaltungsschicht.
Von dort aus werden sie als \acs{JSON} an die Benutzerschicht weitergeleitet.

\subsection{Umwandlung der Entität zu und von einem \acf{DTO}}
\label{ssec:umwandlung_dto}

Wie bereits beschrieben kommen in diesem Projekt \acp{DTO} zum Einsatz.
Um eine Umwandlung zu realisieren wurden zwei Java-Klassen implementiert: \jinline|EntityToToHelper| und \jinline|ToToEntityHelper|. 

Die Klasse \jinline|EntityToToHelper| ist dafür verantwortlich, die zuvor über das \acs{DAO} angefragte Entität mit all ihren Attributen in ein \acs{DTO} umzuwandeln, sodass es transferierbar ist. \\
Ein Beispiel für die Umwandlung von einer Entität in ein \acs{DTO} befindet sich im Anhang \vref{lst:EntityToToHelper_movie}. \\

Die Klasse \jinline|ToToEntityHelper| ist dafür verantwortlich das zuvor über das \acs{JSON} übermittelte \acs{DTO} mit all seinen Attributen in ein Entität umzuwandeln, sodass es persistier-, änder- oder löschbar ist.\\
Ein Beispiel für die Umwandlung von einem \acs{DTO} in eine Entität befindet sich im Anhang \vref{lst:ToToEntityHelper_movie}.
