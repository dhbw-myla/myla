\section{\acf{JSON}}
\label{sec:json}

Für den Datenaustausch zwischen Front- und Back-End wird mit der \acf{JSON} realisiert.
Grund ist, dass hier rein die Nutzdaten transferiert werden.
Ferner gestaltet sich das Auslesen und Generieren der \acs{JSON}-Dateien als recht \enquote{einfach}. \\
Ein Array bzw. eine Liste wird in \acs{JSON} mit [ ] dargestellt.
Objekte werden mit \{ \} dargestellt.
Jedes Attribut und jede Zeichenkette wird in \acs{JSON} in Anführungszeichen gesetzt (siehe Quelltext \vref{lst:example_movie}).

\begin{lstlisting}[language=json, caption={Beispiel eines Films im \acs{JSON}-Format}, label={lst:example_movie}]
{"movie": {
	"id": 1,
	"title": "Kampf der Titanen",
	"duration": 120,
	"hall": {
		"id": 1,
		"name": "Kino 1",
		"seats": [{"seat": {"id": 1, ...}}]
	},
	"shows": [{"show": {"id": 1, ...}}]
 }
}
\end{lstlisting}
