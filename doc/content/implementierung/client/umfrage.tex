% !TeX root = ../../../master.tex

\subsection{Teilnahme Umfrage}
\label{ssec:UmfrageImplement}

Wie in Abb. \vref{fig:UmfrageImplement} dargestellt, soll der Teilnehmer hier auf den zuvor erstellten Frageboge geleitet werden. 
Hier kann dieser diesen ausfüllen und nach Beendigung über einen Button abschicken. 
Der Teilnehmer erhält am Ende ein Feedback, dass seine Teilnahme erfolgreich war.

Abb. \vref{fig:UmfrageImplement}  stellt eine Umfrage zur Projektmanagement-Vorlesung dar, inder der Teilnehmer drei Fragen beantworten soll. 
Exemplarisch sind hier bei drei Fragetypen angeführt: 
\begin{itemize}
	\item Multiple Choice: Auswahl der Notationsart für zukünftige Projekte
	\item Skala: Benotung der Vorlesung auf einer Skala von 1 - 5
	\item Ja/Nein: Ob der Teilnehmer die Vorlesung erneut besuchen würde 
\end{itemize}

Über den Button \jinline|Complete| kann der Teilnehmer die Umfrage beenden. 
Er erhält ein visuelles Feedback über seine Teilnahme.

\begin{figure}[hp]
	\centering
	\includegraphics[width=0.95\textwidth, keepaspectratio]{img/client/TeilnahmeUmfrage.png}
	\captionsetup{justification=centering, format=plain}
	\caption[\acf{UI}: Umfrage basierend auf der Vorlage]{\acf{UI}: Umfrage basierend auf der Vorlage \\ \quelleScreenshot}
	\label{fig:UmfrageImplement}
\end{figure}
