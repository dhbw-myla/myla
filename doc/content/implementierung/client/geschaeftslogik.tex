% !TeX root = ../../../master.tex

\subsection{Geschäftslogik}
\label{ssec:GeschaeftslogikClient}

In diesem Projekt soll React zum Einsatz kommen, welches bereits im Grundlagenteil kurz beschrieben wurde (Kapitel~\vref{ssec:React}).
Durch die Verwendung von React wird keine hardware-spezifische Software benötigt.
Die Anwendung kann sowohl als Webanwendung als auch auf einem mobilen Gerät benutzt werden.
Dadurch wird Anforderung der endgeräteunabhängigen Nutzung (\hyperref[Anf:A1]{A1}) erfüllt.
React wurde \ua deshalb für die Entwicklung des Front-Ends ausgewählt, da es seit mehreren Jahren kontinuierlich als die beliebteste Front-End-Bibliothek gilt\autocite[Vgl.][]{stackoverflow_Top_Frameworks} und zudem verhältnismäßig leicht zu erlernen ist.
Darüber hinaus werden Komponenten verwendet, die als modulare Grundlage dienen und ein wiederholtes Schreiben des Codes verhindern sollen (Boilerplate-Code).


Für die Unterteilung der verschiedenen Komponenten wurde der Wert besonders auf eine angenehme \acf{UX} und ein benutzerfreundliches \acf{UI} gelegt.
Da die Anwendung im Rahmen eines Projektes der \acs{DHBW} Mannheim konzeptioniert wird, sollen die Komponenten gemäß Anforderung~\hyperref[Anf:A16]{A16} das entsprechende Farbschema nutzen (vgl. Kapitel~\vref{sec:konzeption:client}).
Im folgenden wird auf den Aufbau der Benutzeroberfläche (\ac{UI}) des Clients eingegangen.
