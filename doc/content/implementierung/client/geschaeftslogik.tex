% !TeX root = ../../../master.tex

\subsection{Geschäftslogik}
\label{ssec:GeschaeftslogikClient}

In diesem Projekt soll React zum Einsatz kommen, welches bereits im Grundlagenteil kurz beschrieben wurde (Kap. \vref{ssec:React}).
React wurde u.a. deshalb für die Entwicklung des Front-Ends ausgewählt, da es seit mehreren Jahren kontinuierlich an Beliebtheit gewinnt\autocite[vgl.]{stackoverflow_Top_Frameworks} und zudem verhältnismäßig leicht zu erlernen ist.
Darüber hinaus werden Komponenten verwendet, die als modulare Grundlage dienen und ein Mehrfachschreiben des Codes verhindern sollen. 

Für die Unterteilung der verschiedenen Komponenten soll auf eine spezielle \acfp{UX} und \acfp{UI} wert gelegt werden. 
Da die Applikation im Rahmen eines Projektes der DHBW Mannheim konzeptioniert wird, sollen das Front-End bzw. die Komponenten an dieses Farbschema angelehnt werden. 
Im folgenden soll auf den Aufbau der Benutzeroberfläche eingegangen werden. 