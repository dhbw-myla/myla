% !TeX root = ../../../master.tex

\subsection{Geschäftslogik}
\label{ssec:GeschaeftslogikClient}

In diesem Projekt soll React zum Einsatz kommen, welches bereits im Grundlagenteil kurz beschrieben wurde (Kap. \vref{ssec:React}).
React wurde \ua deshalb für die Entwicklung des Front-Ends ausgewählt, da es seit mehreren Jahren kontinuierlich an Beliebtheit gewinnt\autocite[vgl.][]{stackoverflow_Top_Frameworks} und zudem verhältnismäßig leicht zu erlernen ist.
Darüber hinaus werden Komponenten verwendet, die als modulare Grundlage dienen und ein wiederholtes Schreiben des Codes verhindern sollen (Boilerpalte-Code). 

Für die Unterteilung der verschiedenen Komponenten soll auf eine spezielle \ac{UX} und \ac{UI} wert gelegt werden. 
Da die Applikation im Rahmen eines Projektes der DHBW Mannheim konzeptioniert wird, sollen das Front-End bzw. die Komponenten wie bereits in der Konzeption erwähnt an dieses Farbschema angelehnt werden (vgl. Kap. \vref{sec:konzeption:client}). 
Im folgenden soll auf den Aufbau der Benutzeroberfläche (\ac{UI}) des Clients eingegangen werden. 