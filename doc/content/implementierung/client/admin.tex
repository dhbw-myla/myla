% !TeX root = ../../../master.tex

\subsection{Administrationsbereich}
\label{ssec:Administrationsbereich}

Zur Verwaltung und Pflege einer Anwendung wird ein Administrator benötigt.
Ob ein Benutzer ein Administrator ist, erkennt er in der Navigationsleiste der Anwendung.
Sofern dies zutrifft, ist der Bereich \emph{Admin} dargestellt (siehe Abbildung~\vref{fig:AdministrationsoberflaecheImplement} oben rechts). \newline
Die dem Administrator zur Verfügung stehenden Funktionen sind folgende:
%
\begin{itemize}
    \item Registrierungsschlüssel-Verwaltung \faKey\xspace (einsehen und ändern),
	\item Benutzerverwaltung \faUsers.
\end{itemize}
%
\begin{figure}[h]
	\centering
	\includegraphics[width=0.95\textwidth, keepaspectratio]{img/client/Admin.png}
	\captionsetup{justification=centering, format=plain}
	\caption[\acl{UI}: Administrationsoberfläche]{\acl{UI}: Administrationsoberfläche \\ \quelleScreenshot}
	\label{fig:AdministrationsoberflaecheImplement}
\end{figure}
%
\subsubsection*{Registrierungsschlüssel \faKey}

Über den Knopf \jinline|Edit| kann der Administrator den Registrierungsschlüssel ändern, um kompromittierte Schlüssel zu ersetzen (siehe Abbildung~\vref{fig:AdminEditRegKeyImplement}).
Über den Knopf \jinline|Show|\xspace kann der aktuelle Registrierungsschlüssel ausgelesen werden.

\begin{figure}[h]
	\centering
	\includegraphics[width=0.95\textwidth, keepaspectratio]{img/client/EditSurveyMasterKey.png}
	\captionsetup{justification=centering, format=plain}
	\caption[\acl{UI}: Setzen eines neuen Registrierungsschlüssels]{\acl{UI}: Setzen eines neuen Registrierungsschlüssels \\ \quelleScreenshot}
	\label{fig:AdminEditRegKeyImplement}
\end{figure}


\subsubsection*{Neuen Benutzer hinzufügen \faUsers}

Darüber hinaus kann er über den Button \jinline|Create| einen neuen Benutzer dem System anfügen. 
\abb \myRefGeneral{fig:AdminCreateUserImplement} zeigt die Eingabeform. 
Der Administrator muss hier den Benutzername und ein zufälliges Passwort wählen. 
Erstellte Benutzer durch den Administrator oder beim Zurücksetzen eines Passwortes erhalten als Sicherheitsfeature seitens der Datenbank \todo{Ref Niko, Ref Sicherheit FE} den Eintrag \jinline|isPasswordChangeRequiered|. 
Dies bedeutet, dass beim Einloggen der Benutzer auf die Seite zum Ändern des Passwortes geleitet wird und sich ansonsten nicht im System bewegen kann, bis er sein Passwort geändert hat. 

Dadurch wird Anforderung~\ref{Anf:A5}, die Pflicht zur Passwortänderung bei manueller Registrierung, erfüllt.

\begin{figure}[hp]
	\centering
	\includegraphics[width=0.95\textwidth, keepaspectratio]{img/client/AdminCreateUser.png}
	\captionsetup{justification=centering, format=plain}
	\caption[\acf{UI}: Neuen Benutzer der Anwendung hinzufügen]{\acf{UI}: Neuen Benutzer der Anwendung hinzufügen \\ \quelleScreenshot}
	\label{fig:AdminCreateUserImplement}
\end{figure}



\subsubsection*{Show users \faUsers}

Der Administrator erhält zur Benutzerverwaltung, wie in Abbildung~\vref{fig:AdminShowUsersImplement} dargestellt, eine Übersicht über alle im System vorhandenen Benutzer.
Zur erleichterten Suche von Benutzern wurde eine Suchleiste in der Übersicht hinzugefügt.
Diese ermöglicht das Suchen von Benutzernamen oder das Auswählen des Benutzers über ein Dropdown-Menü.
Darüber hinaus erfährt der Administrator den aktuellen Status der in der Anwendung befindlichen Benutzer, wie etwa den Administratorenstatus.

\begin{figure}[H]
	\centering
	\includegraphics[width=0.95\textwidth, keepaspectratio]{img/client/AdminShowUsers.png}
	\captionsetup{justification=centering, format=plain}
	\caption[\acl{UI}: Benutzerübersicht]{\acl{UI}: Benutzerübersicht \\ \quelleScreenshot}
	\label{fig:AdminShowUsersImplement}
\end{figure}

Der eigene Nutzer wird dabei besonders als \faUser\xspace hervorgehoben, wohingegen andere Benutzer lediglich durch das Icon \faUser[regular]\xspace (Silhouette) dargestellt werden. \newline
Zudem wurde hier das Zurücksetzen des Passworts des Benutzers implementiert, da Benutzer ihr Passwort regelmäßig vergessen.\autocite[Vgl.][]{statistaPasswortVergessen}
Dadurch wird Anforderung~\hyperref[Anf:A6]{A6}, die Pflicht zur Passwortänderung bei zurückgesetztem Passwort, umgesetzt.
Abbildung~\vref{fig:AdminSetNewPasswordImplement} stellt das Zurücksetzen des Passworts über ein Pop-up dar.

\begin{figure}[!htb]
	\centering
	\includegraphics[width=0.95\textwidth, keepaspectratio]{img/client/AdmiSetPasswordOfUser.png}
	\captionsetup{justification=centering, format=plain}
	\caption[\acl{UI}: Zurücksetzen des Passworts]{\acl{UI}: Zurücksetzen des Passworts \\ \quelleScreenshot}
	\label{fig:AdminSetNewPasswordImplement}
\end{figure}

Ein weiteres Feature der Anwendung ist das Ernennen von Administratoren.
Dies geschieht über den Knopf \jinline|Promote To Admin|, welcher in Abbildung~\vref{fig:AdminPromoteToAdminImplement} dargestellt ist.
Zusätzlich dazu muss der Nutzer diese Beförderung über einen weiteren Dialog bestätigen, um Fehler zu vermeiden.
Visuell werden Administratoren hervorgehoben, da der Knopf \jinline|Promote To Admin| durch \jinline|Is Admin| ersetzt wird.

\begin{figure}[!htb]
	\centering
	\includegraphics[width=0.95\textwidth, keepaspectratio]{img/client/AdminPromoteToAdmin.png}
	\captionsetup{justification=centering, format=plain}
	\caption[\acl{UI}: Administrator-Ernennung]{\acl{UI}: Administrator-Ernennung \\ \quelleScreenshot}
	\label{fig:AdminPromoteToAdminImplement}
\end{figure}


