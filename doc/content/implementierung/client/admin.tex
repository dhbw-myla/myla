% !TeX root = ../../../master.tex

\subsection{Administrationsbereich}
\label{ssec:Administrationsbereich}

Geplant ist, dass die Anwendung einen Administrationsbereich erhält, die es ermöglichen soll: 
\begin{itemize}
    \item Den Registierungsschlüssel einzusehen und zu ändern,
    \item Einen neuen Benutzer anzulegen,
    \item Sowie alle Benutzer in der Anwendung anzuzeigen. 
\end{itemize}

Abb. \myRefGeneral{fig:MockAdministrationsbereich} stellt das Mockup des Administrationsbereich dar. 
Über einen Button soll der Administrator dem Registierungsschlüssel ändern können, sodass im Falle eines \emph{Leaks} des Registierungsschlüssels ein Registrieren eines neuen Benutzers nicht mehr möglich ist. 
Darüber hinaus soll der Administrator ebenfalls neue Benutzer anlegen können sowie alle Benutzer in der Anwendung anzeigen können. 

Hier soll es wie in Abb. \myRefGeneral{MockMakeAdmin} dargestellt wird, soll der Administrator die Möglichkeit haben, das Passwort eines Benutzer neu zu setzen im Falle eines Vergessesens des Benutzers.
Des weitern soll auch ein Benutzer zu einem Administrator upgegraded werden. 

\begin{figure}
	\missingfigure{Administrationsbereich}
	\caption[Mock: Administrationsbereich]{Mock: Administrationsbereich \\ \quelle}
	\label{fig:MockAdministrationsbereich}
\end{figure}

\begin{figure}
	\missingfigure{MockMakeAdmin}
	\caption[Mock: MockMakeAdmin]{Mock: MockMakeAdmin \\ \quelle}
	\label{fig:MockMakeAdmin}
\end{figure}