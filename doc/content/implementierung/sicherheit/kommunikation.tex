% !TEX root =  ../sicherheit.tex
\subsection{Kommunikation}


% VORSCHLAG - Rene aus BA:


\subsubsection{Transferprotokoll}\label{chapter:https}

Zur erhöhten Sicherheit bei der Kommunikation werden Anfragen ausschließlich über \ac{HTTPS} zugelassen, wodurch die Kommunikation zwischen Server und Client verschlüsselt wird.
Dies erhöht die Vertraulichkeit der Daten und stellt deren Integrität sicher.
Diese zwei Schutzziele dienen dazu einen \ac*{MITM}-Angriff oder einen Lauschangriff zu verhindern, da der Inhalt der Kommunikationspakete nicht mitgelesen werden kann und eine Veränderung der Pakete durch eine fehlerhafte Ent- und erneute Verschlüsselung bemerkt wird.

\ac{HTTPS} verwendet dabei \ac{TLS} zur Verschlüsselung und unterscheidet sich, außer über die \ac{URL}, welche bei der Verwendung von \ac{HTTPS} mit \enquote{https://} beginnt, und die standardmäßige Nutzung des Port 443, nicht weiter von \ac{HTTP}.

\ac{TLS} setzt auf eine asymmetrische Verschlüsselung, welche zunächst einen Schlüsselaustausch zwischen Client und Server vorsieht.
Dies geschieht nach der Version \ac{TLS} 1.3 über drei verschiedene Wege, welche jeweils aufgrund ihrer gewährleisteten Zukunftssicherheit ausgewählt wurden.
Diese Zukunftssicherheit dient dazu, dass nicht alle potentiell von einem Angreifer abgefangenen \ac{TLS}-Sessions entschlüsselt werden können, sobald der private Schlüssel einer Session gefunden wurde.

Im Rahmen der Arbeit soll aus diesem Grund \ac{TLS} 1.3 zur Sicherung der \ac{API} verwendet werden, sodass die sensiblen Daten des \ac{ERP}-Systems nicht durch einen potentiellen Angriff an Dritte weitergegeben werden.\autocite{rf-RFC8446}
