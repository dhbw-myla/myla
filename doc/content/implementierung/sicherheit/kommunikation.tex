% !TEX root =  ../sicherheit.tex
\subsection{Kommunikation}

\subsubsection{Transferprotokoll}\label{chapter:https}

Um die Sicherheit während der Kommunikation zu erhöhen dürfen Anfragen an den Webserver der Anwendung ausschließlich über \ac{HTTPS} zugelassen werden, sodass  die Kommunikation zwischen Server und Client verschlüsselt wird.
Dies stellt die Integrität der Daten sicher und erhöht damit ebenfalls ihre Vertraulichkeit.
Diese zwei Schutzziele dienen dazu einen potentiellen Angriff, wie etwa einen \ac*{MITM}-Angriff oder einen Lauschangriff, zu verhindern, indem das Ausspähen des Inhalts der Kommunikationspakete durch die vorhandene Verschlüsselung verhindert wird oder etwaige Veränderung der Pakete durch eine fehlerhafte Ent- und erneute Verschlüsselung bemerkt wird.

Zur Verschlüsselung verwendet \ac{HTTPS} die \ac{TLS} und unterscheidet sich grundlegend, neben der \ac{URL}, welche bei der Verwendung von \ac{HTTPS} mit \enquote{https://} beginnt, und der standardmäßigen Nutzung des Port 443, nicht weiter von \ac{HTTP}.

\ac{TLS} setzt dabei auf eine asymmetrische Verschlüsselung, wobei im ersten Schritt ein Schlüsselaustausch zwischen Client und Server vorgesehen ist.
Dies geschieht nach der Version \ac{TLS} 1.3 über drei verschiedene Wege, die aufgrund ihrer Zukunftssicherheit ausgewählt wurden.
Diese Zukunftssicherheit dient dazu, auch langfristige Angriffe zu verhindern, da so nicht alle potentiell von einem Angreifer abgefangenen \ac{TLS}-Sessions entschlüsselt werden können, sobald der private Schlüssel einer Session gefunden wurde.\autocite{rf-RFC8446}
