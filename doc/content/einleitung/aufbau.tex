% !TEX root =  ../../master.tex
\section{Aufbau}

\textit{\textbf{Bemerkung der Autoren:}
Diese Arbeit bzw. dieses Dokument erhebt nicht den Anspruch einer wissenschaftlichen Arbeit, sondern spiegelt lediglich den Entwicklungsprozess der erstellten Software wieder. 
Hierzu werden entsprechende Grundlagen vermittelt.}

Diese werden im nächsten Kapitel vorgestellt.
Zunächst werden hierbei die theoretischen Grundlagen in Abschnitt~\vref{sec:theoretisch} beschrieben.
Zu Beginn wird hier das Programmierparadigma \acs{REST} vorgestellt.
Anschließend daran folgen Ausführungen zu den Techniken hinter Microservices sowie der Methodik des Prototyping.
Ebenso werden Grundlagen zu relationalen Datenbanken, die in dieser Arbeit Anwendung finden, kurz erläutert.
Danach wird die Theorie zum Erstellen von Fragebögen behandelt.
Abschließend wird auf allgemeine Aspekte von Learning Analytics eingegangen. 

Im Folgenden werden in Abschnitt~\vref{sec:technisch} die technischen Grundlagen gelegt, um ein Verständnis für die verwendeten Technologien zu schaffen. 
Den Anfang bildet die Technologie Docker.
Dabei wird insbesondere auf die Container-Virtualisierung eingegangen.
Um einen Bezug zu den theoretischen Grundlagen der relationalen Datenbanken zu schaffen, bildet der Abschnitt über PostgreSQL ein reales Anwendungsbeispiel für eben diese.
Schlussendlich werden die Grundlagen mit einem Einblick in das verwendete Front-End-Framework React abgeschlossen. 

Darauf folgend wird im Kapitel~\vref{ch:analyse} eine Ist-Analyse der bestehenden Learning-Analytics-Software durchgeführt. 
Dabei wird näher darauf eingegangen, inwiefern die vorherige Lösung in die neue Implementierung eines Umfragetools für Studierende und Dozierende mit einwirken kann.
Um einen klar abgegrenzten Start in die Implementierung zu legen, werden die Anforderungen an die neue Software detailliert aufgeführt. 
Nachdem die Anforderungen evaluiert sind, werden alternative bestehende Lösungen analysiert und gegenüber der im Nachgang zu entwickelnden Software abgegrenzt.
Zum Abschluss der Analyse werden drei Personas als mögliche Nutzer der Software vorgestellt, um basierend auf diesen ein Konzept für die Anwendung zu entwickeln.

In Kapitel~\vref{ch:konzeption} wird auf die Konzeption eingegangen. 
Zuerst wird die grundlegende Architektur der Software beschrieben.
Im Speziellen wird geklärt, inwiefern die Bestandteile der Software miteinander interagieren sollen.
Anschließend werden die einzelnen Komponenten, Server und Client, konzeptionell beschreiben.
Es wird dabei noch einmal explizit auf die Sicherheit der Anwendung eingegangen.

Nach der ausführlichen Erläuterung der Konzeption wird in Kapitel~\vref{ch:implementierung} die Implementierung der Software-Bestandteile beschrieben. 
Ebenfalls wird auf das verwendete Docker-Netzwerk eingegangen, welches diese miteinander verbindet und die Kommunikation in einem abgesicherten System ermöglicht.
Die Implementierung orientiert sich dabei an den zuvor konzeptionierten Ideen.
Jedoch sind im Verlauf des Projektes einige Abweichungen entstanden.
Überwiegend ist dies in den Erfahrungen, die bei der Entwicklung gesammelt wurden, begründet.
Darüber hinaus konnten mangels Zeit nicht alle geplanten Features vollständig eingearbeitet werden.

Nach der erfolgreichen Implementierung wird in Kapitel~\vref{ch:nutzerhandbuch} ein Nutzerhandbuch vorgestellt. 
In diesem werden die wesentlichen Nutzungsaspekte im Detail erläutert.
Dabei wird beispielsweise beschrieben, wie Umfrageersteller sich bei der Software registrieren und anmelden können.

Die Dokumentation wird in Kapitel~\vref{ch:ausblick} zum Abschluss gebracht. 
Dabei wird ein Fazit zur Gruppen- und Seminararbeit gezogen.
Hierbei wird insbesondere Bezug auf das Ergebnis, also die entwickelte Software zur Durchführung von Umfragen, genommen.
Zuletzt werden im Ausblick dieser Arbeit die möglichen Aspekte für eine Weiterentwicklung beleuchtet.
