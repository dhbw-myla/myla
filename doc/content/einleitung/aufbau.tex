% !TEX root =  ../../master.tex
\section{Aufbau}

In der nachfolgenden Dokumentation werden diverse theoretische und technische Grundlagen vermittelt.
Dabei wird im theoretischen Part genauer darauf eingegangen, was man unter Representational State Transfer versteht, was ein Microservice ist, wie ein Fragebogen verwirklicht wird, was eine relationale Datenbank kennzeichnet und wodurch der Bereich der Learning Analytics charakterisiert ist.
In den technischen Grundlagen wird die Containervirtualisierungssoftware Docker, das relationale Datenbankmanagementsystem PostgreSQL, sowie die Frontend-Softwarebibliothek React erläutert.
 
Die gestellten Anforderungen werden analysiert und wiedergegeben. Zusätzlich werden User Stories dargestellt.

Nachfolgend wird auf die Konzeption eingegangen, wobei das Gesamtkonzept, dessen Architektur, Server und Clients spezifischer dargestellt werden.
Anschließend werden Personas, die User-Journey, sowie Server und Clients unabhängig vom Gesamtkonzept dargelegt.
Die Sicherheit wird noch einmal im speziellen wiedergegeben, wobei dabei auf die Aspekte der Authentifizierung, den Datenschutz sowie die Art der Verschlüsselung eingegangen wird.

Das folgende Kapitel beschreibt die Implementierung.
Dabei wird die Verwirklichung der Serverarchitektur mittels Geschäftslogik, Datenhaltung und Schnittstellen veranschaulicht.
Nachfolgend wird die Implementierung des Clients beschrieben.
Den Abschluss des Kapitels stellt das Sicherheitskonzept dar, wobei die Zugriffskontrolle, die Art der Kommunikation sowie die Validierung der vorherrschenden Sicherheit spezifiziert werden.

Die Dokumentation wird mittels Ausblick zu einem Abschluss gebracht, wobei dieser auf ein vorangehendes Fazit zur Gruppen- und Seminararbeit folgt.
