% !TEX root =  ../../master.tex
\section{Motivation und Hintergrund}

In der heutigen digitalisierten Welt sind Online-Umfragen ein bewährtes Mittel der Sozialforschung.
Das beste Beispiel hierfür ist das Unternehmen Alphabet Inc., welches durch seine Google-Umfrage-Applikation Android-Nutzern regelmäßig Umfragen unterbreitet und damit aktive Sozialforschung betreibt.
Damit auch die \ac{DHBW} Mannheim eine Möglichkeit besitzt, eigene Online-Umfragen ohne die Nutzung von Drittanbieter-Software zu erstellen, um Daten für entsprechende Sozialforschungen zu erheben

Motivation
- Dokumentation für Gruppenarbeit
- Implementierungsprozess von Umfragetool
- persönliche Weiterentwicklung auf Erfahrungsgewinn für weitere Projekte in der berufelichen Laufbahn

Hintergrund
- es wird ein Umfragetool benötigt, welches standardisiert eingesetzt werden kann und die Datenspeicherung kontrolliert werden kann
- 

Diese Dokumentation beschreibt die im Umfang des Moduls Integrationsseminar entwickelte Umfrageapplikation MyLA der Gruppe um Sascha Görnert, Rene Fischer, Erik Jansky, Niko Lockenvitz, Martin Sandig und Julian Rolle.
