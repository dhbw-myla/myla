% !TEX root =  ../../master.tex
\section{Motivation und Zielsetzung}

In der heutigen digitalisierten Welt sind Online-Umfragen ein bewährtes Mittel der Sozialforschung.
Das beste Beispiel hierfür ist das Unternehmen Alphabet Inc., welches durch seine Google-Umfrage-Applikation Android-Nutzern regelmäßig Umfragen unterbreitet und damit nicht nur seine eigene Produkte, sondern auch aktive Sozialforschung betreibt.
Damit auch die \acs{DHBW} Mannheim eine Möglichkeit besitzt, eigene Online-Umfragen ohne die Nutzung von Drittanbieter-Software zu erstellen, um Daten für entsprechende Sozialforschungen zu erheben, soll im Rahmen dieser Seminararbeit eine solche Anwendung entwickelt werden.

Dabei liegt die Zielsetzung in der Gestaltung und Umsetzung einer Applikation, mit der verschiedene Umfragen personalisiert erstellt und ausgewertet werden können.


Zielsetzung der Anwendung ist es, Dozenten zu ermöglichen mittels Umfragen den Studenten eine Möglichkeit zu bieten Gelerntes zu verinnerlichen und Anmerkungen zu den gestellten Fragen zu geben.
Dem Dozenten wird damit eine Plattform geboten, solche Umfragen in verschiedenen Kursen zu wiederholen und zu gruppieren und die Ergebnisse der Umfrage betrachten zu können, womit der Dozent auf spezifische Fragen und Probleme gezielter eingehen kann und es ermöglicht wird den Lernfortschritt der Studenten überblicken zu können.
 
%Niko macht nichts und ist hier afk!

Motivation
- Dokumentation für Gruppenarbeit
- Implementierungsprozess von Umfragetool
- persönliche Weiterentwicklung auf Erfahrungsgewinn für weitere Projekte in der berufelichen Laufbahn

Hintergrund
- es wird ein Umfragetool benötigt, welches standardisiert eingesetzt werden kann und die Datenspeicherung kontrolliert werden kann
- 

Zielsetzung
-  

Diese Dokumentation beschreibt die im Umfang des Moduls Integrationsseminar entwickelte Umfrageapplikation MyLA der Gruppe um Sascha Görnert, Rene Fischer, Erik Jansky, Niko Lockenvitz, Martin Sandig und Julian Rolle.
