% !TEX root =  ../../master.tex
\section{Motivation und Zielsetzung}

In der heutigen digitalisierten Welt sind Online-Umfragen ein bewährtes Mittel der Sozialforschung.
Das beste Beispiel hierfür ist das Unternehmen Alphabet Inc., welches durch seine Google-Umfrage-Applikation Android-Nutzern regelmäßig Umfragen unterbreitet und damit nicht nur die eigenen Produkte, wie YouTube, zu verbessern, sondern auch Werbevorschläge zu optimieren.
Auch für die \acs{DHBW} Mannheim ist eine solche Möglichkeit zur Sozialforschung von großem Nutzen.
Aus diesem Grund soll im Rahmen dieser Seminararbeit eine eigene Anwendung für Online-Umfragen entwickelt werden, sodass die \acs{DHBW} in Zukunft ohne die Nutzung von Drittanbieter-Software auskommen kann.
Damit soll gewährleistet werden, dass alle erhobenen Daten lediglich zur Sozialforschung der \acs{DHBW} Mannheim verwendet werden.

So liegt das Ziel der Arbeit auf der Gestaltung und Umsetzung einer Webanwendung, die es ermöglicht Umfragen zu erstellen, zu verwalten und auszuwerten.
Dabei sollen Dozenten mithilfe der Anwendung die Möglichkeit haben, Umfragen für ihre Studierenden zu erstellen.
Diese haben dadurch die Chance, ihre Meinung zur Vorlesung mitzuteilen, Gelerntes zu verinnerlichen oder auch um Anmerkungen zu gestellten Fragen zu geben.
Die zu entwickelnde Plattform soll dabei auch die Möglichkeit bieten, solche Umfragen in verschiedenen Kursen zu wiederholen, zu gruppieren und die Ergebnisse der Umfragen betrachten zu können.
Auf spezifische Fragen und Probleme soll somit gezielter eingegangen werden können.
Außerdem soll damit der Lernfortschritt der Studierenden überblickt werden.
Grundlegend dient die Umfrage-Anwendung ebenfalls als Datengrundlage für den großen Bereich der Learning Analytics und soll die Datenerhebung dazu erleichtern.


Diese Dokumentation beschreibt die im Umfang des Moduls Integrationsseminar entwickelte Umfrageapplikation \enquote{MyLA} der Gruppe um Sascha Görnert (DER), Rene-Pascal Fischer (TUK), Erik Jansky (SAP), Niko Lockenvitz (SAP) sowie Martin Sandig (SAP).
Julian Rolle (SAP) steht auch auf dieser Arbeit.
