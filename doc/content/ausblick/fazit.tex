% !TEX root =  ../../master.tex
\section{Fazit zur Gruppen- und Seminararbeit}
\label{sec:Fazit}

Bei der Arbeit handelt es sich um die Dokumentation der Verwendung und Neuentwicklung einer Webanwendung.
Von den verwendten Grundlagen, sowohl aus technischer als auch theoretischer Sicht, bis hin zu einem Nutzerhandbuch, wurden alle Schritte zur Erstellung der Anwendung beschrieben.
Die einzelnen Schritte der Konzeption und Implementierung wurden ausführlich erläutert, um die Überlegungen hinter den einzelnen Entscheidungen zu begründen.

Im Rahmen der Seminararbeit wurde eine Webanwendung zur Umfrageerstellung, -verwaltung und -ausführung erstellt.
Dazu wurden zunächst alle verwendeten Grundlagen, sowohl aus technischer als auch theoretischer Sicht, erläutert.
Anschließend wurde die Konzeption der Webanwendung beschrieben, dabei wurde die Client-Server-Architektur näher erläutert.
Dazu wurde im Bereich des Servers auf die Gestaltung der zugrundeliegenden Datenbank und der Back-End-\acp{API} in einer \ac{REST}-Struktur eingegangen.
Zudem wurden erste Mock-Ups der Benutzeroberfläche beschrieben, die eine grobe Orientierung für die Implementierung geliefert haben.
Weiterhin wurden wesentliche Punkte bezüglich sicherheitskritscher Aspekte, nämlich der Authentifizierung sowie der Autorisierung und dem zu verwendeden Transferprotokoll, verdeutlicht.
Dem folgt die Implementierung, welche die zuvor geschriebene Konzeption mit Hilfe der Front-End-Bibliothek \emph{React} zur Visualisierung und der JavaScript-Laufzeitumgebung \emph{Node.js} und dem Framework \emph{Express} verwirklicht wurden.

Dadurch konnten alle zu Beginn der Arbeit beschriebenen Anforderungen erfüllt werden.
Mit Hilfe der erstellten Software, welche durch Docker einfach bereitzustellen ist, können Umfragen erstellt, verwaltet und ausgewertet werden.
Gleichzeitig können Probanten über die angebotene Plattform an einer Umfrage teilnehmen.