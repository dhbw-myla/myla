% !TEX root =  ../../master.tex
\section{Ausblick}
\label{sec:Ausblick}

Die Anwendung hat zum aktuellen Stand noch ein großes Potenzial für Erweiterungen und Optimierungen.
Sie zeigt derzeit einen soliden Stand mit dem Hauptaugenmerk des Erstellens von Umfragen von verschiedenen Benutzern.

Es erscheint sinnvoll, der Applikation eine Art Forum, ähnlich einem Diskussionsforum, anzufügen.
Hintergrund ist, dass \zb Dozenten wie \dutzi ein Forumseintrag erstellen können, um hier Diskussionen zu führen.
So könnten die Teilnehmer aus dem Kurs wie \weigert über die Fragestellung diskutieren.
Andererseits könnten sich Studierende dort auch über verschiedene Themenbereiche zur Prüfungsvorbereitung austauschen und hier Dokumente teilen. \newline
Um dies zu realisieren, müssten Datenbankanpassungen vorgenommen werden, damit das Forum dargestellt werden kann und auch Studenten einen Benutzer mit Systemzugang erhalten.

% - Back-End hat bereits weitere Funktionen implementiert, die nun nach und nach im Front-End nachgezogen werden können
Das Back-End wurde bereits um eine Vielzahl an Funktionen ergänzt, welche lediglich im Front-End nachgezogen werden müssen.
Dazu zählen, dass:
%
\begin{itemize}
    \item Umfragevorlagen \zb nach Vorlesungen gruppiert,
    \item Kommentare zu Umfragen hinzugefügt,
    \item Start- und Endzeitpunkt von Umfragen festgelegt
    \item und Umfragevorlagen und Fragen öffentlich als Template für andere zur Verfügung gestellt
\end{itemize}
%
werden können.

Die Anwendung wurde ganzheitlich in englischer Sprache umgesetzt.
Sie könnte mittels Internationalisierung effizient um die Unterstützung weiterer Sprachen erweitert werden.
Dazu muss lediglich für jede Sprache eine Datei angelegt werden, die die Übersetzungen der einzelnen verwendeten Textbausteine auflistet.
Neben einem manuellen Wechsel der Sprache kann hier standardmäßig die vom Benutzer präferierte Sprache angezeigt werden, wenn dieser seinen Webbrowser bzw. sein System entsprechend eingestellt hat.

Zur Optimierung der Auswertung wäre es von Nutzen, Umfrageergebnisse, die auf derselben Vorlage beruhen, miteinander zu vergleichen.
Dies umfasst auf der einen Seite die Darstellung des zeitlichen Verlaufs von Umfrageergebnissen, die wiederholt in einem Kurs durchgeführt werden (Panel).
Auf der anderen Seite können ebenfalls mehrere Kurse untereinander verglichen werden.

Ein zusätzliches Feature der Anwendung wäre das Speichern und Exportieren der Resultate zur weiteren Verarbeitung im \acs{CSV}- oder \acs{PDF}-Format.

Des Weiteren könnte eine Ergänzung weitere Informationen rund um die Umfrage und deren Resultate beinhalten.
So könnte etwa ein Stichprobenrechner hilfreich sein, um den benötigten Stichprobenumfang einer Umfrage zu errechnen, sodass die Umfrage als repräsentativ im gewünschten Fall zu bewerten ist.
Dabei werden Kennzahlen wie die Größe der Grundgesamtheit, das Konfidenzniveau und die Fehlermarge als Errechnungsgrundlage benötigt.
In diesem Zusammenhang können auch direkt die Auswertungsoptionen im Front-End um weitere Diagrammtypen erweitert werden.
