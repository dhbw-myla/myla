% !TEX root =  ../../master.tex
\section{Ausblick}
\label{sec:Ausblick}

Die Applikation hat noch nicht ihren vollen Umfang erreicht.
Sie zeigt derzeit einen soliden Stand mit dem Hauptaugenmerk des Erstellen von Umfragen von verschiedenen Benutzern.

\subsubsection*{Forum}
Sinnvoll wäre es, eine Art Forum, ähnlich einem Diskussionsforum der Applikation, anzufügen.
Hintergrund ist, dass \zb Dozenzen wie \dutzi ein Forumseintrag machen können, um hier Diskussionen zu führen.
So könnten die Teilnehmer aus dem Kurs wie \weigert über die Fragestellung diskutieren.
Andererseits könnten sich Studierende dort auch über verschiedene Themenbereich zur Prüfungsvorbereitung austauschen und hier Dokumente teilen. \newline
Um dies zu realisieren, müssten Datenbankanpassungen vorgenommen werden, um das Forum darzustellen und auch weitere Benutzer als Student dem System hinzufügen.
Wichtig hierbei ist, dass es zu jedem Forum mehrere Forumseinträge geben kann.
Jeder Forumseintrag wird von einem Benutzer verfasst. \newline
So kann \dutzi sehen ob \weigert einen Eintrag gestellt hat.
Gegebenenfalls könnten \dutzi und \weigert auch direkt miteinander schreiben.

\subsubsection*{Individual-Chat \& Gruppenchat}
Ein interessantes Themenfeld ist das Thema Chat in einer Anwendung.
So ist vorstellbar, dass die Anwendung über eine Chatfunktion verfügt.
Studierende könnten sich hier in Gruppenchats oder per Direktchat austauschen.
Ein weiteres denkbares Feature ist der 1:1-Chat von Dozierenden und Studierenden.
So könnte sich \weigert mit seinen Fragen an \dutzi wenden, um kleinere Fragen zu stellen. \newline
Hierfür sollten die Benutzer über einen geeigneten Benutzernamen verfügen, um auch bei einer Abwesenheit auf seine Chatnachrichten zuzugreifen.
