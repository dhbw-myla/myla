% !TEX root =  master.tex
\section{Kritische Reflexion und Bewertung}
\multipleauthorsection{\authorRF}{\authorEJ}

\subsection{Funktionalität und Qualität der Software}

\subsubsection*{Front-End}
Das Front-End wurde erfolgreich nach den eigenen Wünschen mit Hilfe der zuvor festgelegten Iterationsschritte umgesetzt.
Darunter fällt die erfolgreiche Anbindung des Front-Ends an das Back-End sowie die komplette Umsetzung der Mockups, sodass die User-Journey wie angedacht komplett durchlaufen werden kann.\\
Das gegenwärtige Design hat noch Verbesserungspotenzial im Hinblick auf die farbliche und formtechnische Konsistenz, die aufgrund des knappen Bearbeitungszeitraumes aber nicht höchste Priorität hatte.
Dies ist vor allem an der Sitzplatzauswahl zu erkennen, da die Darstellung nicht mit dem Design der sonstigen Website übereinstimmt.
Jedoch steht dies in keinem Verhältnis zu den Besonderheiten der Sitzplatzauswahl, da diese durch die Anordnung in Form eines Koordinatensystems deutlich über die Erwartungen hinaus steigt.
Zudem wurde auf der Startseite des Kinos, der Filmübersicht, ein Kontent-Karussell implementiert, um dem Nutzer eine Übersicht der aktuellen Blockbuster zu präsentieren.
Hierbei können nur Bilder mit festgelegten Dimensionen eingesetzt werden, da sich sonst beim Wechsel der Inhalt verschiebt und das Nutzerlebnis beeinträchtigt.
Die Bilder der Filme allgemein können bisher nur in einem festgelegten Format (JPG) gespeichert werden, da sie sonst nicht referenziert oder dargestellt werden können.
Generell sind die Probleme des Front-Ends lediglich marginale Restriktionen, welche bereits im Backlog als zukünftige Verbesserung aufgenommen wurden und beeinträchtigen die Funktionsweise des Systems in keiner Weise.

\subsubsection*{Back-End}
Das Back-End wurde ebenso mit Hilfe der Iterationsschritte erfolgreich nach den eigenen Wünschen umgesetzt.
Dazu zählen die erfolgreiche Implementierung von \acs{REST}-Services sowie eine Umsetzung der als Vorgabe deklarierten Blockade bereits gebuchter Sitze.
Auch, wenn das Back-End die vorgegebenen und eigenen Ziele überschreitet, ist es notwendig die Arbeit konstruktiv zu kritisieren.
So ist die Komplexität des Back-Ends, durch die Verwendung einer Datenbank und der dazugehörigen Drei-Schichten-Architektur mit \acsp{DTO} sowie Entitäten, weitaus größer als bei einem vergleichbaren simplen Modell, was ursprünglich gefodert war.
Daraus ergeben sich sowohl Vor- als auch Nachteile, welche auf der einen Seite ein weitaus fortschrittlicheres Kinoreserverierungssystem erlauben, auf der anderen Seite aber viele Ressourcen gefodert haben es zu entwickeln.
Zudem wurden diverse eigene Methoden entwickelt, die unter anderem das Konvertieren in ein \acs{JSON}-Objekt ermöglichen.
Auch durch diese Methoden liegt die Systemfunktionalität des Back-Ends weit über den Vorgaben und überschreitet die selbst gesetzten Ziele.

\subsubsection*{Zusammenfassung}
Letztendlich ist zu sagen, dass die Funktionalität und Qualität der Software für den aktuellen Entwicklungsstand sowie den zurückliegenden Bearbeitungszeitraum überaus positiv zu betrachten ist.
Dies ist aufgrund der unzähligen Arbeitsstunden Einzelner und des gesamten Teams erst ermöglicht worden.

\subsection{Zusammenarbeit im Team}

Die Gruppenarbeit ist allgemein positiv zu werten.
So konnten alle Beteiligten Einblicke in möglicherweise unbekannte Bereiche gewinnen oder sich weiterbilden und den Kenntnisstand weiter ausbauen.

Die Absicht sich auch noch über die Vorgaben der Zielsetzung hinaus weiterzubilden führte dazu, dass ein großer Anteil der Gruppenarbeit auf individueller Basis entstanden ist und die Projektarbeit auf diesen aktuellen Stand gehoben hat.
Das Projekt besteht aus einem voll funktionalen Back-End mit einer Datenbank zur Datenhaltung inklusiver der Verwendung von Webservices sowie ein Front-End mit diversen nicht vorgegebenen Features.
Diese beiden Bereiche können deshalb als Alleinstellungsmerkmal angesehen werden.\\
Jedoch wurde dadurch die Zusammenarbeit manchmal in den Hintergrund gerückt und es wurden Entscheidungen, welche vom Team demokratisch gefällt werden sollten, durch wenige Personen entschieden und eine Umsetzung begonnen.
Als Beispiel hierfür wäre die Implementierung einer Datenbank sowie die Implementierung der Grundzüge des Back-Ends in der Weihnachtspause zu nennen.\\
Hervorzuheben ist, dass das gemeinsame Erarbeiten von Prozessen oftmals das Verstehen des Sachverhalts erleichtert hat.

Durch fehlende Ambitionen zu Demotivierungen, welche wiederum einseitige Arbeitsverhältnisse zur Folge hatten.
Auch wenn einzelne Gruppenmitglieder nicht voller Ehrgeiz und Eifer strotzten konnte das Arbeitsdefizit mit Hilfe der Gruppendynamik ausgeglichen werden.
Ebenfalls konnten Wissensdefizite und Schwächen Einzelner durch Stärken des Teams überwunden und ausgemerzt werden.

Trotz vorhandener Misskommunikation war die Arbeitsatmosphäre im Team entspannt, da auch in der Gruppe beschlossene Aufgaben in einem eigenen Arbeitstempo erledigt werden konnten.

Besonders positiv hervorzuheben ist der erste Sprint, da hier erfolgreich als Team auf ein Ziel hin gearbeitet wurde und jeder seine individuellen Stärken einsetzen konnte.
Dies hatte einen schnellen Start zur Folge, welcher es frühzeitig ermöglichte mit der Implementierung zu beginnen.
Hierbei war eine klare Richtung des Projekts zu erkennen, was allerdings durch ein fehlendes Projekt-Management im Team dazu geführt hat, dass die Energie und die Ambitionen nicht aufrecht gehalten und in ein noch besseres und erfolgreicheres Projekt umgewandelt werden konnten.

Endgültig ist zu sagen, dass die Autoren als Team aus den Erfahrungen gelernt haben und nun in weiteren Projekten ein sinnvolles Projektmanagement umsetzen werden.
Dadurch sollte nicht nur die Motivation aller Team-Mitglieder gesteigert, sondern auch die Produktivität erhöht werden.
Diese Probleme und Lösungen sind das Ergebnis eines konstruktiven Feedbackgesprächs, welches am Ende des Projekts geführt wurde und bei dem jedes Mitglied seine Erfahrungen und Erkenntnisse teilen konnte.
