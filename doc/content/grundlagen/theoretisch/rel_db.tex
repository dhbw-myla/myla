% !TEX root =  ../theoretisch.tex
\subsection{Relationale Datenbanksysteme}
\label{sec:grundlagen:reldb}
\subsubsection{Allgemein}

Relationale Datenbanksysteme stellen das Grundgerüst vieler Anwendungen dar, welche Datenveränderungen häufig und vor allem in kurzer Zeit verlässlich speichern müssen.
Dazu gehören unter anderem \ac{ERP}-Systeme oder auch von Content-Management-Systemen von Unternehmen.
Gemäß \citeauthor{Book_DB_2} ist ein Datenbanksystem eine selbstständige und auf Dauer ausgelegte Datenorganisation, welche einen Datenbestand sicher und flexibel verwalten kann.
Sie soll dem Benutzer einen einfachen Zugriff auf die Daten bieten und muss verhindern, dass ein Benutzer Daten manipulieren und einsehen kann, für die er keine Zugriffsrechte hat.
Der Benutzer muss die Möglichkeit haben, die Daten anzupassen, ohne dass Anwendungsprogramme angepasst werden müssen.\autocite[Vgl.][S.5 f.]{Book_DB_2}

Grundlegend dienen die Datenbanksysteme somit zur Datenverwaltung mithilfe von Tabellen, auch als \enquote{Relationen} bezeichnet.
Neue Relationen können dabei jederzeit hinzugefügt werden. 
Dadurch ist das Datenbanksystem ständig erweiterbar und nicht an eine im Voraus festgelegte Struktur gebunden.
Jede Spalte einer Tabelle besitzt einen Bezeichner, etwa \enquote{Nachname}, wie es beispielhaft in der Relation \emph{Student} (siehe Tab. \vref{tab:RelStudent}) dargestellt wird.
Jede Spalte stellt dabei ein Attribut der Relation dar.
Die Zusammenfassung aller Werte, welche in einer Spalte stehen können, werden als Domäne bezeichnet.
Eine solche Domäne wäre etwa \emph{WWI17SEB, WWI18SEA}, wenn zunächst die Annahme getroffen wird, dass in einer Spalte \emph{Kurse} lediglich die Werte \emph{WWI17SEB} und \emph{WWI18SEA} eingetragen sind.
Ist ein Wert noch nicht bekannt, wird der Wert \texttt{NULL} eingetragen, was in der Regel durch ein \enquote{-} visualisiert wird.
Zeilen einer Relation werden jeweils als Tupel bezeichnet und bestehen aus meheren Werten verschiedener Spalten.\autocite[Vgl.][S.9 ff.]{Book_DB_2}

\begin{table}
    \centering
    \begin{tabular}[h]{l | l | l | l}
        ID & Nachname & Vorname & Kurs \\ \hline
        1 & Fischer & Rene & WWI17SEB \\
        2 & Görnert & Sascha & WWI17SEB \\
        3 & Meier & Tina & WWI18SEA \\
        \dots & \dots & \dots & \dots \\
        \end{tabular}
        \caption{Beispiel Relation: Student}
        \label{tab:RelStudent}
\end{table}


\subsubsection{Bestandteile}
\label{ssec:Datenbanken_Bestandteile}
Datenbanksysteme sind immer nach einem Schema ausgerichtet, dass ihren Aufbau definiert.
So beinhaltet jedes Datenbanksystem ein \ac{DBMS} und eine Datenbank.
Das \ac{DBMS} ist für die logische Datenverwaltung der Datenbank verantwortlich, die aus einer oder mehreren Datenbanken bestehen kann. 
Programme können nur über bestimmte Schnittstellen des \ac{DBMS} auf die Datenbanken zugreifen, \zb mittels der Datenbanksprache \ac{SQL}.\autocite{Book_DB_1}\autocite{Book_DB_2}
Der Vorteil, über ein normiertes Interface eines \ac{DBMS} auf die Datenbank zuzugreifen, liegt darin, dass der Programmierer keine Kenntnis über die innerliche Struktur der Daten haben muss.
Dies übernimmt das \ac{DBMS}.\autocite[Vgl.][S.4]{Schicker2017DatenbankenSQL}


\subsubsection{Beziehungen zwischen Relationen}
Beziehungen zwischen unterschiedlichen Relationen werden mithilfe der \emph{Primärschlüssel \engl{Primary Key}} sowie \emph{Fremdschlüssel \engl{Foreign Key}} realisiert.
Der \emph{Primary Key} ist ein eindeutiger Identifikationswert eines Tupel und wird durch Unterstreichen innerhalb der visuell dargestellten Relation gekennzeichnet.
Dieser kann entweder aus mehreren Attributen zusammengesetzt sein oder aus einem eindeutigen Identifikationsmerkmal, beispielsweise einer fortlaufend erhöhten Zahl, auch als \texttt{ID} bezeichnet, bestehen.
Als Primärschlüssel wird dabei ein Kandidatenschlüssel, welcher laut Definition das Tupel eindeutig identifizieren muss, ausgewählt.
Ein solcher Kandidatenschlüssel kann aus mehreren Attributen bestehen, welche diese Bedingung erfüllen oder auch aus dem gesamten Tupel.
Da die Suche über den Primary Key jedoch mit der Größe dessen komplexer wird, wird in der Praxis in den meisten Fällen auf eine künstliche, fortlaufende \texttt{ID} zurückgegriffen.\autocite{Book_DB_2}


\subsubsection{\acl{ER-Modell}}
Der Aufbau einer Datenbank kann mithilfe eines \acs{ER-Modell}s graphisch abgebildet werden. 
Dazu werden Entitäten \engl{Entities} welche die Relationen repräsentieren, und Beziehungen \engl{Relationships} zwischen den Entitäten in einer standardisierten Form dargestellt.
Eine Entität bildet dabei Dinge aus der realen Welt ab (Umfrage, Student, Gruppenarbeit) und besitzt Eigenschaften bzw. Attribute, die sie beschreiben (Name, Alter, Geschlecht, Umfrageart, Anzahl der Mitgleider).
Ein Sonderfall einer Entität ist dabei die sogenannte \enquote{schwache} Entität.
Ein Objekt wird schwache Entität genannt, wenn sie abhängig von einer anderen ist und gleichzeitig ohne diese nicht existieren kann.
Beispielhaft kann ein Raum nicht ohne ein Gebäude existieren, was ihn zu einer schwachen Entität macht. 
Die Beziehungen zwischen den Entitäten werden über drei Beziehungswerte, \texttt{1}, \texttt{c = \{0, 1\}} sowie \texttt{n} (bzw. \texttt{m}), abgebildet.
Daraus resultieren diese Beziehungen:

\begin{itemize}
    \item 1:1-Beziehung \newline
    Dies beschreibt eine Beziehung, bei der jeweils ein Objekt genau einem anderen zugeordnet werden kann und umgekehrt.
    Zum Beispiel besitzt ein Student einen Studentenausweis und ein Studentenausweis ist genau einem einzelnen Studenten zugeordnert.
    \item 1:n-Beziehung \newline
    Diese Beziehung beschreibt ein Objekt, welches einem oder mehreren Objekten zugeordnet werden kann.
    So kann ein Student nur Teil eines Kurses sein, aber der Kurs kann aus einem oder mehreren Studenten bestehen.
    \item 1:c-Beziehung \newline
    Dies bedeutet, dass ein Objekt entweder keinem oder genau einem Objekt zugeordnet werden kann.
    Meistens werden dadurch zusätzliche Eigenschaften ausgedrückt, welche in vielen Fällen nicht vorhanden sind.
	Beispielsweise kann dabei ein Student als wissenschaftliche Hilfskraft in der Universität arbeiten und besitzt deshalb weitere Eigenschaften, die die meisten Studenten nicht besitzen.
	Hauptsächlich dient diese Beziehung der Datensparsamkeit, da so erhebliche nicht genutzte reservierte Speicherbereiche vermieden werden können.
    \item n:m-Beziehung \newline
    Relationen mit einer solchen Beziehung besitzen Objekte, die jeweils mehrere Objekte zugewiesen bekommen, aber auch meheren anderen Objekten zugewiesen werden können.
    Werden die Entitäten Student und Gruppenarbeit betrachtet, so kann ein Student Autor mehrerer Gruppenarbeiten sein, aber auch eine Gruppenarbeit besitzt mehrere Studenten als Autoren.
\end{itemize}
% Normalisierungsformen ???

Durch die sogenannten \enquote{Normalisierungen} werden große Relationen, die redundante Informationen speichern, auf ihre verschiedenen Teilaspekte heruntergebrochen und somit der verwendete Speicherplatz optimiert.
Zudem wird durch die Normalisierung die Datenkonsistenz gewährleistet und Datenredundanz verhindert, da jeder Datensatz nur an einer Stelle vorliegen kann.
Daten, die einer festen Struktur folgen, können dadurch einfach in die Datenbank persitiert sowie effizient abgefragt werden.

Die Nachteile relationaler Datenbanksysteme ergeben sich aus ihren Vorteilen.
Die aufgebaute Struktur, die Aufteilung eines Objektes auf viele verschiedene Spalten und Relationen, bildet ein starres System, welches Anpassungen während des Betriebs schwierig machen kann und ein Verständnis der gespeicherten Daten voraussetzt.
Um weitere Aspekte der Daten ebenfalls verwalten zu können, kann es notwendig sein, die komplette Struktur aufwändig anzupassen.
