% !TEX root =  ../theoretisch.tex
\subsection{Relationale Datenbanken}

Der grundlegende Aufbau relationaler Datenbanken sind Tabellen, Relationen, welche mit Daten gefüllt sind. 
Neue Relationen können jederzeit hinzugefügt werden, die Datenbank kann ständig erweitert werden.
Jede Spalte erhält einen Namen, beispielsweise \enquote{Nachname}, wie in der Relation \emph{Student} (siehe \vref{tab:RelStudent}) dargestellt und wird als Attribut bezeichnet. 
Die Zusammenfassung aller Werte, welche in einer Spalte stehen können, werden als Domäne bezeichnet.
Unter der Annahme, dass es nur die Kurse \emph{WWI17-SEB} und \emph{WWI18-SEA} gibt, wäre die
dazugehörige Domäne entsprechend {WWI17-SEB, WWI18-SEA}. 
Ist ein Wert noch nicht bekannt, wird \enquote{Null} anstelle des Wertes eingetragen, dargestellt durch ein \enquote{-} in vielen
Relationen. 
Eine Zeile der Relation wird als Tupel bezeichnet.\autocite[vgl. S.9 ff.][]{Book_DB_2}


\subsubsection{Beziehungen zwischen Relationen}
Beziehungen zwischen unterschiedlichen Relationen werden mit dem \emph{Primary Key} bzw. \emph{Foreign Key} realisiert. 
Der \emph{Primary Key} ist ein eindeutiger Identifikationswert eines Tupel und wird durch unterstreichen innerhalb der visuell dargestellten Relation gekennzeichnet. 
Dieser kann entweder aus mehreren Attributen zusammengesetzt sein, oder aus einem eindeutigen Identifikationsmittel, beispielsweise einer fortlaufende ID, bestehen. 
In der Praxis wird hierbei in den meisten Fällen auf eine ID zurückgegriffen, da eine Eindeutigkeit mittels des Zusammensetzens mehrere Attribute nicht immer gewährleistet und aufwändiger ist.\autocite[vgl. S.38][]{Book_DB_2}


\begin{table}
    \begin{tabular}[h]{l | l | l | l}
        ID & Nachname & Vorname & Kurs \\ \hline
        1 & Fischer & Rene & WWI17-SEB \\
        2 & Görnert & Sascha & WWI17-SEB \\
        3 & Meier & Tina & WWI18-SEA \\
        \dots & \dots & \dots & \dots \\
        \end{tabular}
        \caption{Beispiel Relation: Student}
        \label{tab:RelStudent}
\end{table}

\subsubsection{ER-Modell}
Eine Möglichkeit um mehrere Relationen zu beschreiben, bietet das ER-Modell (Entity Relationship). 
Es bedient sich dabei mehreren Kenngrößen, die bei der Beschreibung dargestellt werden. 
Mit Hilfe von Entitäten (Entity) und Beziehungen (Relationship) können so ganze relationale Datenbankmodelle beschrieben werden.
Eine Entität bildet dabei etwas aus der realen Welt ab und (Kunde, Reise) besitzt Eigenschaften bzw. Attribute, die sie beschreiben (Name, Alter, Geschlecht).
Ein Sonderfall ist die schwache Entität. Ein Objekt wird schwache Entität genannt, wenn sie abhängig von einer anderen ist und ohne sie nicht existieren könnte. 
Ein Raum könnte ohne ein Gebäude nicht existieren und ist somit eine schwache Entität. 
Entitäten können durch Beziehungen zwischen einander verknüpft werden. 
Hierbei gibt es drei Beziehungswerte: 1; c = {0, 1} ; m (bzw. n). 
Daraus resultieren diese Beziehungen:

\begin{itemize}
    \item 1:1 Beziehung \newline
    Beschreibt eine Beziehung, bei der jeweils ein Objekt einem anderen zugeordnet werden kann und umgekehrt. 
    Zum Beispiel kann ein Studen eine Matrikelnummer haben und diese Matrikelnummer kann auch nur einem Studenten zugeordnet werden.
    \item 1:m Beziehung \newline
    Diese Beziehung beschreibt ein Objekt, dass keins oder mehrere andere Einträge bei einem anderen haben kann. 
    Ein Student kann nur einen Kurs besitzen aber in einem Kurs können mehrere Studenten sein. 
    \item 1:c Beziehung \newline
    Dies bedeutet, dass ein Objekt einen oder keinen Eintrag besitzt. 
    Meistens werden dadurch Zusatzeigenschaften ausgedrückt. 
    Ein Student kann eine Wissenschaftliche Hilfskraft sein, aber auch nicht. 
    Jedoch ist ein Student zwingend nötig, um einen Wissenschaftliche Hilfskraft zu beschreiben (schwache Entität).
    \item n:m Beziehung \newline
    Diese Beziehung steht für viele Verknüpfungen auf beiden Seiten. 
    Betrachtet man die Entitäten Student und Gruppenarbeit, so kann ein Student mehrere Gruppenarbeiten machen und eine Gruppenarbeit kann mehrere Studenten beinhalten.
\end{itemize}
% Normalisierungsformen ???

Es gibt eine große Anzahl von Datenbanken. Hier ist ein Auszug von möglichen Datenbanken:\autocite{online_Datenbank}
\begin{itemize}
	\item Hierarchisches Datenbankmodell
	\begin{itemize}
		\item älteres DB-Modell
		\item Lesezugriff schnell, da Baumstruktur
		\item Nachteil: Speicher der Daten in Baumstruktur
		\item Verknüpfungen über Eltern-Kind-Beziehungen dargestellt
	\end{itemize}
	\item Netzwerkdatenbankmodell
	\begin{itemize}
		\item unterschiedliche Suchwege, um Datensatz zu ermitteln
		\item Ähnlichkeit mit Hierarchischen Datenbankmodell
		\item Übersichtlichkeit verringert sich, wenn Modell ständig weiter wächst
	\end{itemize}
	\item Relationales Datenbankmodell
	\begin{itemize}
		\item drei wichtige Bausteine: Tabellen, Attribute, Beziehungen
		\item Ansammlung von Tabellen
		\item Vertiefungen werden über Primärschlüssel dargestellt
		\item Normalisierung muss eingehalten werden (3 Stufen $\rightarrow$ nicht immer sinnvoll)
	\end{itemize}
	\item Objektorientiertes Datenbankmodell
	\begin{itemize}
		\item ähnlich wie in JAVA: Daten und Funktion in einem Objekt speichern
		\item Abfrage-Sprache Object Query Language (OQL) 
		\item Praxis kaum Anwendung, da Trennung von Daten und Funktion stattfinden muss
		\item  mit zunehmender Anzahl an Daten nimmt Performance ab im Vgl. zu rel. DB
	\end{itemize}
\end{itemize}

Eine relationale Datenbank besteht aus einer Vielzahl unterschiedlicher Relationen, welche mithilfe von Fremdschlüsseln in Beziehungen zueinander stehen können.
Durch Normalisierungen werden große Objekte auf verschiedene Teilaspekte herunter gebrochen. 
Außerdem wird durch Normalisierung die Konsistenz der Daten immerzu gewährt, da jeder Datensatz nur an einer Stelle vorliegen kann. Daten, die einer festen Struktur folgen, können auf einfachem Weg in die Datenbank integriert, sowie schnell
und effizient abgefragt werden.

Die Nachteile relationaler Datenbanken, ergeben sich aus deren Vorteilen. 
Die aufgebaute Struktur, die Aufteilung eines Objektes auf viele verschiedene Spalten und Relationen, bildet ein starres System, welches nicht ohne weiteres während des Betriebs angepasst werden kann. 
Um neue Daten verwalten zu können, kann es notwendig sein, die komplette Struktur aufwändig anzupassen.