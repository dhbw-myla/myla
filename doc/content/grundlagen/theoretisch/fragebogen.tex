% !TEX root =  master.tex
% PLAGIAT
\subsection{Fragebogen}
\label{acqusition}
\enquote{Dem guten Frager ist schon halb geantwortet}, dieser Meinung war schon Friedrich Nietzsche um 1885. 
In diesem Kapitel wird näher darauf eingegangen, was eine gut gestellte Frage übermittelt und wie diese Fragen innerhalb eines Fragebogens aufzubauen sind, um die bestmögliche Antwort zu erhalten\autocite{Nietzsche}.
Dieses Kapitel behandelt ebenfalls den Aufbau eines Fragebogens. 
Es wird näher darauf eingegangen, wie ein Fragebogen aufgebaut sein muss, um den Beantwortet durch den kompletten Fragebogen hinweg animiert zu halten. 
Weiterhin werden die verschiedenen Fragentypen erläutert und deren Einsatzmöglichkeiten verdeutlicht. 

Die Aufgabe des Designs eines Fragebogens fängt bereits bei dem Anschreiben an.
Da ein Fragebogen sich nicht nur auf die Befragung an sich bezieht, sondern auch auf den Prozess, der diese Umfrage begleitet.
Sollen darin Informationen wie Name und die Adresse des Absenders, Thema der Befragung, Zusammenhang zum Thema, Anonymitätsgarantie für den Beantwortenden, Rücksendetermin, Begründung für die Auswahl der Empfänger und die benötigte Zeit für das Ausfüllen enthalten sein\autocite[S.29]{2003Fragebogen}. %2003 Fragebogen S.29
Diese Faktoren beeinflussen die Bereitschaft, den Fragebogen auszufüllen. 
Schon im Anschreiben spielt der zeitliche Faktor eine sehr wichtige Rolle. 
25\% der Personen den Fragebogen gar nicht öffnen die Umfrage gar nicht, wenn diese zehn Minuten dauert.
Dieser Effekt nimmt bei längeren Umfragen immer weiter zu\autocite[S.353]{NFP}.
Das Höchstmaß zur Bereitschaft für die Beantwortung einer Umfrage beträgt meistens 15 Minuten \autocite[S.37]{2009Fragebogen}. %2009_BookEvalOnline 37
Eine kürzere, prägnante Umfrage ist aus diesem Grund eine Möglichkeit, um die Antwortrate zu erhöhen.

Des Weiteren spielt die der Faktor Zeit eine große Rolle, eine Zeitangabe akkurat sein.
Es führt zu Frustration, wenn die Zeitangabe geringer als die tatsächliche Befragung und die damit verbunden einen Abbruch des Fragebogens einher ziehen. 
Eine Fortschrittsanzeige verdeutlicht dem Beantwortenden, inwieweit er in seinem Beantwortungsprozess fortgeschritten ist. 
Dem mittlerweile \enquote{geschulten Blick für Dilettantismus} in der Onlinewelt, muss gerecht gekommen werden.
Schreibfehler und Ungereimtheiten müssen vor dem Versenden bereits ausgemerzt sein\autocite[S.37]{2009Fragebogen}. %2009_BookEvalOnline 37

Die Rückläuferquote kann weiter erhöht werden, wenn es eine Ankündigung des Fragebogens vor dem Versenden gibt. 
Je nach Wichtigkeit der Umfrage kann das dazu führen, dass die Befragten sich mental auf Umfrage vorbereiten\autocite[S.38]{Umfragenforschung}.
%https://link.springer.com/content/pdf/10.1007%2F978-3-531-91852-5.pdf S38
Dem Mode-Effekt kann entgegengewirkt werden, wenn es sich um eine internetgestützte Umfrage handelt.
In Interviews führt dieser Effekt zu Verfälschung der Ergebnisse, da in einem direkten Dialog mitgeteilt werden muss, was die teilweise unerwünschte Wahrheit ist\autocite[S.163]{Umfragenforschung}.
%S163
 
Um Informationen zu erhalten, gibt es verschiedenste Fragearten, die ausgewählt werden können.
Egal, ob es sich um eine offene oder eine geschlossene Frage handelt\autocite[S.31]{2009Fragebogen}. %2009Fragebogen S.31
Besitzen die Fragen als Ziel, dass wertvolle Informationen gesammelt werden.
Die Standards für Evaluationen beschreiben den Sachverhalt weiter und spiegeln wieder, dass eine durchdachte Fragestellung ebenso wichtig ist.

\begin{quote} \enquote{Erhebungsverfahren und Datenquellen sollen so gewählt werden, dass die Zuverlässigkeit der gewonnenen Daten und ihre Gültigkeit bezogen auf die Beantwortung der Evaluationsfragestellungen nach fachlichen Maßstäben sichergestellt sind. Die fachlichen Maßstäbe sollen sich an den Gütekriterien der empirischen Forschung orientieren.} Genauigkeitsstandard 5: Valide und reliable Informationen\autocite{DegEval} %https://www.degeval.org/degeval-standards/kurzfassung/
\end{quote}

Bei Offene Fragen gibt es keine Antwortmöglichkeiten.
Die Beantworter sollen mit ihren eigenen Worten Stellung beziehen.
Durch das Offenlegen von Motiven, Begründungen, Verbesserungsvorschlägen bezüglich des Evaluationsgegenstandes können die Forscher Erkenntnisse gewinnen.
Der Befragte und dessen Sichtweise bezüglich des speziellen Themas wird in den Vordergrund der Betrachtung gestellt\autocite[S.31]{2009Fragebogen}. %2009Fragebogen S.31
Die Person fühlt sich wertgeschätzt, ein motivierender Effekt stellt sich ein, da die Meinung der dieser Person sehr wertgeschätzt wird. 
Ob eine Notwendigkeit besteht, eine offene Frage zu formulieren, ist jedoch erst zu prüfen.
Wenn die detaillierte Meinung eines Umfrageteilnehmers zu einem Thema abgefragt werden soll, dann ist eine offene Frage wertvoller. 
 
%Plagiat
Durch das Verwenden von geschlossenen oder auch standardisierten Fragen werden die Beantwortenden eingeschränkt.
Das Spektrum der möglichen Antworten muss dabei vorher vollständig bekannt und auch abgegrenzt sein\autocite[S.66]{2014Fragebogen}. % (s66 2014 book fragenbogen)
Durch Einfachnennung oder Mehrfachnennung gibt es die Möglichkeit innerhalb des Spektrums der geschlossenen Fragen eine Variation der Fragestellungen darzustellen\autocite[S.31]{2009Fragebogen}. %2009Fragebogen S.31
Da es bei Fragen mit Skalen auch nur die Möglichkeit gibt, in einem vordefinierten Bereich seine Antwort abzugeben, sind es prinzipiell auch geschlossene Fragen.
Jedoch ist die Zuordnung in der Auswertung bei vorgefertigten Antwortmöglichkeiten leichter.

Da bei offenen Fragen die Zuordnung in Themengebiete geschieht, an welche vorher nicht gedacht werden konnte, sind selbst in Fällen, wo scheinbar alle Möglichkeiten durchdacht sind, immer noch abwegige Antworten möglich.
Da es eine Freitextfrage ist, wird der Antwortende nicht an der Abgabe dieser gehindert. 
Als Beispiel dient eine Applikation zum Darstellen des Essensangebotes, in dieser besteht die Möglichkeit, per Freitextfrage Feedback zur dieser abzugeben.
Das Feedback soll sich laut der Fragestellung direkt auf die Applikation beziehen.
Trotzdem wird es Beantwortende geben, welche die Auswahl der Gerichte in einer Kantine bewerten.


Im Anhang \vref{list:gebote} sind die zehn Gebote der Fragenformulierung definiert\autocite{RolfPorst}. %RolfProstFragen S. 689
Diese sind von R. Porst aufgestellt und sollen als Leitfaden gelten, um verständliche Fragen zu formulieren.
Da es aber kein allgemeines Kochrezept ist, um Fragen zu designen, sind diese Regeln in den meisten Fällen schnell wieder vergessen und werden deshalb nicht richtig angewandt.%RolfProstFragen S.698

Fragen können in geschlossene und offene Fragen eingeteilt werden.
Da der Aufbau von geschlossenen Fragen jedoch bisher nur mit der Anwesenheit von vorgegebenen Antworten definiert wurde, ist es nötig, einen Blick auf die möglichen Skalen zu werfen.

Mögliche Skalen, die verwendet werden, können sich in nominale, ordinale,\\
 Intervall-, Ratio-, verbalisierte und endpunktbenannte Skalen einteilen\autocite[S.71]{2014Fragebogen}. %2014 Fragebogen S71.
Skalen mit einer geraden Anzahl von Antworten führen dazu, dass es keine mittlere Meinung gibt. 
Der Antwortende wird durch das Design der Frage gezwungen, sich für eine Seite, ob positiv oder negativ, zu entscheiden.
Unbeachtet davon, ob der Skalenmittelpunkt wirklich der Skalenmittelpunkt oder nur der Skalenpunkt in der Mitte ist, neigt der Beantwortende dazu, diesen als solchen zu interpretieren\autocite[S.83]{2014Fragebogen}.%2014 Fragebogen S.83

Unabhängig davon, ob eine Skala gerade oder ungerade ist, kann durch die Breite dieser ein großer Effekt erzeugt werden. 
Eine zu schmale Skala führt dazu, dass nicht genügend differenziert werden kann.
Im Gegenzug dazu kann eine zu breite Skala dazu führen, dass die Grenzen verschwimmen.
Es wird für Fragesteller und Beantworter schwierig, die einzelnen Schritte benennen zu können. 
Als extremes Negativbeispiel ist eine Skala mit 20 möglichen Skalenpunkten zu sehen.
Es ist fraglich, ob eine Unterscheidung zwischen den Skalenpunkten 16 und 17 noch möglich ist.
Darum gibt es die Faustregel bei endpunktbenannten Fragen nicht weniger als fünf Punkte und nicht mehr als neun Skalenpunkte anzubieten\autocite[S.87]{2014Fragebogen}. %(2014 Book Fragebogen S 87)
Intuitiv gesehen sollen Fragen ebenfalls durch die Gliederung von niedrig/negativ nach positiv/hoch leichter verständlich sein. Eine einheitliche Einteilung ist vom Vorteil, um den Benutzer nicht mit Umdenken in ein neues System zu belasten\autocite[S.89/90]{2014Fragebogen}. %(2014 Book Fragebogen S89/90)

Auf dem ersten Blick ist das Wohlbefinden eines Benutzers beim Beantworten des Fragebogen nicht inhärent wichtig. 
Jedoch ist User Experience ein Konzept, um den Beantwortenden nicht zu überfordern und keine Zeit mit der Verwirrung zu verschwenden.
Aus diesem Hintergrund haben sich in der Umfragepraxis numerische (endpunktbenannte) Likert-Skalen bewährt\autocite{Likert}. %https://de.statista.com/statistik/lexikon/definition/82/likert_skala/ 
Likert-Skalen werden meistens mit einer ungeraden Anzahl an Merkmalausprägungen verwendet. Der Skalenmittelpunkt ist dabei direkt mit der neutralen Antwort gleichzusetzen\autocite{ISO}. \label{Likert} 
%https://www.procontext.de/aktuelles/2010/03/iso-9241210-prozess-zur-entwicklung-gebrauchstauglicher-interaktiver-systeme-veroeffentlicht.html

\begin{quote}
	\enquote{User Experience umfasst demzufolge alle Effekte, die ein Produkt bereits vor der Nutzung (antizipierte Nutzung) als auch nach der Nutzung (Identifikation mit dem Produkt oder Distanzierung) auf den Nutzer hat. Usability wiederum fokussiert auf die eigentliche Nutzungssituation (Effektivität und Effizienz).}\autocite{User}%https://www.researchgate.net/publication/256495056_User_Experience_mit_Fragebogen_messen_-_Durchfuhrung_und_Auswertung_am_Beispiel_des_UEQ
\end{quote}

In einem Interview kann die offene Frage so formuliert werden, dass diese an den Gesprächsfluss gut angepasst ist.
Diese Möglichkeit besteht jedoch bei einem Fragebogen nicht, von daher müssen die Fragen vorher gut durchdacht werden und einen besonderen Fokus erhalten.
Des Weiteren sind offene Fragen bei der Beantwortung zeitaufwendiger als geschlossene Fragen. 
Die Antwort muss von Grund aus überdacht und nicht nur abgewägt werden. 
Daher ist eine präzise, verständliche und mit zusätzlichen Informationen ausgeschmückte Formulierung wichtig. 
Da zu viele offene Fragen den Probanden abschrecken, ist eine Mischung aus geschlossenen und offenen Fragen wichtig.
Bevor eine Freitextfrage gestellt werden kann, sind Erinnerungsfragen hilfreich. Diese führen zum Thema hin und helfen dem Antwortenden bessere Antworten zu finden\autocite[S.35]{2009Fragebogen}.%2009_BookEvalOnline 35

%2009Fragebogen S.36 in Struktur einarbeiten

Der Fragebogen wurde nach der in diesem Kapitel erläuterten Theorie aufgestellt und im Anhang der Abbildungen \vref{fig:SurveyStart} bis \vref{fig:Survey6} abgebildet.
Die Reihenfolge der Fragen sind von Abbildung \ref{fig:SurveyStart} bis \ref{fig:Survey6} in Reihenfolge der Darstellung im Anhang angeordnet.
Es handelt sich um eine Umfrage, die mit Google Forms durchgeführt wurde.
Zusätzlich ist zu erwähnen, dass die Seitenanzahl, wie in \vref{fig:Survey1} dargestellt, sich nur auf die einzelnen Abschnitte bezieht.
Um eine bessere Auswertung zu ermöglichen gibt es mehr Abschnitte, die als Fragen gestellt werden. Dies kann den Benutzer zwar verwirren, aber da nun das Feedback aus \vref{fig:Survey4} sich direkt auf \vref{fig:Survey3} bezieht, ist die Auswertung erleichtert.    
Eine direkte Zuordnung von der Freitextfrage zu der geschlossenen Frage wird direkt durchgeführt. 
Dies ist zwar später möglich, würde aber unnötigen Aufwand bedeuten in der Auswertung.
