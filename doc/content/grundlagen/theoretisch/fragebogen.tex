% !TEX root =  master.tex
\subsection{Fragebogen}
\label{acqusition}
\enquote{Dem guten Frager ist schon halb geantwortet}, dieser Meinung war schon Friedrich Nietzsche um 1885.\autocite{Nietzsche}
In diesem Kapitel wird näher darauf eingegangen, was eine gut gestellte Frage übermittelt und wie diese Fragen innerhalb eines Fragebogens aufzubauen sind, um die bestmögliche Antwort zu erhalten.
Dieses Kapitel behandelt ebenfalls den Aufbau eines Fragebogens.
Es wird näher darauf eingegangen, wie ein Fragebogen aufgebaut sein muss, um den Beantworter durch den kompletten Fragebogen hinweg animiert zu halten.
Weiterhin werden die verschiedenen Fragentypen erläutert und deren Einsatzmöglichkeiten verdeutlicht.

Die Aufgabe des Designs eines Fragebogens fängt bereits beim Anschreiben an.
Da sich ein Fragebogen nicht nur auf die Befragung als solche bezieht, sondern auch auf den Prozess, der diese Umfrage begleitet, sollen folgende Informationen enthalten sein:\autocite[Vgl.][S. 29]{2003Fragebogen} %2003 Fragebogen S.29
%
\begin{itemize}
	\item Name und die Adresse des Absenders
	\item Thema der Befragung
	\item Zusammenhang zum Thema
	\item Anonymitätsgarantie für den Beantwortenden
	\item Rücksendetermin
	\item Begründung für die Auswahl der Empfänger
	\item Benötigte Zeit zum Ausfüllen
\end{itemize}
%
Diese Faktoren beeinflussen die Bereitschaft, den Fragebogen auszufüllen.
Schon im Anschreiben spielt der zeitliche Faktor eine sehr wichtige Rolle.
\citeauthor{NFP} gibt an, dass 25\% der Personen den Fragebogen gar nicht öffnen oder gar die Umfrage nicht ausfüllen, wenn diese mit einem Aufwand von zehn Minuten verbunden ist.
Dieser Effekt nimmt bei längeren Umfragen immer weiter zu.\autocite[Vgl.][S. 353]{NFP}
Das Höchstmaß zur Bereitschaft für die Beantwortung einer Umfrage beträgt meistens 15 Minuten.\autocite[Vgl.][S. 37]{2009Fragebogen}%2009_BookEvalOnline 37
Eine kürzere, prägnante Umfrage ist aus diesem Grund eine Möglichkeit, um die Antwortrate zu erhöhen.

Des Weiteren spielt der Faktor Zeit eine große Rolle.
Eine Zeitangabe für die Bearbeitung des Fragebogens sollte dementsprechend akkurat sein.
Es führt zu Frustration, wenn die Zeitangabe geringer ist als die tatsächliche Befragung.
Damit einhergehend kann dies auch zu einem Abbruch des Fragebogens bzw. der Befragung führen.
Eine Fortschrittsanzeige verdeutlicht dem Antwortenden, inwieweit er in seinem Beantwortungsprozess fortgeschritten ist.
Dem mittlerweile \enquote{geschulten Blick für Dilettantismus} in der Onlinewelt, muss gerecht gekommen werden.
Schreibfehler und Ungereimtheiten müssen vor dem Versenden bereits ausgemerzt sein.\autocite[Siehe][S. 37]{2009Fragebogen} %2009_BookEvalOnline 37

Die Rücklaufquote kann weiter erhöht werden, wenn es eine Ankündigung des Fragebogens vor dem Versenden gibt.
Je nach Wichtigkeit der Umfrage kann das dazu führen, dass die Befragten sich mental auf die Umfrage vorbereiten.\autocite[Vgl.][S. 38]{Umfragenforschung}
%https://link.springer.com/content/pdf/10.1007%2F978-3-531-91852-5.pdf S38
Dem Mode-Effekt kann entgegengewirkt werden, wenn es sich um eine internetgestützte Umfrage handelt.
In Interviews führt dieser Effekt zu Verfälschung der Ergebnisse, da in einem direkten Dialog mitgeteilt werden muss, was die teilweise unerwünschte Wahrheit ist.\autocite[Vgl.][S. 163]{Umfragenforschung}
%S163

Um Informationen zu erhalten, gibt es verschiedenste Fragearten, die ausgewählt werden können.
Dabei ist es egal, ob es sich um eine offene oder eine geschlossene Frage handelt.\autocite[Vgl.][S. 31]{2009Fragebogen} %2009Fragebogen S.31
Besitzen die Fragen als Ziel, dass wertvolle Informationen gesammelt werden.

\begin{quote} \enquote{Erhebungsverfahren und Datenquellen sollen so gewählt werden, dass die Zuverlässigkeit der gewonnenen Daten und ihre Gültigkeit bezogen auf die Beantwortung der Evaluationsfragestellungen nach fachlichen Maßstäben sichergestellt sind. Die fachlichen Maßstäbe sollen sich an den Gütekriterien der empirischen Forschung orientieren.} Genauigkeitsstandard 5: Valide und reliable Informationen\autocite[Siehe][]{DegEval} %https://www.degeval.org/degeval-standards/kurzfassung/
\end{quote}

Die Standards für Evaluationen beschreiben den Sachverhalt weiter und spiegeln wieder, dass eine durchdachte Fragestellung ebenso wichtig ist.

Bei offenen Fragen gibt es keine Antwortmöglichkeiten.
Die Beantworter sollen hier mit ihren eigenen Worten Stellung beziehen.
Durch das Offenlegen von Motiven, Begründungen oder Verbesserungsvorschlägen bezüglich des Evaluationsgegenstandes können die Forscher Erkenntnisse gewinnen.
Der Befragte und dessen Sichtweise bezüglich des speziellen Themas wird in den Vordergrund der Betrachtung gestellt.\autocite[Vgl.][S. 31]{2009Fragebogen} %2009Fragebogen S.31
Die Person fühlt sich und ihre Meinung wertgeschätzt, wodurch sich ein motivierender Effekt einstellt.
Ob eine Notwendigkeit besteht, eine offene Frage zu formulieren, ist jedoch erst zu prüfen.
Wenn die detaillierte Meinung eines Umfrageteilnehmers zu einem Thema abgefragt werden soll, dann ist eine offene Frage grundsätzlich geeignet.

Bei dem Verwenden von geschlossenen oder auch standardisierten Fragen werden die Antwortmöglichkeiten eingeschränkt.
Ebenfalls müssen alle möglichen Antworten vorher vollständig bekannt und abgegrenzt sein.\autocite[Vgl.][S. 66]{2014Fragebogen} % (s66 2014 book fragenbogen)
Durch Einfach- oder Mehrfachnennung gibt es die Möglichkeit innerhalb des Spektrums der geschlossenen Fragen eine Variation der Fragestellungen darzustellen.\autocite[Vgl.][S. 31]{2009Fragebogen} %2009Fragebogen S.31
Zu den geschlossenen Fragen gehören auch skalierte Fragen, da diese die Möglichkeit bieten, in einem vordefinierten Bereich Antworten abzugeben.
Da das Antwortspektrum vorher bekannt ist, wird die Auswertung der gegebenen Antworten erleichtert.

Im Gegensatz dazu, sind die Zuordnungen von Themen bei einer offenen Fragestellung schwierig.
Durch die freie Möglichkeit der Antwortabgabe sind dem Antwortenden keine Grenzen gesetzt.
Da es sich um Freitextfragen handelt, wird der Befragte nicht an der Abgabe seiner Antwort mit einer ganz eigenen Interpretation der Fragestellung gehindert.
Als Beispiel dient eine Applikation zum Präsentieren des Essensangebotes.
In dieser besteht die Möglichkeit, per Freitextfrage Feedback zur Applikation abzugeben.
Laut der Fragestellung soll sich das Feedback direkt auf die Applikation beziehen.
Trotzdem wird es Antworten der Befragten geben, welche die Essensauswahl in einer Kantine bewerten.

Wie zuvor beschrieben können Fragen in offene und geschlossene Fragestellungen eingeteilt werden.
Da das Wesen von geschlossenen Fragen bisher mit der Anwesenheit von vordefinierten Antworten angegeben wurde, ist es nötig, einen Blick auf die möglichen Einteilungen der Skalen einer Fragestellung zu werfen.

Skalen werden in
\begin{itemize}
	\item nominale Skalen,
	\item ordinale Skalen,
	\item Intervallskalen
	\item Ratioskalen
	\item verbalisierte Skalen
	\item und endpunktbenannte Skalen
\end{itemize}

eingeteilt.\autocite[Vgl.][S. 71]{2014Fragebogen}
Skalen mit einer geraden Anzahl von Antworten besitzen keine neutrale Meinung.
Der Befragte wird durch das Design der Frage gezwungen, sich für eine Seite zu entscheiden (positiv oder negativ).
Unbeachtet davon, ob der Skalenmittelpunkt wirklich der Skalenmittelpunkt oder nur der Skalenpunkt in der Mitte ist, neigt der Befragte dazu, diesen als solchen zu interpretieren.\autocite[Vgl.][S. 83]{2014Fragebogen}%2014 Fragebogen S.83

Die Breite einer Skala kann einen großen Effekt auf den Befragten haben.
Dabei ist die Beschaffenheit der Skala davon unabhängig.
Eine Skala, welche sich nicht genügend differenzieren lässt, ist eine zu schmale Skala.
Im Gegensatz dazu verschwimmen die Grenzen bei einer zu breiten Skala.
Die einzelnen Punkte können nicht mehr benannt werden und für beide Parteien, Fragesteller und Befragter, wird die Differenzierung schwierig. \newline
Als Negativbeispiel ist eine Skala mit 17 möglichen Skalenpunkten zu sehen.
Es ist fraglich, ob eine Unterscheidung zwischen den Skalenpunkten 14 und 15 noch möglich ist.
Darum hat sich die Faustregel etabliert, bei endpunktbenannten Fragen nicht weniger als fünf und nicht mehr als neun Skalenpunkte anzugeben.\autocite[Vgl.][S. 87]{2014Fragebogen} %(2014 Book Fragebogen S 87)
Die Einteilung der Skala von negativ nach positiv ist intuitiv leichter zu verstehen.
Um den Befragten nicht zu verwirren, ist innerhalb eines Fragebogens eine einheitliche Einteilung der Skalen von Vorteil.
Somit wird der Benutzer nicht mit Umdenken in ein neues Skalensystem belastet.\autocite[Vgl.][S. 89 f]{2014Fragebogen} %(2014 Book Fragebogen S89/90)

Auf den ersten Blick ist das Wohlbefinden eines Benutzers beim Beantworten des Fragebogen nicht inhärent wichtig.
Es ist jedoch wichtig, den Befragten nicht zu überfordern und keine Zeit durch Verwirrung zu verschwenden.
Dieses Konzept wird auch als \acf*{UX} bezeichnet.
In der Umfragepraxis haben sich aus diesem Hintergrund numerische (endpunktbenannte) Likert-Skalen bewährt.\autocite[Vgl.][]{Likert}
Meistens besitzen Likert-Skalen eine ungerade Anzahl an Merkmalausprägungen.
Der Skalenmittelpunkt ist dabei direkt mit der neutralen Antwort gleichzusetzen.\autocite[Vgl.][]{ISO}

\begin{quote}
	\enquote{User Experience umfasst demzufolge alle Effekte, die ein Produkt bereits vor der Nutzung (antizipierte Nutzung) als auch nach der Nutzung (Identifikation mit dem Produkt oder Distanzierung) auf den Nutzer hat. Usability wiederum fokussiert auf die eigentliche Nutzungssituation (Effektivität und Effizienz).}\autocite[Siehe][]{User}
\end{quote}

In einem Interview können die Fragestellungen gut an den Gesprächsfluss angepasst werden.
Bei einem Fragebogen besteht diese Möglichkeit jedoch nicht.
Von daher müssen die Fragen vorher gut durchdacht werden und einen besonderen Fokus erhalten.
Des Weiteren ist die Beantwortung und Auswertung von offenen Fragen zeitaufwändiger als die Beantwortung von geschlossenen Fragen.
Die Antwort muss von Grund auf überdacht und nicht nur abgewägt werden.
Daher ist bei offenen Fragen eine verständliche, präzise und mit zusätzlichen Informationen ausgeschmückte Formulierung wichtig.
Eine Mischung aus geschlossenen und offenen Fragen ist vonnöten, damit der Proband nicht abgeschreckt wird.
Bevor eine Freitextfrage gestellt werden kann, können zur Hilfe Erinnerungsfragen genutzt werden.
Diese helfen dem Befragten, bessere Antworten zu finden und führen zum Thema hin.\autocite[Vgl.][S. 35]{2009Fragebogen}
