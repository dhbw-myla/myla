% !TEX root =  master.tex
% PLAGIAT
\subsection{Fragebogen}
\label{acqusition}
\enquote{Dem guten Frager ist schon halb geantwortet}, dieser Meinung war schon Friedrich Nietzsche um 1885. 
In diesem Kapitel wird näher darauf eingegangen, was eine gut gestellte Frage übermittelt und wie diese Fragen innerhalb eines Fragebogens aufzubauen sind, um die bestmögliche Antwort zu erhalten.
Das allgemeine Umfeld, in welchem ein Fragebogen gestellt wird, sowie der Aufbau dessen wird näher beleuchtet.\autocite{Nietzsche}.

\subsubsection{Vorbereitung} 
Um die gewünschten Antworten von den Beantwortenden zu erhalten, ist es wichtig, dass der Fragesteller sich an eine gewisse Struktur und Vorgehensweise hält.
Mit Hilfe eines gut strukturierten Fragebogens, wird begünstigt, dass Fragestellungen nach den Erwartungen beantwortet werden.
Die Vorbereitungen für die Evaluation eines Fragebogens sind ein wichtiger Aspekt, für die Durchführung einer Befragung mittels onlinegestützter Tools. 

Die Schritte zu einer effektiven Befragung lassen sich in fünf Teilbereiche einteilen. 
Der erste Punkt ist ein genaues Ziel des Fragebogens zu definieren. 
Weiterhin müssen die Erwartungen an die Ergebnisse bei allen beteiligten Parteien richtig gesetzt werden. 
Dies beinhaltet die Personen, welche einen Fragebogen entwerfen, Personen die den Fragebogen beantworten sowohl als die von deren Ergebnissen profitieren wollen.\autocite[Preparing for the Survey]{Perfect}

Das Design des Fragebogens ist die wichtigste Aufgabe einer Befragung. 
Alle Aspekte, wie zum Beispiel das Timing und die Länge des Fragebogens, sind hierbei sehr wichtig. 
Nachdem der Fragebogen geschrieben und der Aufbau fertig definiert ist, wird er an den Mitarbeiter geschickt.
Im letzten Schritt werden die gesammelten Ergebnisse analysiert, gesammelt und kommuniziert\autocite[Preparing for the Survey]{Perfect}.
 %https://proquest-tech-safaribooksonline-de.ezproxy-dhma-1.redi-bw.de/book/communications/writing/9780071664011/part-one-preparing-for-the-survey/ch01_html

Bei der Definition des Umfrageziels geht es darum, neben dem definierten Ziel, erstes Feedback zu der Befragung einzuholen. 
Dadurch können weitere Aspekte und wichtige Messwerte entdeckt werden, welche sonst nicht in Betracht gezogen werden.
So ist es wichtig, dass ein Fragebogen einen Mehrwert für das Unternehmen erzeugt. 
Infolge dessen lässt sich die Sinnhaftigkeit einer Befragung anhand der Sinnhaftigkeit der Ergebnisse abwägen. 
Die Sinnhaftigkeit der Ergebnisse ist nicht an den Wert zu messen, welcher durch das Ergebnis erzeugt wird. 
Der Vorteil, welcher durch die Analyse der Antworten gewonnen werden kann, ist als Ziel der Umfrage anzusehen\autocite[Defining Survey Goals]{Perfect}. 

Für die Durchführung einer Befragung sollte es mehrere Gründe geben. 
Um ein Beispiel für die Verwendung eines Fragebogens zu geben können wichtige Gründe, das Priorisieren von Unternehmenszielen anhand von objektiven Daten, sein. 
Dabei können durch komplette Antworten, die Ergebnisse der Befragung besser analysiert werden. 
Wenn ein Mitarbeiter erst Wochen später nach einem Event dieses bewerten soll, wird er nicht die gleiche Qualität der Antworten abgeben, wie bei einer Ad-hoc-Befragung.
Dadurch sind zeitnahe Umfragen bezüglich dieses Events ratsam. 
Diese Momentaufnahme gilt zur Betrachtung der Eindrücke, direkt nach dem Ereignis.

Die Mitarbeiter können durch rechtzeitiges Kommunizieren der Ziele des Fragebogens innerhalb des Unternehmens, auf die Umfrage vorbereitet werden. 
Somit kann die Antwortquote erhöht werden. 
%https://proquest-tech-safaribooksonline-de.ezproxy-dhma-1.redi-bw.de/book/communications/writing/9780071664011/part-one-preparing-for-the-survey/ch03_html

Im nächsten Schritt müssen die richtigen Erwartungen an den Fragebogen gesetzt werden. 
Eine mögliche Erwartung an eine Umfrage kann sein, dass anhand des Feedbacks Veränderungen im Unternehmen implementiert werden. 
Daher können falsch kommunizierte Ziele einen falschen Eindruck entstehen lassen.
Wenn der Mitarbeiter bei einer Umfrage die klare Erwartungen besitzt, dass scih etwas verändert, aber dann nichts passiert, kann dies zu Frustration führen.
Ebenfalls kann der Mitarbeiter eingeschüchtert sein eine Antwort abzugeben, da er Angst vor großen und schwerwiegenden Veränderungen besitzt\autocite[Chapter 3
Setting Expectations]{Perfect}.

%PLAGIAT
Im Gegensatz dazu entsteht durch die Herausgabe eines Fragebogens die Idee, dass sich etwas verändern wird und die Meinung des Mitarbeiters wirklich in diesem Falle zählt.
Wenn sich in Zuge dessen nichts ändert, kann dies die Mitarbeiterzufriedenheit senken und somit einen negativen Einfluss auf das Unternehmen besitzen.
Die Bereitschaft zur Veränderung ist auf beiden Seiten wichtig. 
Sowohl Mitarbeiter und Manager müssen sich dessen bewusst sein.
Der Vorgesetzte muss sich im Vorhinein klar sein, dass er bereit sein muss eine Veränderung herbeizuführen, wenn diese gewünscht ist.
So kann das Feedback vielleicht nicht direkt entscheiden, wie die Veränderung aussehen soll, jedoch kann es eine Richtung für spätere Entscheidungen geben. 
Das Zitat von Kevin Murphy untermauert diese Aussage\autocite[Chapter 3
Setting Expectations]{Perfect}.
\begin{quote} \enquote{We know we’ll be busy and will need to be great listeners. We make that commitment before we start.} Kevin Murphy, managing partner at the marketing communications firm Trone \end{quote}

Dieser Schritt wird nicht ohne Grund auch als \enquote{make-or-break step} bezeichnet. 
Wenn die Verhandlungen an dieser Stelle scheitern, dann wird der Fragebogen höchstwahrscheinlich ein Misserfolg.
Daher ist es wichtig, dass eine ehrliche und offene Kommunikation zwischen den Parteien angestrebt wird.

Bei der dritten Säule der Entwicklung eines Fragebogens geht es um den eigentlichen Kern einer Befragung, dem Fragebogen.
Diese Aufgabe ist von mehreren Parteien zu bewältigen, da ein objektiv entwickelter Fragebogen entstehen soll.
Dieser wird sonst durch die subjektive Meinung einzelner Individuen beeinflusst\autocite[S.21]{2009Fragebogen}.

Laut Caren Goldberg, HR Professorin, ist die Antwortrate bei größeren Unternehmen niedriger als bei kleinen Unternehmen.
Dies lässt sich durch die unterschiedlichen Abteilungen und Bereiche begründen.
So kann eine Befragung nicht das Gefühl der Dringlichkeit erzeugen, wenn es nicht direkt mit den Aufgaben verbunden ist. 

Um ein Gefühl der Dringlichkeit und Hilfsbedürftigkeit zu erzeugen, ist es von Vorteil den Mitarbeiter direkt anzusprechen.
Durch eine genaue Schilderung des Umfrageziels und die Erläuterung der Beweggründe kann ebenfalls noch Verständnis für das Problem hervorgerufen werden.

Des Weiteren gibt es noch andere Grundvoraussetzungen, die beachtet werden müssen.
Das Timing der Umfrage lässt sich in unterschiedliche Kategorien unterteilen.
Es gibt geplante bzw. terminierte Umfragen, welche z. B. für eine jährlich wiederkommende Mitarbeiterbefragung verwendet werden.
Im Gegensatz dazu gibt es noch Umfragen, welche aus einen bestimmten Drang heraus evaluiert werden.
Damit können spezielle Probleme, Herausforderungen oder Möglichkeiten in Angriff genommen werden.
Die Befragung bzgl. des IT Showrooms ist durch ein spezielles Anliegen und zukünftiges Problem entstanden und wird deshalb auch unabhängig von einem Zeitplan präventiv durchgeführt\autocite[Timing is Everything]{Perfect}.
%https://proquest-tech-safaribooksonline-de.ezproxy-dhma-1.redi-bw.de/book/communications/writing/9780071664011/chapter-4-putting-it-together/ch04lev1sec1_html

Außerdem spielen Faktoren wie die Länge des Fragebogens eine Rolle. Je schneller eine Umfrage zu beantworten ist, desto kleiner ist die Hemmschwelle vor dem Reagieren auf die Umfrage.
Durch eine Garantie der Anonymität der Antworten, kann ein Teilnehmer an der Umfrage ebenfalls seine Gedanken wahrheitsgemäß ausdrücken.
Ihm wird die Angst genommen, dass seine Antwort negativ mit ihm assoziiert wird\autocite[Chapter 4
Putting It Together]{Perfect}.

Weitere Aspekte, die in das Design eines Fragebogens mit einwirken, sind die Stichprobengröße, Altersverteilung der Teilnehmer und Preise für die Beantwortung des Fragebogens. Ebenfalls ist es wichtig, ob es eine Onlineumfrage ist, wie deren zeitlichen Rahmenbedingungen gesteckt sind sowie als auch der Zeitrahmen zum Beantworten der Umfrage\autocite[Chapter 4
Putting It Together]{Perfect}.


Wenn die Designphase des Fragebogens beendet ist, kommt als nächster Schritt das Versenden der Umfrage.
Hierbei sind jedoch einige Dinge zu betrachten. Das Anschreiben muss so aufgebaut sein, dass die wichtigen Informationen bereits enthalten sind. 
Schon vor dem Ausfüllen der Umfrage muss klar gemacht werden, was das Ziel des Fragebogens ist, wer die Befragung durchführt und wie die Ergebnisse verwendet werden sollen. Vielleicht steht die Umfrage auch im Interesse der Befragten. Dieser Aspekt muss herausgestellt werden. 
Weiterhin geht es um das Fördern und Fordern des Engagements. 
Durch das Suchen von Hilfe, dem Angeben der benötigten Zeit und dem Hinweis auf die Anonymität können Hemmschwellen und Bedenken beseitigt werden. 
Dem Empfänger muss die Angst genommen und dabei auch noch das Interesse an der Umfrage geweckt werden. 
Extrinsische Anreize können helfen, um eher eine Rücksendung zu erhalten.
Falls es einen Grund gibt, warum spezielle Empfänger ausgewählt wurden, sind diese auch zu benennen.
Diese Phase ist beendet, wenn entweder keine Antworten mehr abgegeben werden oder die Frist zum Beantworten des Fragebogens überschritten wurde\autocite[S.29]{MwdeF}.

%https://link.springer.com/content/pdf/10.1007/978-3-663-09178-3.pdf S.29

Nach dem Abschließen der Umfrage geht es an das Analysieren der Ergebnisse. 
Zuerst ist an dieser Stelle das Sichten der Daten am wichtigsten.
Informationen, die keinen Mehrwert für die Auswertung bringen, können direkt herausgefiltert werden.
Darunter fallen jedoch nicht Informationen, die unerwünscht für das erhoffte Ergebnis sind, sondern Antworten, die tatsächlich keinen Wert besitzen.
So besteht bei einer offenen Frage weiterhin die Möglichkeit diese zu umgehen, indem z. B. sinnlose Aneinanderreihungen von Buchstaben und Zahlen in dieses Feld eingetragen werden.

Genauer wird dieses Thema jedoch im Kapitel 4 beschrieben, da wird dieses Thema näher behandelt und die Möglichkeiten der Auswertung in Detail erläutert\autocite[S.45]{MwdeF}.	%https://link.springer.com/content/pdf/10.1007/978-3-663-09178-3.pdf S.45

\subsubsection{Auswahl der Stichprobe und Anforderungen}
Dieses Kapitel behandelt die Stichprobenauswahl. Dabei geht es genauer darum, wie festgelegt wird, welche Personen an einer Umfrage teilnehmen.
Je nach Anwendungsfall kann die Auswahl einer kleinen Gruppe von Personen mit expliziten Expertenwissen geeignet sein, dies sind dann qualitative Befragungsmethoden. Es gibt aber auch die Möglichkeit eine große Menge an Testpersonen zu befragen, dies ist dann eine quantitative Befragung.
Ebenfalls steht in diesem Kapitel der Befragte im Vordergrund und welche Rechte und Pflichten dieser im Rahmen einer Befragung besitzt. 

%MAYER, Horst Otto. Interview und schriftliche Befragung: Grundlagen und Methoden empirischer Sozialforschung. Walter de Gruyter, 2012. S 60
\enquote{Stichprobe[n] [sind] so auszuwählen, dass die Werte der interessierenden Merkmale in der Stichprobe sich möglichst wenig von der Grundgesamtheit unterscheiden.} \autocite{Interview}


%S 51ff
Die Stichprobenauswahl kann sich direkt auf die Qualität der Befragung auswirken.
Es gibt die Möglichkeit, quantitative und qualitative Methoden der Umfrage anzuwenden.
Meist ist eine Befragung der Grundgesamtheit aus Kostengründen nicht möglich. 
Da eine Erhebung mittels einer onlinegestützten Umfrage die Möglichkeit darbietet, Informationen simultan von verschiedenen Personen zu bekommen, skaliert der Aufwand einer quantitativen Befragung nicht mit der Anzahl der Antworten\autocite{Multi}. %Backhaus, K./Erichson, B./Plinke, W./Weiber, R., Multivariate Analysemethoden –Eine anwendungsorientierte Einführung, 15. Auflage, Berlin/Heidelberg 2018

Da die Stichprobe ebenfalls direkt die Grundgesamtheit abbilden kann, ist es wichtig, herauszufinden, wer Teil der Grundgesamtheit ist. 
Wenn es möglich ist, aus der Grundgesamtheit jeden zu quantifizieren, handelt es sich um eine offene Grundgesamtheit. 
Da die Daten nach dem Einzelbesuch im IT Showroom anonymisiert werden, sind diese jedoch nicht mehr direkt zugänglich und somit ist die Grundgesamtheit geschlossen.	


Es gibt verschiedene Möglichkeiten um die Besucher des IT Showrooms zu befragen, ein Weg kann das direkte Anschreiben der Bucher einer Guided Tour sein, da diese im System hinterlegt sind.
Durch das Anschreiben eines Guided Tour Buchers können nicht direkt alle Personen angesprochen werden, aber die Nachricht kann weitervermittelt werden. 
Da diese die Organisation der Gruppe übernehmen, besitzen sie auch die Kontaktdaten der Besucher aus ihrem direkten Umfeld.
Das Problem hinter dieser Methode ist, dass einige Personen gar nicht erst infrage kommen, um eine Antwort zu geben. 
Die Personen, welche den Fragebogen weitergeleitet bekommen, sind zwar in gewisser Weise zufällig, jedoch trotzdem eingeschränkt auf eine einzelne Gruppe.  
Neben den Besuchern einer Guided Tour gibt es noch die Personen, die einzeln im IT Showroom waren und die Personen, die gar nicht da waren. 
Da jedoch die direkte Meinung zum Inhalt der einzelnen Stationen befragt wird, fallen alle Mitarbeiter heraus, die noch nicht im IT Showroom waren. 
Jedoch gibt es noch die Gruppe der Einzelbesucher, auf die Daten dieser Personen kann leider nicht zugegriffen werden. 
Die Daten sind nur für kurze Zeit im System hinterlegt und werden aus Sicherheitsgründen nach dem Beenden aller Stationen im IT Showroom anonymisiert. 
In anderen Worten sind diese Personen nicht quantifizierbar und können somit nicht befragt werden.
Mit diesem Fehler in der Auswahl der Stichprobe muss gerechnet werden. 
Die Ergebnisse werden sich in Richtung der Besucher einer Guided Tour verschieben. 
Durch das Erlebnis einer solchen Tour, werden voraussichtlich positivere Meinungen abgebildet, als bei den Einzelbesuchern.

Da die dargestellte mögliche Stichprobenauswahl aufgrund von einer eingeschränkten Grundgesamtheit nicht perfekt ist, muss beleuchtet werden, wie die Auswahl der Stichprobe mit einem ausreichenden Datensatz aussieht.
Die wichtigste Voraussetzung dabei ist, dass jede Person einzeln zu identifizieren ist.
Da die SAP SE sich in verschiedene Vorstandsbereiche, wie z. B. die \enquote{Human Ressources} aufteilt, welche sich wiederum in Funktionsbereiche aufteilen, kann aus diesem Konstrukt potentiell eine firmenweite Grundgesamtheit gebildet werden.
Angenommen, dass Personen aus allen Bereichen Besucher im IT Showroom waren und die Distribution der einzelnen Funktionen ebenfalls als gegeben gilt, dann bietet sich eine Klumpenauswahl an\autocite{Stichprobe}. %https://madoc.bib.uni-mannheim.de/50935/1/HQM02-Planung-von-Stichprobenerhebungen.pdf S.25
Durch das Bilden einer kleinen Menge von Personen, die das komplette Kollegium repräsentativ abbilden, können die Wünsche und Gefühle bzgl. des IT Showrooms dargestellt werden.
Wichtige Faktoren, die mit hineinspielen, sind die bereits erwähnten Vorstandsbereiche und Funktionen, sowie Gehalt, Alter, Jahre bei SAP und Vorkenntnisse in den Themen des IT Showrooms.
Das Gehalt der Mitarbeiter spielt eine Rolle, weil sich dies direkt auf die Mitarbeiterzufriedenheit auswirkt und diese wiederum auf die allgemeine Zufriedenheit des Mitarbeiters.
Dadurch wird wohlwollender geantwortet und die Ergebnisse fallen positiver aus\autocite{Winter}.
%https://madoc.bib.uni-mannheim.de/862/1/Winter.pdf S. 28
Das Alter ist notwendig in die Betrachtung mit einzubeziehen, da es die Fähigkeit und die Vorkenntnisse beeinflussen kann, wie gut sich an neue Technologien adaptiert wird\autocite{prensky2001digital}. Eine Grundeinstellung, positiv oder negativ, kann sich manifestieren.
Die Vorkenntnisse in den Themen des IT Showrooms und die Jahre bei SAP beeinflussen durch die Erfahrung in der Branche den Kenntnisstand der vorgestellten Themen im IT Showroom.
Bei der Auswahl der Klumpen ist jedoch darauf zu achten, dass die Klumpen heterogen sind und genau erfasst werden, welche Faktoren die Grundgesamtheit ausmachen. 
Wenn dies nicht der Fall ist, kann es zu Stichprobenfehlern kommen, welche allein durch das ausgewählte Stichprobenverfahren entstanden sind.
Diesen Effekt nennt man auch den Klumpeneffekt.
Eine einfache Zufallsauswahl kann in solchen Fällen die Grundgesamtheit besser abbilden als ein Klumpen.

Unter der Annahme, dass innerhalb des Klumpens jede Person auf den Fragebogen antwortet, lässt sich anhand der Slovin-Formel, die Stichprobengröße errechnen. Von den insgesamt 4.600 Besuchern innerhalb des IT Showrooms waren 1.300 Teilnehmer an einer Guided Tour. 

\hspace*{12mm}%
Slovin-Formel: $\frac{ Populationsgrösse }{ (1 + Populationsgrösse \cdot Fehlermarge^{ 2 }) } = Stichprobengrösse$

Da wir nur diese 1.300 Teilnehmer potenziell befragen können, ist dies die Populationsmenge\autocite{SapReporting}. %interne Quelle des Reportingtools für IT Showroom
Die Fehlermarge wird mit etwa 10\% betitelt, da diese Genauigkeit für die Ergebnisse ausreichend sind. 
Nachteil der Slovin-Formel ist, dass sie den Sachverhalt vereinfacht darstellt und somit keinen großen Spielraum für weitere Einblicke bietet. 
Da diese Betrachtung nur einen Hinweis geben soll, wie viele Personen an der Befragung am Ende teilnehmen sollen, ist dies ausreichend, um einen ersten Einblick zu erhalten.
Anhand der gegebenen Werte kommt eine Stichprobengröße von 93 Personen heraus.
Dies ist der Zielwert, welcher erreicht werden soll, um als repräsentative Stichprobe zu gelten.

Wenn die Stichproben mit einem geeigneten Stichprobenverfahren durchgeführt sind, muss ein Fragebogen entwickelt werden. 
Dieser wird an die Personen versendet und hoffentlich ausgefüllt. 
In Kapitel 3.3 wird dieses Thema näher erläutert. 
Der Empfänger eines Fragebogens besitzt unterschiedliche Aufgaben.

Dabei muss vereinfacht betrachtet der Fragebogen \enquote{nur beantwortet} werden.
Jedoch gibt es Aufgaben und Anforderungen, welche während dieses Prozesses bewusst oder auch unterbewusst zu erledigen sind.


Der Prozess des Beantwortens einer Frage lässt sich generell in fünf Schritte einteilen.
Diese Einteilung ist in der Abbildung \ref{fig:Frage} dargestellt.
Erst muss die gestellte Frage verstanden werden, dies teilt sich in einen pragmatischen und semantischen Teil ein. 
Als nächster Schritt werden relevante Informationen aus dem Gedächtnis abgerufen und auf dieser Grundlage kann ein Urteil gebildet werden. 
Dieses Urteil ist sehr auf die individuellen Erfahrungen einer Person gestützt und wird vermutlich von Person zu Person etwas unterschiedlich sein. 
Nachdem dieses Urteil festgesetzt wurde, wird dies in ein Antwortformat umgemünzt und für das Verständnis des Auswertenden nach den Grundlagen der kooperativen Kommunikation ausgewertet\autocite[S.19ff]{2014Fragebogen}.%2014 Book Fragebogen s19 ff

Zuerst muss die Frage von dem Beantworter verstanden werden. 
Dabei sind die semantischen sowie pragmatischen Werte wichtig, welche solchen Fragen innewohnen.
Als semantische Komponente einer Frage wird das Verstehen der Frage angesehen.
Eine Frage kann auf verschiedenste Weisen verwirrend für den Leser sein. 
Das kann von einer unklaren, schwierigen oder auch mehrdeutigen Formulierung bis hin zu unterschiedlich bzw. individuell interpretierten Fragestellungen alles beinhalten. 

\begin{quote} \enquote{Wie hat dir der Ausstellungsraum gefallen?} \end{quote}

Diese Frage ist in vielerlei Hinsicht missverständlich formuliert. 
Durch die offene Fragestellung und der unkonkreten Spezifizierung des Fragezieles kann sich die Antwort entweder auf den Inhalt oder die Gestaltung des IT Showrooms beziehen. 
Hierbei ist jedoch überhaupt fraglich, ob aus dem Kontext heraus klar wird, welcher Ausstellungsraum angesprochen wird. 
Durch individuelle Erfahrungen mit einem anderen Ausstellungsraum können einzelne Personen und sogar Gruppen diese Frage falsch interpretieren.

Eine bessere Formulierung der Frage muss diese Faktoren mit einbeziehen.
\begin{quote} \enquote{Wie hat dir der Inhalt des IT Showrooms in Walldorf bei der SAP SE gefallen?} \end{quote}

In der überarbeitenden Version wird genau konkretisiert, um welches Objekt es sich handelt, so können Verwirrungen vermieden werden.

Die pragmatische Komponente einer Frage lässt sich darauf eingrenzen, welche Informationen der Fragestellende wissen möchte. 
Um am Beispiel der Fragestellung über den IT Showroom und dessen Inhalt zu bleiben, bezieht sich es jetzt nicht auf das Verstehen der Frage, sondern auf das Interpretieren der Fragestellung und dem Sondieren der Informationen, die der Fragesteller erwartet. 
Der Antwortende hat eine Meinung bezüglich des Themas, welche auf Erfahrungen und Erlebnisse gestützt sind. 
Nach einer kurzen Bedenkzeit wird er aus diesen Erfahrungen seine Meinung bilden können.

Da nach dem Verstehen der Frage und dem Finden einer möglichen Antwort immer noch nicht geklärt ist, ob die Antwort auch passend für die Fragestellung ist, gibt es Grundregeln der kooperativen Kommunikation\autocite{Grice}. %H.P. Grice 1975
Zu diesen gehören vier Maxime, diese teilen sich auf in Qualität, Quantität, Zusammenhang und Methode.

Zur Maxime der Qualität gehört, dass Aussagen wahrheitsgemäß abgegeben werden. 
So darf das Ergebnis nicht verfälscht werden, weil dabei die Hoffnung besteht, so einen größeren positiven oder negativen Effekt erzeugen zu können.
Eine Radikalisierung der Antwort zu diesem Zweck ist nicht von Vorteil, weil es ein falsches Bild signalisiert.

Die Maxime der Quantität bezieht sich darauf, dass nur Informationen mitgeteilt werden, welche auch inhärent wichtig für die Antwort sind. Das Zitat von Thom Renzie beschreibt gut, was unnötige Informationen für einen Effekt mit sich bringen.

\begin{quote}\enquote{Verwirrung lässt sich wunderbar stiften, indem man die Informationsmenge erhöht.} Thom Renzie \end{quote}

Bei der Maxime des Zusammenhanges müssen die Informationen den Verlauf des Gespräches fördern. 
Aussagen die nicht direkt auf das Gesprächsziel abzielen oder haltlos ohne Zusammenhang getroffen wurden, sind nicht förderlich. 

Die letzte Maxime ist der Methode zuzuordnen.
Die Methoden ist mit der Gesprächstechnik gleichzusetzen. 
Es ist wichtig, eindeutige Aussagen zu treffen, die verständlich formuliert sind. 

Da nicht nur die Kommunikation Regeln unterliegt, sondern die Evaluation einer Meinung mittels eines Fragebogens an sich, ist es wichtig, Standards und gesetzliche Regeln einzuhalten. 
In Bezug auf die Forschungsethik und den Datenschutz gibt es neben dem Art. 5 DSGVO\autocite{DSGVO} %https://dsgvo-gesetz.de/art-5-dsgvo/
auch Standards, die sich speziell auf die Erhebung von Umfragen beziehen.

\begin{quote} \enquote{Evaluationen sollen so geplant und durchgeführt werden, dass Sicherheit, Würde und
		Rechte der in eine Evaluation einbezogenen Personen geschützt werden.} Fairnessstandard 2: Schutz individueller Rechte\autocite{DegEval} %https://www.degeval.org/degeval-standards/kurzfassung/
	\end{quote}

Daraus ergeben sich die wichtigen Teilaspekte der Freiwilligkeit, Vertraulichkeit und Anonymität für die Durchführung\autocite[S.57]{2009Fragebogen}. %2009EvalOnline S.57
Diese sind unter allen Umständen zum Schutz der Antwortenden, Fragesteller und den Ergebnissen des Fragebogens einzuhalten.
Diese Grundsätze und die gelegten Grundlagen müssen auch im Kapitel \vref{design} mit in das Design einbezogen werden. 

\pagebreak
\subsubsection{Inhaltliches Design}
\label{design}
Dieses Kapitel behandelt den Aufbau eines Fragebogens. 
Es wird näher darauf eingegangen, welche Struktur ein Fragebogen besitzen muss, um den Benutzer durch den kompletten Fragebogen hinweg animiert zu halten. 
Weiterhin werden die verschiedenen Fragentypen erläutert und deren Einsatzmöglichkeiten verdeutlicht. 

Da ein Fragebogen sich nicht nur auf die Befragung an sich bezieht, sondern auch auf den Prozess, der diese Umfrage begleitet, fängt die Aufgabe des Designs eines Fragebogens bereits bei dem Anschreiben an.
Darin sollen Informationen wie Name und die Adresse des Absenders, Thema der Befragung, Zusammenhang zum Thema, Anonymitätsgarantie für den Beantwortenden, Rücksendetermin, Begründung für die Auswahl der Empfänger und die benötigte Zeit für das Ausfüllen enthalten sein\autocite[S.29]{2003Fragebogen}. %2003 Fragebogen S.29
All diese Faktoren haben einen Einfluss auf die Bereitschaft, den Fragebogen zu beantworten. 
Der zeitliche Faktor spielt dabei schon in dem Anschreiben eine sehr wichtige Rolle. 
Bereits wenn die Umfrage zehn Minuten dauert, öffnen 25\% der Personen den Fragebogen gar nicht.
Dieser Effekt nimmt bei längeren Umfragen immer weiter zu\autocite[S.353]{NFP}. %nfp031 S353
15 Minuten sind meist das Höchstmaß an Bereitschaft für die Beantwortung einer Umfrage\autocite[S.37]{2009Fragebogen}. %2009_BookEvalOnline 37
Aus diesem Grund ist eine kürzere, prägnante Umfrage eine Möglichkeit, um die Antwortrate zu erhöhen.

Da der Faktor Zeit eine große Rolle spielt, muss die Zeitangabe akkurat sein.
Es kann zu Frustration führen, wenn die Zeitangabe geringer ist als die tatsächliche Befragung und damit verbunden einen Abbruch des Fragebogens einher ziehen. 
Durch eine Fortschrittsanzeige wird dem Beantwortendem verdeutlicht, inwieweit er in seinem Beantwortungsprozess fortgeschritten ist. 
Da es in der Onlinewelt einen mittlerweile gut ausgeprägten \enquote{geschulten Blick für Dilettantismus} gibt, muss diesem gerecht werden.
Schreibfehler und Ungereimtheiten müssen vor dem Versenden bereits ausgemerzt sein\autocite[S.37]{2009Fragebogen}. %2009_BookEvalOnline 37

Um die Rückläuferquote weiter zu erhöhen, hilft es, eine Ankündigung des Fragebogens zu versenden. 
Je nach Wichtigkeit der Umfrage kann das dazu führen, dass die Befragten in gewisser Weise auf die Umfrage mental vorbereiten\autocite[S.38]{Umfragenforschung}.
%https://link.springer.com/content/pdf/10.1007%2F978-3-531-91852-5.pdf S38
Da eine online gestützte Umfrage durchgeführt wird, kann dem Mode-Effekt entgegengewirkt werden. In Interviews kann es zur Verfälschung der Ergebnisse kommen, da in einem direkten Dialog mitgeteilt werden muss, was die teilweise unerwünschte Wahrheit ist\autocite[S.163]{Umfragenforschung}.
%S163

Wenn die Basis gelegt ist, muss der Fragebogen entwickelt werden. 
Dabei ist der Aufbau des Fragebogens der nächste wichtige Aspekt.  
Der Fragebogen wurde, wie in der Abbildung \ref{fig:er} dargestellt, aufgebaut. 
Zuerst wird mit einer Einleitung zum Thema und Inhalt des Fragebogens hingeführt. 
Danach folgen Anleitungen und Hinweise zum Ausfüllen des Fragebogens.
Dabei wird der Anonymitätsfaktor herausgestellt und die Verfahren zum Ausfüllen der Fragen behandelt.

Im nächsten Schritt, auch Recall genannt, werden die Erinnerungen an den Besuch im IT Showroom nochmals aufgefrischt. 
Konkret werden die Inhalte der einzelnen Stationen zusammengefasst aufgeführt.

Im Hauptteil der Befragung wird zuerst eine Sprungfrage implementiert, diese führt als Abkürzung zu den Fragen bzgl. des zukünftigen IT Showrooms. Dabei gilt dies aber nur für Personen, die nicht persönlich den IT Showroom erlebt haben. 
Für alle Leute, die im IT Showroom waren, werden dann Fragen zum aktuellen IT Showroom gestellt. Diese Fragen sind hauptsächlich geschlossene Fragen und haben die Aufgabe als Aufwärmfragen zu fungieren.

Der letzte Teil der Befragung ist am wichtigsten für die restliche Arbeit. 
Hier sind offene Fragen gestellt, die mit einer Freitextantwort beantwortet werden sollen.
Der komplette Fragebogen dient letztendlich zum Hinführen zu dieser einen Frage.
Die Freitextfrage beschäftigt sich mit den Wünschen der Mitarbeiter bezüglich der kommenden Inhalte des IT Showrooms.


Nachdem die Grundstruktur erläutert ist, wird es wichtig, einen Blick in die Fragestrukturen zu werfen.
 
Um Informationen zu erhalten, gibt es verschiedenste Fragearten, die ausgewählt werden können.
Egal, ob es sich um eine geschlossene oder eine offene Frage handelt\autocite[S.31]{2009Fragebogen}. %2009Fragebogen S.31
Beide Fragearten verfolgen ein Ziel, das Sammeln von wertvollen Informationen.
Das Zitat aus den Standards für Evaluationen beschreibt den Sachverhalt weiter und spiegelt wieder, dass eine durchdachte Fragestellung ebenso wichtig ist.

\begin{quote} \enquote{Erhebungsverfahren und Datenquellen sollen so gewählt werden, dass die Zuverlässigkeit der gewonnenen Daten und ihre Gültigkeit bezogen auf die Beantwortung der Evaluationsfragestellungen nach fachlichen Maßstäben sichergestellt sind. Die fachlichen Maßstäbe sollen sich an den Gütekriterien der empirischen Forschung orientieren.} Genauigkeitsstandard 5: Valide und reliable Informationen\autocite{DegEval} %https://www.degeval.org/degeval-standards/kurzfassung/
\end{quote}

Offene Fragen geben keine Antworten vor. Die befragte Person soll mit ihren eigenen Worten Stellung beziehen.
Durch das Offenlegen von Motiven, Begründungen, Verbesserungsvorschlägen bezüglich des Evaluationsgegenstandes können die Forscher Erkenntnisse gewinnen.
Der Befragte und dessen Sichtweise auf das spezielle Thema wird in den Vordergrund der Betrachtung gestellt\autocite[S.31]{2009Fragebogen}. %2009Fragebogen S.31
Ein motivierender Effekt stellt sich ein, da die Meinung der einzelnen Person sehr wertgeschätzt wird. 
Jedoch ist erst zu prüfen, ob eine Notwendigkeit besteht, um eine offene Frage zu formulieren. 
Wenn die detaillierte Meinung eines Umfrageteilnehmers zu einem Thema abgefragt werden soll, dann ist eine offene Frage wertvoller. 
 

Durch das Verwenden von geschlossenen oder auch standardisierten Fragen werden die Beantwortenden eingeschränkt.
Das Spektrum der möglichen Antworten muss dabei vorher vollständig bekannt und auch abgegrenzt sein\autocite[S.66]{2014Fragebogen}. % (s66 2014 book fragenbogen)
Durch Einfachnennung oder Mehrfachnennung gibt es die Möglichkeit innerhalb des Spektrums der geschlossenen Fragen eine Variation der Fragestellungen darzustellen\autocite[S.31]{2009Fragebogen}. %2009Fragebogen S.31
Da es bei Fragen mit Skalen auch nur die Möglichkeit gibt, in einem vordefinierten Bereich seine Antwort abzugeben, sind es prinzipiell auch geschlossene Fragen.
Jedoch ist die Zuordnung in der Auswertung bei vorgefertigten Antwortmöglichkeiten leichter.

Da bei offenen Fragen die Zuordnung in Themengebiete geschieht, an welche vorher nicht gedacht werden konnte, sind selbst in Fällen, wo scheinbar alle Möglichkeiten durchdacht sind, immer noch abwegige Antworten möglich.
Da es eine Freitextfrage ist, wird der Antwortende nicht an der Abgabe dieser gehindert. 
Als Beispiel dient eine Applikation zum Darstellen des Essensangebotes, in dieser besteht die Möglichkeit, per Freitextfrage Feedback zur dieser abzugeben.
Das Feedback soll sich laut der Fragestellung direkt auf die Applikation beziehen.
Trotzdem wird es Beantwortende geben, welche die Auswahl der Gerichte in einer Kantine bewerten.


Im Anhang \vref{list:gebote} sind die zehn Gebote der Fragenformulierung definiert\autocite{RolfPorst}. %RolfProstFragen S. 689
Diese sind von R. Porst aufgestellt und sollen als Leitfaden gelten, um verständliche Fragen zu formulieren.
Da es aber kein allgemeines Kochrezept ist, um Fragen zu designen, sind diese Regeln in den meisten Fällen schnell wieder vergessen und werden deshalb nicht richtig angewandt.%RolfProstFragen S.698

Fragen können in geschlossene und offene Fragen eingeteilt werden.
Da der Aufbau von geschlossenen Fragen jedoch bisher nur mit der Anwesenheit von vorgegebenen Antworten definiert wurde, ist es nötig, einen Blick auf die möglichen Skalen zu werfen.

Mögliche Skalen, die verwendet werden, können sich in nominale, ordinale,\\
 Intervall-, Ratio-, verbalisierte und endpunktbenannte Skalen einteilen\autocite[S.71]{2014Fragebogen}. %2014 Fragebogen S71.
Skalen mit einer geraden Anzahl von Antworten führen dazu, dass es keine mittlere Meinung gibt. 
Der Antwortende wird durch das Design der Frage gezwungen, sich für eine Seite, ob positiv oder negativ, zu entscheiden.
Unbeachtet davon, ob der Skalenmittelpunkt wirklich der Skalenmittelpunkt oder nur der Skalenpunkt in der Mitte ist, neigt der Beantwortende dazu, diesen als solchen zu interpretieren\autocite[S.83]{2014Fragebogen}.%2014 Fragebogen S.83

Unabhängig davon, ob eine Skala gerade oder ungerade ist, kann durch die Breite dieser ein großer Effekt erzeugt werden. 
Eine zu schmale Skala führt dazu, dass nicht genügend differenziert werden kann.
Im Gegenzug dazu kann eine zu breite Skala dazu führen, dass die Grenzen verschwimmen.
Es wird für Fragesteller und Beantworter schwierig, die einzelnen Schritte benennen zu können. 
Als extremes Negativbeispiel ist eine Skala mit 20 möglichen Skalenpunkten zu sehen.
Es ist fraglich, ob eine Unterscheidung zwischen den Skalenpunkten 16 und 17 noch möglich ist.
Darum gibt es die Faustregel bei endpunktbenannten Fragen nicht weniger als fünf Punkte und nicht mehr als neun Skalenpunkte anzubieten\autocite[S.87]{2014Fragebogen}. %(2014 Book Fragebogen S 87)
Intuitiv gesehen sollen Fragen ebenfalls durch die Gliederung von niedrig/negativ nach positiv/hoch leichter verständlich sein. Eine einheitliche Einteilung ist vom Vorteil, um den Benutzer nicht mit Umdenken in ein neues System zu belasten\autocite[S.89/90]{2014Fragebogen}. %(2014 Book Fragebogen S89/90)

Auf dem ersten Blick ist das Wohlbefinden eines Benutzers beim Beantworten des Fragebogen nicht inhärent wichtig. 
Jedoch ist User Experience ein Konzept, um den Beantwortenden nicht zu überfordern und keine Zeit mit der Verwirrung zu verschwenden.
Aus diesem Hintergrund haben sich in der Umfragepraxis numerische (endpunktbenannte) Likert-Skalen bewährt\autocite{Likert}. %https://de.statista.com/statistik/lexikon/definition/82/likert_skala/ 
Likert-Skalen werden meistens mit einer ungeraden Anzahl an Merkmalausprägungen verwendet. Der Skalenmittelpunkt ist dabei direkt mit der neutralen Antwort gleichzusetzen\autocite{ISO}. \label{Likert} 
%https://www.procontext.de/aktuelles/2010/03/iso-9241210-prozess-zur-entwicklung-gebrauchstauglicher-interaktiver-systeme-veroeffentlicht.html

\begin{quote}
	\enquote{User Experience umfasst demzufolge alle Effekte, die ein Produkt bereits vor der Nutzung (antizipierte Nutzung) als auch nach der Nutzung (Identifikation mit dem Produkt oder Distanzierung) auf den Nutzer hat. Usability wiederum fokussiert auf die eigentliche Nutzungssituation (Effektivität und Effizienz).}\autocite{User}%https://www.researchgate.net/publication/256495056_User_Experience_mit_Fragebogen_messen_-_Durchfuhrung_und_Auswertung_am_Beispiel_des_UEQ
\end{quote}

In einem Interview kann die offene Frage so formuliert werden, dass diese an den Gesprächsfluss gut angepasst ist.
Diese Möglichkeit besteht jedoch bei einem Fragebogen nicht, von daher müssen die Fragen vorher gut durchdacht werden und einen besonderen Fokus erhalten.
Des Weiteren sind offene Fragen bei der Beantwortung zeitaufwendiger als geschlossene Fragen. 
Die Antwort muss von Grund aus überdacht und nicht nur abgewägt werden. 
Daher ist eine präzise, verständliche und mit zusätzlichen Informationen ausgeschmückte Formulierung wichtig. 
Da zu viele offene Fragen den Probanden abschrecken, ist eine Mischung aus geschlossenen und offenen Fragen wichtig.
Bevor eine Freitextfrage gestellt werden kann, sind Erinnerungsfragen hilfreich. Diese führen zum Thema hin und helfen dem Antwortenden bessere Antworten zu finden\autocite[S.35]{2009Fragebogen}.%2009_BookEvalOnline 35

%2009Fragebogen S.36 in Struktur einarbeiten

Der Fragebogen wurde nach der in diesem Kapitel erläuterten Theorie aufgestellt und im Anhang der Abbildungen \vref{fig:SurveyStart} bis \vref{fig:Survey6} abgebildet.
Die Reihenfolge der Fragen sind von Abbildung \ref{fig:SurveyStart} bis \ref{fig:Survey6} in Reihenfolge der Darstellung im Anhang angeordnet.
Es handelt sich um eine Umfrage, die mit Google Forms durchgeführt wurde.
Zusätzlich ist zu erwähnen, dass die Seitenanzahl, wie in \vref{fig:Survey1} dargestellt, sich nur auf die einzelnen Abschnitte bezieht.
Um eine bessere Auswertung zu ermöglichen gibt es mehr Abschnitte, die als Fragen gestellt werden. Dies kann den Benutzer zwar verwirren, aber da nun das Feedback aus \vref{fig:Survey4} sich direkt auf \vref{fig:Survey3} bezieht, ist die Auswertung erleichtert.    
Eine direkte Zuordnung von der Freitextfrage zu der geschlossenen Frage wird direkt durchgeführt. 
Dies ist zwar später möglich, würde aber unnötigen Aufwand bedeuten in der Auswertung.
