% !TEX root =  ../theoretisch.tex
\subsection{Fragebogen}
"Dem guten Frager ist schon halb geantwortet" (oder sowas) F. Nietzsche
- Was muss beachtet werden, damit ein Fragebogen gestellt werden kann?
  - Fragebogen ist nicht individualisiert
  - klares Ziel 
  - alle nötigen Informationen in Fragebögen enthalten
  - Datenschutz / Anonymisierung 
  - Aufbau des Fragebogens (Einleitung - Recall - ...)

- Fragetypen (mit Beispielen)
  - offen 
    - dauert länger
    - wahre Meinung des Nutzers 
    - Fragendesign sehr wichtig! 
    - aufwendige Auswertung 
    - qualitative Antworten -> qualitative Auswertung
  - halboffen
    - Basically bullshit, da ein falsches Bild vermittelt wird
    - hat Charakter von geschlossen, aber versucht die Vorteile von offen mit zu kombinieren
  - geschlossen
    - feste Antwortmöglichkeiten
    - schneller beantwortet
    - leicht auszuwerten
    - quantitative Auswertung

- Skalen
  - Likert
  - NPS
  - UEQ

- Stichprobenauswahl -> vielleicht mit ins Tool rein?
  - Slovin (einfach)
- Was haben wir verwendet, bieten wir an?
- Warum ist das Fragebogendesign bei uns besser als bei anderen Tools?
