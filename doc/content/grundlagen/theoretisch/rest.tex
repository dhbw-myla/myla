% !TEX root =  ../theoretisch.tex
\subsection{Representational State Transfer}

Die Technologie der Schnittstellen zur Kommunikation zwischen Softwaresystemen hat sich im Verlauf der Jahre stetig geändert. 
Hierzu gibt es verschiedenste Arten von Schnittstellen. 
Gerade die bekannten Internetunternehmen wie Google\autocite{MS-GoogleLLC.2020}, Twitter\autocite{MS-TwitterInc..01.03.2020} oder Microsoft\autocite{MS-MicrosoftCorporation.21.05.2018} bieten teilweise dutzende öffentliche Schnittstellen in Form von \acp{API} an.

Der wohl bekannteste und derzeit aktuellste Schnittstellentyp von \acp{API} sind auf dem \ac{REST}-Paradigma basierende Schnittstellen. 
Dieses Paradigma bzw. diese Architekturweise wurde von Roy Thomas Fielding in seiner Dissertation\autocite{MS-Fielding.} vorgestellt und wurde ursprünglich für verteilte hypermediale Systeme entworfen. 
\ac{REST} ist selbst kein Standard, sondern beschreibt sechs Prinzipien für eine Web-Architektur. 
Das erste Prinzip beruht auf einer Client-Server-Architektur, bei der die Benutzerschnittstelle von der Datenhaltung entkoppelt wird.
Dadurch soll eine höhere Skalierbarkeit und Portabilität für verschiedene Plattformen gewährleistet werden.
Ebenfalls ist hierdurch eine unabhängige Weiterentwicklung der Komponenten möglich. 
Das zweite Prinzip bezieht sich auf die Zustandslosigkeit (engl. \enquote{stateless}). 
Damit ist gemeint, dass alle Anfragen vom Client an den Server stets alle nötigen Informationen beinhalten, die der Server benötigt. 
Dieser arbeitet somit kontextlos und muss sich keine Sitzungsinformationen speichern. 
Alle Anfragen können somit atomar und isoliert voneinander betrachtet werden. Dadurch erhöht sich die Zuverlässigkeit sowie ebenfalls die Skalierbarkeit.

Der dritte Grundsatz nach \citeauthor{MS-Fielding.} ist der Cache. 
Damit ist das Zwischenspeichern bzw. das Vorhalten von Daten gemeint, die schon einmal benutzt wurden. 
Laut diesem Prinzip sollen die Daten als \enquote{cachable} oder \enquote{not cachable} markiert werden. 
Dadurch soll die gesamte Netzwerkeffizienz verbessert bzw. die Netzwerkauslastung verringert werden. 
Dem Vorteil der Performanceverbesserung steht jedoch die Problematik, das Original auf dem Server mit dem Inhalt des Caches synchron zu halten, gegenüber. 
Veraltete Daten im Cache können gegebenenfalls zu fehlerhaften Ergebnissen führen. 
Das nächste Prinzip bezieht sich auf die Gestaltung der Schnittstellen. 
Diese sollen in einer einheitlichen und standardisierten Form definiert werden. 
Dadurch vereinfacht sich laut \citeauthor{MS-Fielding.} die Gesamtsystemarchitektur. 

Das fünfte Prinzip nach \citeauthor{MS-Fielding.} ist die hierarchische Strukturierung des Gesamtsystems in verschiedene Schichten. 
Hierdurch wird eine Restriktion vorgenommen, bei der jede Komponente des Systems nur mit der unmittelbar angrenzenden Schicht kommunizieren darf. 
Dadurch werden einzelne Dienste von der Außenwelt abgekapselt bzw. von der Umgebung isoliert und die Sicherheit des Systems erhöht. 
Das letzte Prinzip der \ac{REST}-Architektur ist \enquote{Code-on-Demand}.
Dieses Prinzip ist optional und bezieht sich darauf, dass die Client-Funktionalität durch das Herunterladen von ausführbarem Code erweitert werden kann. 

Neben den gerade genannten sechs Prinzipien beschreibt \citeauthor{MS-Fielding.} in seiner Dissertation noch die zu \ac{REST} gehörigen Datenelemente. 
Die wichtigsten Datenelemente sind dabei die Ressource und die Ressourcenbezeichnung (engl. \enquote{Resource Identifier}). 
Eine Ressource stellt dabei eine Abstraktion jeglicher Informationen dar. 
Es ist dabei irrelevant, ob es sich um ein Bild, Dokument oder um komplexe Sachverhalte handelt. 
Der Resource Identifier dient, wie der Name schon sagt, zur eindeutigen Identifizierung einer Ressource. 
Hierzu werden in der Regel \acp{URI} oder \acp{URN} verwendet. 
Ressourcen selbst werden entweder als statisch oder nicht statisch angesehen.
Der Unterschied besteht darin, dass bei einer statischen Ressource stets der gleiche Wert zurückgegeben wird und sich dieser bei einer dynamischen Ressource ändern kann.\autocite[Vgl.][S. 86-90]{MS-Fielding.}

Wie bereits erwähnt, beschreibt \ac{REST} keinen Standard, sondern Architekturprinzipien zur Gestaltung der \acp{API}. 
Zur Umsetzung dieser eignet sich besonders das \ac{HTTP}, welches bereits zu einem Standard ausgearbeitet wurde. 
\acp{API}, die nach \ac{REST}-Prinzipien entworfen sind und mit dem \ac{HTTP}-Protokoll arbeiten, werden \ac{REST}ful-\acp{API} genannt.\autocite[Vgl.][S. 180]{MS-Wolff.2016}
Das \ac{HTTP}-Protokoll stellt die denkbar perfekte Grundlage für \ac{REST} dar. 
Durch die im \ac{HTTP}-Protokoll implementierten Methoden wie GET, POST, PUT und DELETE ist es möglich die bereitgestellten Ressourcen zu verwalten.
Außerdem erfüllt es weitere wesentliche Bedingungen, die von \ac{REST} gefordert werden, wie beispielsweise die Zustandslosigkeit. 
Die aktuelle Version ist \ac{HTTP}/2\autocite{MS-Belshe.2015}, jedoch wird auch die ältere Version \ac{HTTP}/1.1\autocite{MS-Fielding.1999} noch verwendet. 
Die Form der Nutzdaten für die Übertragung bei \ac{REST}ful-\acp{API} ist dabei meist die \ac{JSON} oder \ac{XML}.\autocite[Vgl.][S. 97]{MS-Tilkov.2015} 
