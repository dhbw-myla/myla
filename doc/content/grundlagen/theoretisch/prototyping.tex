% !TEX root =  ../theoretisch.tex
\subsection{Prototyping}

%PLAGIAT
Das Prototyping ist die in dieser Arbeit hauptsächlich verwendete wissenschaftliche Methodik, um das gegebene Ziel zu erreichen.
Beim Prototyping handelt es sich um eine Vorgehensweise in der Softwareentwicklung, die als Ergänzung zu herkömmlichen Softwareentwicklungsansätzen wie dem Phasen- oder Lebenszyklus-orientierten Ansätzen gilt.\autocite[Vgl.][S. 14]{MS-Floyd.1984} 
Zu diesen zählt beispielsweise das Wasserfall- oder V-Modell.

Ziel vom Prototyping ist es, die Kommunikation bei der Softwareentwicklung zu verbessern und dadurch schneller Feedback zu erhalten.\autocite[Vgl.][S. 3]{MS-Floyd.1984} 
Dabei soll bereits im frühen Stadium des Entwicklungsprozesses ein Prototyp der Software erstellt werden, um fehlende und fehlerhafte Anforderungen zu erkennen.\autocite[Vgl.][S. 368]{MS-Alpar.2019}
Nach \citeauthor{MS-Alpar.2019} ist ein Prototyp \enquote{eine frühe ausführbare Version [einer Software], die bereits die relevanten grundlegenden Merkmale des späteren Produkts aufweist.}\autocite[Siehe][S. 369]{MS-Alpar.2019}

Das Prototyping besteht nach \citeauthor{MS-Floyd.1984} im Wesentlichen aus vier Schritten. 
Zunächst müssen die relevanten Funktionalitäten geklärt werden, welche der Prototyp umfassen soll. 
Dabei sollte darauf geachtet werden, dass der Prototyp die Funktionalitäten umfasst, die für etwaige Demonstrationen wichtig sind. 
Außerdem sollte dieser niemals die gesamten Funktionalitäten des finalen Produktes umfassen. 
Im nächsten Schritt wird der Prototyp gemäß der gewählten Funktionalitäten umgesetzt. 
Im Anschluss findet eine Evaluation des Prototyps statt. 
Dabei muss darauf geachtet werden, dass bei der Evaluation alle beteiligten Personen über die gewählten Kriterien und Funktionalitäten im Bilde sind. 
Bei der Evaluation werden Erfahrungen gesammelt und Verbesserungsmöglichkeiten eingeholt. 
Der letzte Schritt dient der Entscheidung, wie mit dem Prototyp weiter verfahren werden soll. 
Dementsprechend wird dieser entweder verworfen oder weiter ausgebaut. 
Demzufolge können die Schritte beliebig oft wiederholt werden.\autocite[Vgl.][S. 4 f.]{MS-Floyd.1984}

Beim Prototyping wird prinzipiell zwischen dem horizontalen und dem vertikalen Verfahren unterschieden. 
Das horizontale Prototyping bezieht sich dabei auf einen gewählten Bereich bzw. eine Schicht einer Architektur, bei der außerdem die Funktionalitäten nicht im Detail implementiert werden. 
Beispielsweise wird nur die grafische Benutzerschnittstelle einer Software entwickelt, die über keine Funktionalitäten der Geschäftslogik verfügt. 
Das vertikale Prototyping bezieht sich hingegen auf alle Schichten einer Architektur. 
Bei diesem wird nur ein ausgewählter Aspekt bzw. eine ausgewählte Funktionalität der Software vollständig implementiert.\autocite[Vgl.][Abschnitt \enquote{Arten von Prototypen}]{MS-Kuhrmann.26.09.2012}$^,$\autocite[Vgl.][S. 4]{MS-Floyd.1984} 

Neben der genannten Unterteilung wird beim Prototyping zusätzlich zwischen dem explorativen, dem experimentellen und dem evolutionären Prototyping unterschieden. 
Beim explorativen Prototyping geht es darum, schrittweise und wiederholt die fachlichen Anforderungen an eine Software einzuholen, wobei iterativ mehrere Prototypen erstellt werden. 
Beim experimentellen Prototyping geht es grundsätzlich um den Nachweis der Realisierbarkeit eines Entwurfs. 
Hierbei wird bereits ein Großteil der benötigten Funktionalitäten umgesetzt. Die letzte Art ist die des evolutionären Ansatzes. 
Hierbei wird davon ausgegangen, dass Anforderungen an das System fehlen bzw. im Verlauf noch hinzugefügt werden. 
Dementsprechend handelt es sich bei dieser Variante um ein iteratives Verfahren, welches die beschriebenen Schritte des Prototyping mehrfach wiederholt, sodass am Ende ein Übergang vom Prototyp zum fertigen Produkt erfolgt.\autocite[Vgl.][S. 370]{MS-Alpar.2019}$^,$\autocite[Vgl.][S. 6-12]{MS-Floyd.1984} 

Die für diese Arbeit gewählte Variante des Prototyping ist das vertikal-experimentelle Prototyping.