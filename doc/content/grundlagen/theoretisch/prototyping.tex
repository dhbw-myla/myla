% !TEX root =  ../theoretisch.tex
\subsection{Prototyping}

Für die Entwicklung dieser Arbeit wurde sich an der Entwicklungsmethode des Prototyping orientiert.
Das Prototyping beschreibt eine Vorgehensweise in der Softwareentwicklung, die eine Ergänzung zu den bekannten Ansätzen wie dem Wasserfall- oder V-Modell darstellt.\autocite[Vgl.][S. 14]{MS-Floyd.1984} 
Beim Prototyping steht die Verbesserung der Kommunikation während der Entwicklung eines Softwareprojektes im Vordergrund.
Hierdurch ist es möglich, schneller Feedback zu sammeln, um fehlende oder fehlerhafte Anforderungen bereits im frühen Stadium des Entwicklungsprozesses zu erkennen.
Hierzu wird ebenfalls schon im frühen Stadium des Entwicklungsprozesses ein Prototyp der Software erstellt.\autocite[Vgl.][S. 3]{MS-Floyd.1984}$^,$\autocite[Vgl.][S. 368]{MS-Alpar.2019} 
Nach \citeauthor{MS-Alpar.2019} ist ein Prototyp \enquote{eine frühe ausführbare Version [einer Software], die bereits die relevanten grundlegenden Merkmale des späteren Produkts aufweist.}\autocite[Siehe][S. 369]{MS-Alpar.2019}

Im Wesentlichen besteht der Prozess des Prototyping aus vier Schritten. 
Als erster Schritt werden die relevanten Funktionalitäten geklärt, die der zu erstellende Prototyp enthalten soll.
Dabei sollte der Prototyp niemals alle Funktionalitäten des finalen Produktes enthalten.
Im nächsten Schritt erfolgt die Umsetzung des Prototyps anhand der zuvor eingegrenzten, festgelegten Funktionalitäten.
Danach erfolgt eine Evaluation des Prototyps.
Bei dieser Evaluation werden Erfahrungen gesammelt und Vorschläge für Verbesserungsmöglichkeiten eingeholt.
Im letzten Schritt wird bestimmt, was mit dem Prototyp passiert.
Dementsprechend wird dieser entweder verworfen oder weiter ausgebaut. 
Weiterhin können die zuvor erläuterten Schritte beliebig oft wiederholt werden.\autocite[Vgl.][S. 4 f.]{MS-Floyd.1984}

Beim Prototyping wird zwischen dem horizontalen und dem vertikalen Verfahren unterschieden. 
Beim horizontalen Prototyping wird eine ausgewählte Schicht einer Architektur als Prototyp entwickelt.
Dabei werden Funktionalitäten nicht im Detail implementiert, sondern nur so, dass der Prototyp vorzeigbar ist.
Beispielsweise wird nur eine grafische Benutzeroberfläche implementiert, die keine Funktionalitäten einer Geschäftslogik aufweist.
Das vertikale Prototyping bezieht sich hingegen auf alle Schichten einer Architektur. 
Bei diesem wird ein ausgewählter Aspekt bzw. eine ausgewählte Funktionalität oder Komponente einer Software vollständig implementiert.\autocite[Vgl.][Abschnitt \enquote{Arten von Prototypen}]{MS-Kuhrmann.26.09.2012}\autocite[Vgl.][S. 4]{MS-Floyd.1984} 

Zusätzlich wird das Prototyping in exploratives, experimentelles und evolutionäres Prototyping eingeteilt.
Beim explorativen Prototyping werden schrittweise und wiederholt neue fachlichen Anforderungen an eine Software eingeholt, wobei iterativ mehrere Prototypen erstellt werden. 
Der Zweck des experimentellen Prototyping ist es, die Realisierbarkeit eines Entwurfs bzw. einer Idee nachzuweisen.
Die letzte Art ist die des evolutionären Ansatzes. 
Bei dieser Art wird davon ausgegangen, dass im Verlauf der Entwicklung weitere Anforderungen an das System gestellt werden bzw. sich vorhandene Anforderungen ändern können. 
Aus diesem Grund wird dieses Verfahren iterativ durchgeführt, sodass der Prototyp stetig weiterentwickelt wird. 
Dabei erfolgt am Ende der Entwicklung ein Übergang von Prototyp zum fertigen Produkt.\autocite[Vgl.][S. 370]{MS-Alpar.2019}$^,$\autocite[Vgl.][S. 6-12]{MS-Floyd.1984} 

Für dieses Projekt wurde das evolutionäre Prototyping gewählt, da das Ziel die Erstellung eines fertigen Produktes ist.