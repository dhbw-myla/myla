% !TEX root =  ../theoretisch.tex
\subsection{Microservice}\label{ch:microservice}

Für die Umsetzung von verteilten Systemen bzw. von Cloud-basierte Anwendungen hat sich in den letzten Jahren die Microservice-Architektur immer weiter etabliert.\autocite[Vgl.][Kapitel \enquote{Monolithic architecture overview}]{MS-Sharma.2016}
Die Verwendung dieser Architektur ist jedoch nicht neu sondern sie wird bereits seit über zehn Jahren vom Unternehmen Amazon genutzt.\autocite[Vgl.][]{MS-Wolff.02.11.2015} 
Die ursprüngliche Idee von Microservices wurde dabei aus dem Ansatz der \enquote{\textit{\ac{SOA}}} hergeleitet.\autocite[Vgl.][S. 1]{MS-Bucchiarone.2018} 
Dabei gibt es für den Begriff Microservice keine konkrete Definition.\autocite[Vgl.][S. 3]{MS-Wolff.2018}. 
Viel mehr wird die Microservice-Architektur als Gegenteil der monolithischen Architektur verstanden, bei der die Software einem einzigen lauffähigen Baustein gleicht.
Dabei umfasst dieser Baustein alle Funktionen der Software, arbeitet autark und wird meist von mehreren Entwicklungsteams gleichzeitig entwickelt.
Dementsprechend ist eine lose Kopplung von einzelnen Bestandteilen nicht möglich.
Bei der Microservice-Architektur hingegen besteht die gesamte Software aus unabhängigen Komponenten, die jeweils von kleinen Teams entwickelt und bereitgestellt werden.\autocite[Vgl.][]{MS-Fowler.25.03.2014} 
Jeder Microservice erfüllt in der Regel dabei nur eine Aufgabe, wie beispielsweise das Bereitstellen einer \ac{REST}-Ressource.

%PLAGIAT

%Der Begriff Microservice bzw. die Microservice-Entwicklung hat in den letzten Jahren immer mehr Aufmerksamkeit bekommen,\autocite[Vgl.][Kapitel \enquote{Monolithic architecture overview}]{MS-Sharma.2016} wobei die Idee nicht neu ist und einige Unternehmen wie Amazon bereits seit über zehn Jahren von der Architektur profitieren.\autocite[Vgl.][]{MS-Wolff.02.11.2015} 
%Besonders durch die wachsende Nachfrage nach Lösungen im Bereich des Cloud Computings\autocite[Vgl.][]{MS-Herrmann.10.02.2017}, hat sich die Nutzung von Microservices als gängige Architektur immer weiter verbreitet. 
%Die Microservice-Architektur ist ursprünglich aus der \acs{SOA} entstanden.\autocite[Vgl.][S. 1]{MS-Bucchiarone.2018} 
%Für den Begriff Microservice gibt es allerdings keine einheitliche Definition\autocite[Vgl.][S. 3]{MS-Wolff.2018}. 
%Am besten lässt sich die Microservice-Architektur als Gegenteil der monolithischen Architektur beschreiben. 
%Bei dieser werden alle Funktionen einer Software in einem einzigen, autark lauffähigen Systembaustein, welcher meist eine eigene Persistenz besitzt, ausgeliefert und entwickelt. 
%Es ist also keine lose Kopplung einzelner Komponenten möglich.\autocite[Vgl.][S. 216]{MS-Vogel.2009} 
%Bei der Microservice-Architektur ist dies anders.
%Die Software setzt sich aus einzelnen Komponenten zusammen. 
%Dabei entsprechen diese jeweils einem oder mehreren Microservices, die je von einem kleinen Entwicklerteam unabhängig voneinander entwickelt und bereitgestellt werden können.\autocite[Vgl.][]{MS-Fowler.25.03.2014} 
%Jeder Microservice erfüllt in der Regel dabei nur eine Aufgabe, wie beispielsweise das Bereitstellen einer \ac{REST}-Ressource.
