% !TEX root =  ../../../master.tex
\subsection{Microservice}
\label{sec:grundlagen:microservices}

Für die Umsetzung von verteilten Systemen bzw. von Cloud-basierten Anwendungen hat sich in den letzten Jahren die Microservice-Architektur immer weiter etabliert.\autocite[Vgl.][Kapitel \enquote{Monolithic architecture overview}]{MS-Sharma.2016}
Die Verwendung dieser Architektur ist jedoch nicht neu, sondern wird bereits seit über zehn Jahren vom Unternehmen Amazon genutzt.\autocite[Vgl.][]{MS-Wolff.02.11.2015} 
Die ursprüngliche Idee von Microservices wurde dabei aus dem Ansatz der \enquote{\textit{\ac{SOA}}} hergeleitet.\autocite[Vgl.][S. 1]{MS-Bucchiarone.2018} 
Dabei gibt es für den Begriff Microservice keine konkrete Definition.\autocite[Vgl.][S. 3]{MS-Wolff.2018}. 
Vielmehr wird die Microservice-Architektur als Gegenteil der monolithischen Architektur verstanden, bei der die Software einem einzigen lauffähigen Baustein gleicht.
Dabei umfasst dieser Baustein alle Funktionen der Software, arbeitet autark und wird meist von mehreren Entwicklungsteams gleichzeitig entwickelt.
Dementsprechend ist eine lose Kopplung von einzelnen Bestandteilen nicht möglich.
Bei der Microservice-Architektur hingegen besteht die gesamte Software aus unabhängigen Komponenten, die jeweils von kleinen Teams entwickelt und bereitgestellt werden.\autocite[Vgl.][]{MS-Fowler.25.03.2014} 
Jeder Microservice erfüllt in der Regel dabei nur eine Aufgabe, wie beispielsweise das Bereitstellen einer einzelnen \ac{REST}-Ressource.