% !TEX root =  ../theoretisch.tex
\subsection{Learning Analytics}

\subsubsection{Allgemein}
\enquote{Learning Analytics} beschreibt die datengestützte Evaluation von Lernfortschritten, dem zukünftigen Lernpotential oder eventuell auftretenden Problemen bei Lernprozessen.
Die Datengewinnung erfolgt dabei sowohl über eine direkte Leistungserfassung der Studierenden, wie etwa eine Seminararbeit oder einer Prüfung, als auch über vom Studierprozess gelöste Tätigkeiten, wie soziale Interaktionen oder Foreneinträge.
Als Ziel des Learning Analytics führen \citeauthor{rf-Johnson2012TheNH} die Echtzeitbetreuung Studierender an, welche auf die allgemeinen, aber auch auf die individuellen, Bedürfnisse und Fähigkeiten der Studierenden eingeht.
\autocite{rf-Johnson2012TheNH}

\subsubsection{Hintergrund}
Statt den herkömmlichen Methoden zur Unterstützung der Lernenden, dem identifizieren gefährdeter Studierender und der individuellen Unterstützung dieser, wird der Ansatz verallgemeinert.
Bei dieser Verallgemeinerung steht die anonyme Datenerhebung aller Studierenden im Fokus, um so vorbeugende allgemine Maßnahmen treffen zu können und die Notwendigkeit individueller Unterstützung zu minimieren.
So soll Learning Analytics bei korrekter Nutzung und Interpretation der Daten es den Lehrenden ermöglichen in der Lehre auf notwendige Aspekte einzugehen und den zusätzlichen Aufwand der persönlichen Betreuung zu reduzieren.
Zudem wird durch eine anonymisierte Problemerfassung die Fairness des Bewertungssystem aufrecht erhalten, da der Sympathieeinfluss bei Prüfungen aufgrund des geringeren individuellen Kontakts vermindert wird.


\subsubsection{Aspekte von Learning Analytics}
Die 7 Aspekte von Learning Analytics\autocite{EJ-LearningAnalytics} beschreiben die wichtigsten Abhängigkeiten, welche im Lernprozess betrachtet werden muss. 
Durch diese wesentlichen Hinweise kann das Lernerlebnis des Lernenden verbessert werden. 
Mit schnellen und pädagogisch richtigen Maßnahmen können die Lernergebnisse langfristig verbessert werden. 

\begin{itemize}
    \item \textbf{Learning Awareness}: den Lernprozess an sich verbessern und aufgrund der Stärken und Schwächen weitere Lernmaßnahmen effektiv festlegen
    \item \textbf{Privacy Awareness}: der Datenschutz steht im Fokus, aus den Lerndaten der Benutzer darf keine Bewertung möglich sein
    \item \textbf{Time Awareness}: Lernergebnisse stellen sich nach einem gewissen Zeitraum rein, voreilige Schlüsse zu ziehen, ist nicht sinnvoll
    \item \textbf{Visuelles Feedback}: die Lerninhalte sollen verständlich und übersichtlich aufgeführt werden um möglichst schnell das Wissen vermitteln zu können
    \item \textbf{Pädagogische Intervention}: durch geschultes pädagogisches Einwirken auf den Lernenden kann das Lernverhalten angepasst werden
    \item \textbf{Big Data}: trotz einer großen Menge der Daten zum analysieren des Lernenden müssen die Umstände und weitere Faktoren betrachtet werden
    \item \textbf{Einsichten und Struktur des Wissens}: durch bessere Einsichten in das Lernverhalten können sich Lernprozesse und Lernstrukuren abbilden
\end{itemize}


\subsubsection{Ursprung in der Sozialforschung}

Learning Analytics wird vielmals mit den Themengebieten wie der künstlichen Intelligenz, Machine Learning, statistischer Analyse und Business Inteligenz in Verbindung gebracht.
Die ursprünglichen Wurzeln des analytischen Ansatzes liegen jedoch in der Sozialforschung bzw. bei der Analyse von menschlichen Interaktionen und Bildungssystemen.\autocite[Vgl.][S. 1383]{ej-GeorgeSiemens}
Dabei werden im wesentlichen die folgenden Ziele verfolgt:\autocite[Vgl.][ab Minute 11]{ej-GeorgSiemensVideo}

\begin{itemize}
    \item Definition eines Lernenden, um dessen Bedürfnisse und Lerngewohnheiten zu verstehen
    \item Wissenstandsüberprüfung, um den Lernprozess nachvollziehen zu können
    \item Herausfinden, wie das Lernen mit Hilfe von Technologie effizienter und personalisierter gestaltet werden kann
    \item Vergleich des Wissensstandes des Studierenden mit den zu lernenden Inhalten, um den Studierenden in seinem Lernprozess zu unterstützen
\end{itemize}

% TODO: More in particular, the history of Learning Analytics is tightly linked to the development of four Social Sciences’ fields that have converged throughout time. 
% These fields pursued, and still do, four goals: 
% -Definition of Learner, in order to cover the need of defining and understanding a learner.
% -Knowledge trace, addressing how to trace or map the knowledge that occurs during the learning process.
% -Learning efficiency and personalization, which refers to how to make learning more efficient and personal by means of technology.
% -Learner – content comparison, in order to improve learning by comparing the learner’s level of knowledge with the actual content that needs to master


\subsubsection{Learning Analytics in Umfrageanwendungen}

Auf dem ersten Blick hat Learning Analytics nicht direkt einen Bezug zu einer Umfrageanwendung.
Jedoch kann bei richtiger Anwendung des Softwaresystems ein Nutzen für Lehrende sowie Studierende gewonnen werden, da so ebenfalls der Lernprozess anonym beobachtet und verbessert werden kann.
Der Fokus des Umfragetools ist darauf ausgerichtet, die Interaktion zwischen Studierenden und Dozierenden zu verbessern und es den Lehrenden erleichtern auch Probleme zu erkennen.
Das Tool ermöglicht somit Dozierenden, mit den richtigen Vorkenntnissen im Bereich Learning Analytics, einen gruppenbezogenen Lernprozess zu etablieren und diesen erfolgreich zu kontrollieren.
Durch das Testen des Lernstandes aller Studierenden eines Kurses können Lücken im Wissensstand oder vernachlässigte Teilbereiche anonymisiert entdeckt und effizient verbessert werden.
Eine daraus folgende allgemeine Auffrischungen des Lerninhaltes innerhalb einer globalen Lernumgebung, ermöglichen die persönliche Weiterentwicklung Studierender mit eventuellen Schwächen.
Dadruch verschiebt sich der Fokus von Learning Analytics von dem Individuum zu einer Lernendenmenge.
