% !TEX root =  ../theoretisch.tex
\subsection{Learning Analytics}

\subsubsection{Allgemein}
\enquote{Learning Analytics} beschreibt die datengestützte Evaluation von Lernfortschritten, dem zukünftigen Lernpotential oder eventuell auftretenden Problemen bei Lernprozessen.
Die Datengewinnung erfolgt dabei sowohl über eine direkte Leistungserfassung der Studierenden, wie etwa eine Seminararbeit oder einer Prüfung, als auch über vom Studierprozess gelöste Tätigkeiten, wie soziale Interaktionen oder Foreneinträge.
Als Ziel der Learning Analytics führen \citeauthor{rf-Johnson2012TheNH} die Echtzeitbetreuung Studierender an, welche auf die allgemeinen, aber auch auf die individuellen Bedürfnisse und Fähigkeiten der Studierenden eingeht.
\autocite{rf-Johnson2012TheNH}

\subsubsection{Hintergrund}
Statt der herkömmlichen Methoden zur Unterstützung der Lernenden, dem Identifizieren gefährdeter Studierender und der individuellen Unterstützung dieser, wird der Ansatz verallgemeinert.
Bei dieser Verallgemeinerung steht die anonyme Datenerhebung aller Studierenden im Fokus, um so vorbeugende allgemeine Maßnahmen treffen zu können und damit einhergehend die Notwendigkeit individueller Unterstützung zu minimieren.
So soll Learning Analytics den Lehrenden bei korrekter Nutzung und Interpretation der Daten ermöglichen, in der Lehre auf notwendige Aspekte einzugehen und den zusätzlichen Aufwand der persönlichen Betreuung zu reduzieren.
Die Fairness des Bewertungssystem wird durch eine anonymisierte Problemerfassung aufrecht erhalten, da der Sympathieeinfluss bei Prüfungen aufgrund des geringeren individuellen Kontakts vermindert wird.

\subsubsection{Aspekte von Learning Analytics}
Die sieben Aspekte von Learning Analytics\autocite[Vgl.][S. 46 ff.]{EJ-LearningAnalytics} beschreiben die wichtigsten Abhängigkeiten, welche im Lernprozess betrachtet werden müssen.
Durch diese wesentlichen Hinweise kann das Lernerlebnis des Lernenden verbessert werden.
Mit schnellen und pädagogisch richtigen Maßnahmen können die Lernergebnisse langfristig verbessert werden.
Nachfolgend werden diese Aspekte aufgelistet:
%
\begin{itemize}
    \item \textbf{Learning Awareness}: Der Lernprozess soll verbessert werden. Aufgrund der Stärken und Schwächen sollen weitere Lernmaßnahmen effektiv festgelegt werden können.
    \item \textbf{Privacy Awareness}: Der Datenschutz steht im Fokus. Es darf daher keine Bewertung aus den Lerndaten der Benutzer möglich sein.
    \item \textbf{Time Awareness}: Lernergebnisse stellen sich nach einem gewissen Zeitraum ein. Voreilige Schlüsse zu ziehen, ist nicht sinnvoll.
    \item \textbf{Visuelles Feedback}: Die Lerninhalte sollen verständlich und übersichtlich aufgeführt werden, um möglichst schnell das Wissen vermitteln zu können.
    \item \textbf{Pädagogische Intervention}: Durch geschultes pädagogisches Einwirken auf die Lernenden kann das Lernverhalten angepasst werden.
    \item \textbf{Big Data}: Trotz einer großen Menge der Daten zum Analysieren des Lernenden müssen die Umstände und weitere Faktoren betrachtet werden.
    \item \textbf{Einsichten und Struktur des Wissens}: Durch bessere Einsichten in das Lernverhalten können sich Lernprozesse und Lernstrukturen abbilden.
\end{itemize}
%
\subsubsection{Ursprung in der Sozialforschung}

Learning Analytics wird vielmals mit den Themengebieten der künstlichen Intelligenz, des Machine Learnings, der statistischen Analyse und der Business Intelligence in Verbindung gebracht.
Die ursprünglichen Wurzeln des analytischen Ansatzes liegen jedoch in der Sozialforschung bzw. bei der Analyse von menschlichen Interaktionen und Bildungssystemen.\autocite[Vgl.][S. 1383]{ej-GeorgeSiemens}
Dabei werden im Wesentlichen die folgenden Ziele verfolgt:\autocite[Vgl.][ab Minute 11]{ej-GeorgSiemensVideo}

\begin{itemize}
    \item Definition eines Lernenden, um dessen Bedürfnisse und Lerngewohnheiten zu verstehen
    \item Wissensstandsüberprüfung, um den Lernprozess nachvollziehen zu können
    \item Herausfinden, wie das Lernen mithilfe von Technologie effizienter und personalisierter gestaltet werden kann
    \item Vergleich des Wissensstandes des Studierenden mit den zu lernenden Inhalten, um den Studierenden in seinem Lernprozess zu unterstützen
\end{itemize}

\subsubsection{Learning Analytics in Umfrageanwendungen}
%TODO Quelle?
Auf den ersten Blick hat Learning Analytics nicht direkt einen Bezug zu einer Umfrageanwendung.
Jedoch kann bei richtiger Anwendung des Softwaresystems ein Nutzen für Lehrende sowie Studierende gewonnen werden, da so ebenfalls der Lernprozess anonym beobachtet und verbessert werden kann.
Der Fokus der Umfrageanwendung ist darauf ausgerichtet, die Interaktion zwischen Studierenden und Dozierenden zu verbessern und es den Lehrenden zu erleichtern, auch Probleme zu erkennen.
Die Anwendung ermöglicht somit Dozierenden, mit den richtigen Vorkenntnissen im Bereich Learning Analytics, einen gruppenbezogenen Lernprozess zu etablieren und diesen erfolgreich zu kontrollieren.
Durch das Testen des Lernstandes aller Studierenden eines Kurses können Lücken im Wissensstand oder vernachlässigte Teilbereiche anonymisiert entdeckt und effizient verbessert werden.
Daraus folgende allgemeine Auffrischungen des Lerninhaltes innerhalb einer globalen Lernumgebung ermöglichen die persönliche Weiterentwicklung Studierender mit eventuellen Schwächen.
Dadurch verschiebt sich der Fokus von Learning Analytics von dem Individuum zu einer Lernendenmenge.
