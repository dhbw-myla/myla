% !TEX root =  ../technisch.tex
\subsection{PostgreSQL}
\label{ssec:PostgreSQL}

\enquote{PostgreSQL}, kurz Postgres, ist ein unter der PostgreSQL Lizenz\autocite{rf-psqllicense} veröffentlichtes \ac{ORDBMS}.
PostgreSQL wird als ein Open-Source-Projekt betrieben.\footnote{\url{https://www.postgresql.org/}} 
Aktuell liegt das Projekt in der Version 12.3 vor.\footnote{\url{https://www.postgresql.org/docs/12/release-12-3.html}} 
Das \ac{ORDBMS} verwendet dabei die Datenbanksprache \ac{SQL} zur Benutzerausführung der \ac{CRUD}-Befehle.%bitte SQL quelle anfügen
Zudem besitzt Postgres ein umfangreiches Transaktionskonzept, welches auch besondere Verfahren, wie die \ac*{MVCC} zur effizienten Verarbeitung konkurrierender Zugriffe auf die Datenbank, beinhaltet.

\ac{DDL}, \ac{DML}, \ac{DCL}, \ac{TCL}

PostgreSQL ermöglich dem Benutzer nativ über die Kommandozeile via \emph{PSQL} zuzugreifen.
Möchte man jedoch ein geeignetes (graphisches) Interface haben, so kann man die Verwaltungsoberfläche via \emph{pgAdmin} öffnen. 



% https://www.ionos.de/digitalguide/server/knowhow/postgresql/
\begin{itemize}
    \item Open-Source-Projekt
    \item typisches Client Server Modell
    \item native Lösung zur Kommunikation PSQL
    \item graphische Oberfläche pgAdmin 
    \item kann in ein Docker eingebunden werden --> irgendwas sinnvolles 
\end{itemize}

% https://www.postgresqltutorial.com/what-is-postgresql/
\begin{itemize}
    \item Foreign-Keys
    \item Views
    \item NOT NULL
    \item UNIQUE
    \item CHECK
    \item PRIMARY KEY
    \item REFERENCES
\end{itemize}

%Hier was über die Lizenz schreiben