% !TEX root =  ../../../master.tex
\subsection{PostgreSQL}
\label{ssec:PostgreSQL}

\enquote{PostgreSQL}, kurz Postgres, ist ein unter der PostgreSQL-Lizenz\autocite{rf-psqllicense} veröffentlichtes \ac{ORDBMS}, was es zu einem Open-Source-Projekt macht.\footnote{\url{https://www.postgresql.org/}}
Aktuell liegt das Projekt in der Version 12.3 vor.\footnote{\url{https://www.postgresql.org/docs/12/release-12-3.html}}
Das \ac{ORDBMS} verwendet die Datenbanksprache \ac{SQL} zur Ausführung der vom Benutzer angegebenen \ac*{CRUD}-Befehlen.
Dabei setzt Postgres auf ein umfangreiches Transaktionskonzept, welches auch besondere Verfahren, wie die \ac*{MVCC} zur effizienten Verarbeitung konkurrierender Zugriffe auf die Datenbank, beinhaltet.
Unter anderem durch dieses Transaktionskonzept ist PostgreSQL vollständig \ac*{ACID}-konform.\autocite[Vgl.][]{rf-psqlfeatures}

Neben den allgemeinen Funktionalitäten, die durch den \ac{SQL}-Standard vorgegeben sind, verwendet PostgreSQL auch eigene spezifische Verbesserungen und Weiterentwicklungen des Standards.
Zudem bietet PostgreSQL ein breites Angebot an selbst- sowie aufgrund der Strutkur als Open-Source-Projekt fremderstellten Erweiterungen.
Zur Benutzung bietet PostgreSQL einerseits nativ ein Kommandozeileninterface mit dem Namen \emph{PSQL} sowie andererseits ein graphisches Interface zur Verwaltung des \ac{ORDBMS} mit dem Namen \emph{pgAdmin}.
Postgres wird in der Regel als Bestandteil des Servers im Client-Server-Modell eingesetzt (vgl. Kapitel \myRefGeneral{ssec:Architektur}).

Ein großer Vorteil von Postgres ist das Erstellen von sogenannten \emph{Views}.
Über diese können dem Benutzer besondere Sichten, insbesondere wenn diese mehrfach verwendet werden, zur Verfügung gestellt werden.
Zudem können auf diesen weitere Berechnungen durchgeführt werden.
Beim Einfügen von Tupeln können Validierungen durchgeführt werden, \zb ob ein Wert zwischen 0 und 5 liegt.
Über eine Restriktion \engl{constraint} können Attribute einzigartig (\emph{unique}) sein.
Dies hilft \ua dann, wenn sich Benutzer bei einer Anwendung registrieren und sich einen Benutzernamen auswählen sollen, der nicht doppelt belegt werden darf.

Die zu Beginn erwähnte Lizenz von Postgres ist der MIT-Lizenz und der BSD-Lizenz ähnlich und erlaubt daher nicht nur die Nutzung, sondern auch das Kopieren, Erweitern und Verteilen der Software oder der Dokumentation dieser für jegliche Zwecke.
Diese Verwendungszwecke unterliegen keinerlei Auflagen und erfordern ebenfalls keinerlei Abgaben von Gebühren, weshalb das \ac{ORDBMS} sich für die Nutzung in diesem Projekt sehr gut eignet.\autocite[Vgl.][]{rf-psqllicense}
