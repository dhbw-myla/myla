% !TEX root =  ../technisch.tex
\subsection{React}
\label{ssec:React}

Das Front-End beschreibt die Darstellung einer Webseite im Browser.
Die wohl bekannteste Sprache, um eine Webseite zu erstellen, ist \acs{HTML} in Verbindung mit \acs{CSS}.
\acs{HTML} beschreibt dabei den Aufbau der Webseite und in \acs{CSS} legt das Aussehen fest.
Um die Übersichtlichkeit im \acs{HTML}-Quelltext zu erhöhen, wird der \acs{CSS}-Quelltext in ein eigene Datei ausgelagert.

Eine heutige Front-End-Entwicklung findet häufig mit sogenannten Frameworks oder Bibliotheken statt.
Diese basieren in der Regel auf \acs{HTML} und \ac{JS}.
Der Unterschied zwischen einem Framework und einer Bibliothek lässt sich wie folgt erklären:
Ein Framework besteht aus mehreren Bibliotheken, wohingegen eine Bibliothek eine Sammlung von Klassen und Funktionen darstellt.\autocite[Vgl.][]{Libary_vs_Framework}
Der Informatiker Ralph Johnson beschreibt ein Framework folgendermaßen:
%
\begin{quote}
	\enquote{Das Framework stellt eine wiederverwendbare, gemeinsame Struktur für Anwendungen zur Verfügung. Entwickler binden das Framework in ihre eigenen Anwendungen ein und erweitern es so, dass es ihre bestimmten Anforderungen erfüllt.}
	\begin{center}{\textit{Ralph Johnson gemeinsam mit Brian Foote}}\end{center}
\end{quote}
%

Das \ac{DOM} bildet die Schnittstelle zwischen \acs{HTML} und \ac{JS}.
Alle \acs{HTML}-Elemente werden in Objekte umgewandelt.
D.~h., dass der Browser nie mit dem eigentlichen \acs{HTML}-Code arbeitet, sondern mit umgewandelten \ac{DOM}-Objekten.
Jeder \emph{Tag}, der in \ac{HTML} geschrieben wird, wird im \ac{DOM} als Objekt dargestellt.
Alle Objekte sind in einer Baumstruktur angeordnet.
Dies wird als Eltern-Kind-Prinzip bezeichnet.
Dies kann anhand der \acs{HTML}-Struktur in Quelltext~\vref{lst:ParentAndChild} erkannt werden.
Das Head- und Body-Objekt sind Kind-Objekte vom \acs{HTML}-Objekt.
Jedes Eltern-Objekt kann mehrere Kind-Objekte haben und diese können wiederum weitere Kind-Objekte besitzen.
\begin{lstlisting}[caption={Eltern-Kind-Prinzip},label={lst:ParentAndChild},language=HTML, showstringspaces={false}]
<!DOCTYPE html>
 <html> <!-- Eltern -->
   <head> <!-- Kind -->
    <title>Demo Eltern und Kind Prinzip</title> <!-- Kindes-Kind -->
   </head>
   <body> <!-- Kind -->
    code here
   </body>
 </html>
\end{lstlisting}

% https://books.google.de/books?id=2zrCDwAAQBAJ&printsec=frontcover&dq=react+hartmann&hl=de&sa=X&ved=0ahUKEwirn-u66uDpAhVExKYKHefEAPsQ6AEIKjAA#v=onepage&q=react%20hartmann&f=false 
React\footnote{\url{https://reactjs.org}} ist ein solches Framework.
Es ist eine Open-Source-JavaScript-Bibliothek mit deren Hilfe sogenannte \ac{SPA} erstellt werden.\autocite[Vgl.][]{hartmann2019react}
Das Hauptaugenmerk von React liegt auf dem komponentenbasierten Aufbau.
Eine Komponente stellt den Zustand der \ac{SPA} dar.
Sie repräsentiert einen fachlichen Teil wie \zb ein \enquote{einfaches} Eingabefeld \jinline|(Input)| oder auch komplexere Komponenten wie ein Formular.
Durch den komponentenbasierten Aufbau kann React nicht nur für die Entwicklung von Webanwendungen, sondern auch von \emph{iOS}- oder \emph{Android}-Apps verwendet werden.
Quelltext~\vref{lst:ReactKomponente} stellt eine Begrüßungs-Komponente dar, die einen Namen als Parameter \jinline|(props)| übergeben bekommt.

\begin{lstlisting}[caption={React-Komponente: Greeting},label={lst:ReactKomponente},language=HTML, showstringspaces={false}]
	import React from 'React'
	
	export const greeting = (name) => {
		return <h3>Hello {name}</h3>
	}
\end{lstlisting}

Um dieses Beispiel vereinfacht darzustellen, wird hierzu nachfolgend auf mehrere Studenten Bezug genommen.
Diese Studenten unterscheiden sich \zb in Vorname, Nachname, Alter und Matrikelnummer.
In React gibt es dementsprechend eine Komponente \jinline|Student|, welche die oben genannten Attribute als Parameter erhält.
Somit kann über einen allgemeinen Studenten ein spezifischer Studenten bekommen werden.

Dieses Vorgehen wird bei React als \emph{deklarative Komponente} bezeichnet und deshalb oft auch unter dem Begriff \texttt{UI as a Function} wiederzufinden.\autocite[Vgl.][]{hartmann2019react}
Wird eine Komponente in der \acs{SPA} geladen, wird das \acs{DOM} angepasst.
Hierbei werden nur die tatsächlichen Änderungen vorgenommen und nicht die komplette Seite geladen. \newline
Komponenten können einen internen Zustand haben \jinline|state| oder über externe Zustände \jinline|(properties)| verändert werden.
So kann \zb eine Komponente Account einen \jinline|state| besitzen.

Wie in Quelltext~\vref{lst:ReactKomponente} zu erkennen ist, hat die Komponente im Return-Statement einen Platzhalter \jinline|{ }|.
Diese ist ein typische React-Syntax.
In der Regel haben diese Komponenten die Dateiendung \acs{JSX}.
Sie wird bei der Verwendung von React empfohlen.
\acs{JSX} ist eine Erweiterung der JavaScript-\emph{Grammatik} und sorgt dafür, die interne Struktur zu ordnen und später in \acs{HTML} darzustellen.\autocite[Vgl.][]{WasIstJSX}

Ein Feature von React ist das Erstellen sogenannter \acp{PWA}.
\aclp{PWA} sind Webanwendungen, die allerdings vergleichbar mit nativen Apps, auf dem Smartphone \enquote{installiert} werden können.
Es lässt sich eine Verknüpfung auf dem \emph{Homescreen} erstellen, sodass die \acs{PWA} wie eine normale App geöffnet werden kann.
Dadurch erhält der Nutzer den Eindruck einer native Anwendung aus dem Appstore des Geräteherstellers.
Solch eine \ac{PWA} kann offlinefähig sein, sodass sie auch verwendet werden kann, wenn der Benutzer keine Internetverbindung hat. \autocite[Vgl.][]{hartmann2019react}
