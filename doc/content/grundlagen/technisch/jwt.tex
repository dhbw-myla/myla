% !TEX root =  ../technisch.tex
\subsection{\acl{JWT}}\label{chapter:jwt}

% PLAGIAT
Als Basis des \ac{JWT} wird die hierarchisch aufgebaute \ac{JSON} verwendet.
Diese Notation wird vorwiegend in der Webentwicklung genutzt, da Webbrowser \ac*{JS} nativ unterstützen und somit kein Parser benötigt wird.
Vergleichbare Formate, wie \ac*{XML} haben zudem eine größere Menge an Zusatzinformationen, weshalb \ac{JSON} für die Datenübertragung besser geeignet ist.

Der \ac{JWT} besteht, laut seinem \ac{RFC}, aus einem \ac{JSON}-Objekt, welches in einer \ac*{JWS}- und/oder \ac*{JWE}-Struktur encodiert wurde.
Der Inhalt des Objekts besteht aus Name-Wert-Paaren, bei denen der Name lediglich von einem String repräsentiert wird, während der Wert aus einem \ac{JSON}-Objekt, einem Array, einer Zeichenkette, einer Zahl oder einem booleschen Ausdruck bestehen kann.

Der \ac*{JOSE}-Header, welcher ebenfalls als \ac{JSON}-Objekt realisiert wird, beinhaltet Informationen über die kryptographischen Funktionen und Parameter, welche zum Erstellen des \ac{JWT} verwendet wurden.
So ist der \ac{JWT} je nach Auswahl als \acs*{JWS} dargestellt und enthält eine digitale Signatur oder ist als \acs*{JWE} dargestellt und enthält einen verschlüsselten Payload.
Auch eine Schachtelung der \acp{JWT} ist möglich, so können verschlüsselte \acp{JWT} als Payload in einem signierten \ac{JWT} übergeben werden.

Die verschiedenen Teile des \ac{JWT} werden Base64url encodiert, um \acp{JWT} in \acp{URL} verwenden zu können.
Dabei werden die drei Teile getrennt encodiert und zusammengeführt, indem sie mit jeweils einem Punkt voneinander getrennt werden.\autocite{rf-JSONORG}\autocite{rf-RFC7519}
