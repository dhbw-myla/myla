% !TEX root =  master.tex


% -------------------------------------------------------
% Dozent
% 
% -------------------------------------------------------
\newpage
\cvsect{Prof. Dr. Dennis Dutzint}
\begin{minipage}[t]{0.5\textwidth} 	\vspace{0.2\baselineskip} % Required for vertically aligning minipages
	\begin{entrylist}
		\entry
		{Name:}
		{Prof. Dr. Dennis Dutzint}
			\entry
		{Alter:}
		{45}
		\entry
		{Tätigkeit:}
		{Professor für Wirtschaftsinformatik}
	\end{entrylist}
	\begin{barchart}{5.0}\hspace{-1mm}
		\baritem{Skills}{4}
	\end{barchart}
\end{minipage}
\hfil
\begin{minipage}[t]{0.4\textwidth} 	\vspace{0.0\baselineskip} % Required for vertically aligning minipages
	\flushright
	\includegraphics[width=0.70\textwidth]{img/personas/prof_dutzin2}
\end{minipage}
\autocite{rf-unsplash-dozent}

Das möchte ich gerne haben:
\begin{itemize}
	\item einfaches, schnelles und unkompliziertes Erstellen neuer Umfragen
	\item Auswertung der Umfrageergebnisse
    \item Wiederverwendung vorhandener Umfragen
    \item Verwaltung von Nutzern
\end{itemize}

Mit der ersten Persona, Prof. Dr. Dennis Dutzint, wird ein technikerfahrener Professor einer Hochschule beschrieben.
Prof. Dr. Dennis Dutzint nimmt primär die Rolle des Umfrageerstellers ein, wobei er für verschiedene Kurse unterschiedliche Umfragen erstellen und auswerten möchte.
Weiterhin möchte er die Rolle eines Administrators übernehmen und die Anwendung betreuen.
