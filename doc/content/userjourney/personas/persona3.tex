% !TEX root =  master.tex

% -------------------------------------------------------
% umfrageteilnehmende Person
% -------------------------------------------------------
\newpage
\cvsect{Julian Weigert}
\begin{minipage}[t]{0.5\textwidth} 	\vspace{0.2\baselineskip} % Required for vertically aligning minipages
	\begin{entrylist}
		\entry
		{Name:}
		{Julian Weigert}
			\entry
		{Alter:}
		{24}
		\entry
		{Tätigkeit:}
		{Student für BWL}
	\end{entrylist}
	\begin{barchart}{5.0}\hspace{-1mm}
		\baritem{Skills}{2}
	\end{barchart}
\end{minipage}
\hfil
\begin{minipage}[t]{0.4\textwidth} 	\vspace{0.0\baselineskip} % Required for vertically aligning minipages
	\flushright
	\includegraphics[width=0.70\textwidth]{img/personas/julian}
\end{minipage}
\autocite{rf-unsplash-student}

Das möchte ich gerne haben:
\begin{itemize}
	\item Umfragen anonym durchführen
    \item Umfragen auf seinem Smartphone durchführen
    \item Umfragen auch später durchführen
\end{itemize}

Mit der dritten Persona, Julian Weigert, wird ein technisch weniger begabter Student einer Hochschule beschrieben.
Julian Weigert nimmt primär die Rolle des Umfrageteilnehmers ein, wobei er die Umfragen gerne an seinem mobilen Endgeräten bearbeiten möchte.
Durch seinen eng getakteten Zeitplan möchte er Umfragen auch Tage nach der Herausgabe des Links beantworten können.
