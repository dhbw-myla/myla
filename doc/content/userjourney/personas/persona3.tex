% !TEX root =  master.tex
\newcommand{\weigert}{Julian Weigert\xspace} %sich zu arbeiten.
% -------------------------------------------------------
% umfrageteilnehmende Person
%
% -------------------------------------------------------
\newpage
\cvsect{\weigert}
\begin{minipage}[t]{0.5\textwidth}
	\vspace{-3.6cm}
	\renewcommand{\arraystretch}{1.5}
	\begin{tabular}{l l}
		Name: & \weigert \\
		Alter: & 24 \\
		Tätigkeit: & Student der BWL \\
		Skills: & 2/5 \hspace{-1cm} \begin{barchart}{5.0}
			\baritemNL{}{2}
		\end{barchart} \\
	\end{tabular}
\end{minipage}
\hfill
\begin{minipage}[t]{0.4\textwidth}
	\flushright
	\includegraphics[width=0.70\textwidth]{img/personas/julian}
\end{minipage}
\autocite{rf-unsplash-student}

Das möchte ich gerne haben:
\begin{itemize}
	\item Umfragen anonym beantworten
    \item Teilnahme an Umfragen über das Smartphone
    \item Zeitliche Unabhängigkeit beim Durchführen der Umfragen
\end{itemize}

Mit der dritten Persona, \weigert, wird ein technisch weniger begabter Student einer Hochschule beschrieben.
\weigert nimmt primär die Rolle eines Umfrageteilnehmers ein, wobei er die Umfragen gerne auf seinem mobilen Endgerät beantworten möchte.
Durch seinen eng getakteten Zeitplan möchte er Umfragen auch Tage nach der Veröffentlichung beantworten können.
