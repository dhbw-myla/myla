% !TEX root =  master.tex
\newcommand{\ariane}{Ariane Aris\xspace}
% -------------------------------------------------------
% Studierende Person
%
% --------------------------------------------------------
\newpage
\cvsect{\ariane}
\begin{minipage}[t]{0.5\textwidth}
	\vspace{-3.6cm}
	\renewcommand{\arraystretch}{1.5}
	\begin{tabular}{l l}
		Name: & \ariane \\
		Alter: & 20 \\
		Tätigkeit: & Studentin der Wirtschaftsinformatik \\
		Skills: & 3/5 \hspace{-1cm} \begin{barchart}{5.0}
			\baritemNL{}{3}
		\end{barchart} \\
	\end{tabular}
\end{minipage}
\hfill
\begin{minipage}[t]{0.4\textwidth}
	\flushright
	\includegraphics[width=0.70\textwidth]{img/personas/ariane}
\end{minipage}
\autocite{rf-unsplash-studentin}

Das möchte ich gerne haben:
\begin{itemize}
	\item einfaches, schnelles und unkompliziertes Erstellen neuer Umfragen für ihre wissenschaftlichen Arbeiten
	\item Auswertung der Umfrageergebnisse
    \item Teilnahme an anderen Umfragen
\end{itemize}

Mit der zweiten Persona, \ariane, wird eine Studentin für Wirtschaftsinformatik einer Hochschule beschrieben.
\ariane nimmt primär die Rolle einer Umfrageerstellerin und sekundär die Rolle einer Teilnehmerin ein.
Dabei möchte sie sowohl einfach eine Umfrage erstellen, als auch bei anderen Umfragen teilnehmen können.
