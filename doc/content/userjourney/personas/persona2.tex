% !TEX root =  master.tex

% -------------------------------------------------------
% Studierende Person
%
% ------------------------------------------------------- Ariane Aris
\newpage
\cvsect{Ariane Aris}
\begin{minipage}[t]{0.5\textwidth} 	\vspace{0.2\baselineskip} % Required for vertically aligning minipages
	\begin{entrylist}
		\entry
		{Name:}
		{Ariane Aris}
			\entry
		{Alter:}
		{20}
		\entry
		{Tätigkeit:}
		{Studentin für Wirtschaftsinformatik}
	\end{entrylist}
	\begin{barchart}{5.0}\hspace{-1mm}
		\baritem{Skills}{3}
	\end{barchart}
\end{minipage}
\hfil
\begin{minipage}[t]{0.4\textwidth} 	\vspace{0.0\baselineskip} % Required for vertically aligning minipages
	\flushright
	\includegraphics[width=0.70\textwidth]{img/personas/ariane}
\end{minipage}
\autocite{rf-unsplash-studentin}

Das möchte ich gerne haben:
\begin{itemize}
	\item einfaches, schnelles und unkompliziertes Erstellen neuer Umfragen für ihre wissenschaftlichen Arbeiten
	\item Auswertung der Umfrageergebnisse
    \item Teilnahme an anderen Umfragen
\end{itemize}

Mit der zweiten Persona, Ariane Aris, wird ein Studentin für Wirtschaftsinformatik einer Hochschule beschrieben.
Ariane Aris nimmt primär die Rolle eines Umfrageerstellers und sekundär die Rolle einer Teilnehmerin ein.
Dabei möchte sie sowohl einfach eine Umfrage erstellen können, als auch bei anderen Umfragen teilnehmen.