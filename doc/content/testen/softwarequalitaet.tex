% !TEX root =  master.tex
\section{Softwarequalität}
\authorsection{\authorEJ}

Die Softwarequalität eines Programmes ist durch verschiedene Merkmale definiert.
Zu diesen gehören Funktionalität, Portabilität, Zuverlässigkeit, Benutzbarkeit, Effizienz und Wartbarkeit.\footnote{\url{https://entwickler.de/online/agile/softwarequalitaet-so-misst-und-verbessert-man-software-114867.html}}
Ebenfalls geht es dabei um die Erfüllung der Erwartungen eines Benutzers an das Programm.
In dem Fall, der für das Fach Fallstudie gestellten Anforderungen, handelt es sich um ein Kinoreservierungssystem, das über Front-End und Back-End steuerbar sein soll.
Diese Qualitätsmerkmale sind wichtig einzuhalten, da sie die Zufriedenheit bei Kunden sowie Programmierern erhöhen, die Zuverlässigkeit der Software steigern, einen reibungsloseren Betriebsablauf gewähren, Kundenwünsche zuverlässiger erfüllt werden können, die Anforderungen besser und angepasster nach den Wünschen des Kunden erfüllt werden können und die Arbeitsprozesse beim Programmieren der Software effektiver gestaltet werden können. % TODO: this sentence is hard to understand
Im Allgemeinen ist Testen also nicht nur für den Kunden der Software wichtig, sondern auch für die Programmierer.
Es bietet beiden Parteien Sicherheit, um mit der Software zu arbeiten und diese zu verwalten.
Dabei ist jedoch zu beachten, dass die Eliminierung von Fehlern mittels Testen immer aufwendiger wird.
Je mehr Code bereits durch Tests abgedeckt wurde, desto teurer wird die Erhöhung der Testabdeckung.
Der zu testende Code wird schwieriger zu testen und die Testerfolge sind nur noch kleinschrittig.
Am Ende übersteigen die Kosten für das Testen einer Codezeile deren Nutzen.
Um für dieses Problem eine Lösung zu finden gibt es das Qualitätsmanagement, es ist dafür da um ein Optimum aus Fehlerkosten und Fehlerverhütungskosten zu finden.
Diese lassen sich zwar nicht genau bestimmen, aber können trotzdem durch Erfahrungswerte, Vergleichswerte und abschätzende Planung mithilfe von Soll- und Istwerten etwas eingegrenzt werden.\footnote{\url{http://www.enzyklopaedie-der-wirtschaftsinformatik.de/lexikon/is-management/Systementwicklung/Management-der-Systementwicklung/Software-Qualitatsmanagement}}
Deshalb ist in den meisten Projekten keine Testabdeckung von 100\% gefordert, sondern es wird sich zwischen Kunde und Programmierer auf einen niedrigeren Prozentsatz geeinigt.
Für das Kinoreservierungsprogramm wird eine Codeabdeckung von $\sim$60\% gefordert und in Kapitel \vref{sec:codeabdeckung} nochmal näher beschrieben.
