% !TEX root =  ../../master.tex
\section{Abgrenzung zu Alternativen}
Im folgenden Abschnitt soll die Anwendung von verschiedenen Alternativen abgegrenzt werden und hervorgehoben werden, wodurch MyLA besonders hervorgehoben werden kann.

Im Vergleich zur Lernplattform Moodle, welche weit verbreitet ist und sehr viele verschiedene Features aufweist, wie beispielsweise dem Anlegen von Kursen, individuellen Dashboards und Kalendern für Nutzer und Kursräume, einen System zum Hochladen und Herunterladen von Dateien, Mitteilungen, dem Tracking von Aktivitäten und einem Texteditor zum Schreiben und Versenden von Texten, bietet die Anwendung MyLA mit den auf Umfragen konzentrierten und spezifizierten Funktionen einen deutlich kleineren Umfang; beide Anwendungen können jedoch geräteunabhängig verwendet werden\autocite[Vgl.][]{jr-moodle-features}.
Genau diese Spezialisierung jedoch unterscheidet MyLA von Moodle.

Mittels Moodle wird es zudem nicht ermöglicht, Umfragen durchzuführen, welche erstellt und von den Dozenten geteilt werden können, genau dies ist eines der Alleinstellungsmerkmale von MyLA.

Es gibt viele verschiedene Umfragetools, bei denen man ebenfalls Umfragen erstellen und erneut verwenden kann, jedoch kann dort jeder Umfragen erstellen und diese teilen.
Mittels einem Registrierungsschlüssel, welcher von einem Administrator gesetzt wurde, kann sichergestellt werden, dass nur autorisierte Personen einen Account erstellen können, dies hat zur Folge, dass Dozenten dieses Tool gezielt dafür nutzen können, um für ihre Kurse und Vorlesungen Umfragen zu erstellen.
Dadurch lässt sich MyLA von sämtlichen online verfügbaren Umfragetools ebenfalls abgrenzen. 

Zusammenfassend kann formuliert werden, dass MyLA vor allem entwickelt wurde, um Dozenten zu ermöglichen, gezielt Umfragen zu erstellen und diese den Studenten verfügbar zu machen, ohne das diese für die Umfrage einen Account oder ähnliches benötigen.
