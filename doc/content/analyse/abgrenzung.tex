% !TEX root =  ../../master.tex
\section{Abgrenzung zu Alternativen}
\label{sec:AbgrenzungZuAlternativen}

Im folgenden Abschnitt soll die Anwendung von verschiedenen Alternativen abgegrenzt werden.
Dabei soll hervorgehoben werden, worin sich das zu entwickelnde Umfrage-Tool von diesen Alternativen unterscheidet.

Zunächst soll Bezug auf die bekannte Lernplattform Moodle genommen werden.
Moodle bietet die Möglichkeit Kurse, individuelle Dashboards und Kalender anzulegen.\autocite[Vgl.][]{ms-moodle-features}
Moodle bietet zunächst offiziell eine Möglichkeit Umfragen zu erstellen, jedoch ist dies durch die verwendeten Einstellungen in der Moodle-Version der \acs{DHBW} Mannheim nicht möglich.\footnote{Laut Aussagen der Betreuerin dieser Arbeit}
Außerdem können diese Umfragen in der Regel nicht an externe, Nicht-Moodle-Nutzer weitergegeben werden.
Genau diese Spezialisierung soll die Software dieses Projektes von Moodle unterscheiden bzw. hervorheben.
Dabei ist besonders entscheidend, dass die Software die wiederholte Nutzung von Umfragen ermöglichen soll.

Neben Moodle gibt es eine Vielzahl von Umfrage-Tools, die zur Erstellung von internen und externen Umfragen genutzt werden können.
Bei diesen ist es jedoch oftmals nicht möglich Vorlagen für Umfragen zu erstellen, sodass Umfragen mehrmals gestellt werden können.
Weiterhin ist die \acs{DHBW} Mannheim an Auflagen gebunden, wie sie Daten erheben darf.
Beispielsweise muss bei einer Datenerhebung darauf geachtet werden, dass der Datenschutz eingehalten und die Anonymität der Teilnehmer gewahrt wird.
Dementsprechend ist eine umfangreiche vorherige Prüfung eines Umfragetools oftmals nicht vermeidbar, um eventuelle Tracking-Funktionen oder ähnliches zu entdecken.
Außerdem hat die \acs{DHBW} Mannheim keinen Einfluss auf potenzielle Änderungen solcher Tools und befindet sich hier in einer entsprechenden Abhängigkeit, die ein Risiko darstellt.
Weiterhin sind professionelle Umfrage-Tools, die gegebenenfalls alle nötigen Voraussetzungen mit sich bringen, entsprechend kostenintensiv.

Aus diesen Gründen ist es für die \acs{DHBW} Mannheim von Nutzen, ein eigenes Umfrage-Tool zu besitzen.
Dieses Tool kann entsprechend an die Bedürfnisse der \acs{DHBW} angepasst werden.
