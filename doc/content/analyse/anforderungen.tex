% !TEX root =  ../../master.tex
\section{Anforderungen}
\label{sec:Anforderungen}

Die Anforderungen an das zu entwerfende und entwickelnde System lassen sich in mehrere Bereiche aufteilen.

Die Anwendung soll unabhängig vom Gerät des Endnutzers funktionieren.
So soll ein Student mit seinem mobilen Endgerät eine Umfrage ausfüllen können, während etwa ein Dozent die Umfrage am Desktop-Computer erstellt.
Daraus ergibt sich Anforderung [A1], die endgeräteunabhängige Nutzung der Anwendung.

Die Registrierung neuer Nutzer soll geschützt werden, um eine zu hohe Auslastung des Systems bzw. Angriffe zu verhindern.
Zudem sollen bereits registrierte und explizit dazu berechtigte Nutzer neue Konten anlegen können.
Somit ergibt sich Anforderungen [A2] und [A3], die Verwendung eines Registrierungsschlüssels bei der eigenen Registrierung sowie die Möglichkeit Nutzer durch bestehende Nutzer anzulegen.

Manuell angelegte Nutzer sollen bei ihrer ersten Anmeldung ein neues Passwort setzen müssen, um eine erhöhte Sicherheit zu gewährleisten.
Ebenfalls sollten Nutzer, deren Passwörter zurückgesetzt wurden, diese erneut setzen müssen.
Dazu ist eine Passwortänderungsseite erforderlich, welche wiederum als Passwortänderung für normale Nutzer verwendet werden kann.
Dies stellen die Anforderungen [A4], die Möglichkeit zur Passwortänderung, [A5], die Pflicht der Passwortänderung bei manueller Registrierung, sowie [A6], die Pflicht zur Passwortänderung bei manuell zurückgesetzten Passwörtern, dar.

Angemeldete Nutzer sollen in der Lage sein, sowohl neue Umfragen zu erstellen als auch bereits selbst erstellte Umfragen zu verwalten.
Die Umfragen sollen dabei mehrere Fragetypen beinhalten können, welche sich aus offenen Fragen, Single-Choice-Fragen, Multiple-Choice-Fragen sowie skalierten Fragen zusammensetzen.
Daraus ergeben sich die Anforderungen [A7], der Erstellung von Umfragen, [A8], der Verwaltung dieser, sowie [A9] der Verwendung verschiedener Fragetypen.

Erstellte Umfragen sollen dabei nicht automatisiert veröffentlicht werden, sondern nur auf Anfrage des Nutzers.
Dadurch soll es Dozenten möglich sein, Umfragen in Ruhe im Voraus vorzubereiten.
Diese unveröffentlichten Umfragen sollen zudem bearbeitbar sein, sodass vorher erkannte Fehler ausgebessert werden können.
Nach der Veröffentlichung sollen Umfragen wiederholt werden können.
Die Wiederholung soll dabei für mehrere Zeiträume, mehrere Kurse sowie mehrere Vorlesungen erfolgen können.
Die Ergebnisse der angelegten Umfragen sollen grafisch einsehbar sein.
Dies ergibt die Anforderungen [A10], die manuelle Veröffentlichung der Umfrage, [A11], die Bearbeitbarkeit unveröffentlichter Umfragen, [A12], die Wiederholbarkeit der Umfragen, und [A13], der grafischen Darstellung der Umfrageergebnisse.

Die Teilnahme an Umfragen soll anonym erfolgen und für die Teilnehmer so simpel wie möglich sein.
Dies stellt die letzten funktionalen Anforderungen [A14], die anonyme Teilnahme an Umfragen, sowie [A15], der Möglichkeit zur einfachen Teilnahme, dar.

Da die Anwendung im Rahmen eines Projektes der \acs{DHBW} Mannheim konzeptioniert wird, ist das Farbschema dieser an das Corporate-Design der \acs{DHBW} Mannheim anzupassen.
Dies ist die einzige nicht-funktionale Anforderung [A16], Farbschema im Corporate-Design.

In Tabelle~\vref{tab:Anforderungen} sind alle Anforderungen noch einmal übersichtlich aufgelistet.
Dabei erfolgt eine Kennzeichnung, ob es sich um eine funktionale (f) oder nicht-funktionale (nf) Anforderung handelt.
Während funktionale Anforderungen definieren, \emph{was} die Software umsetzen soll, geben nicht-funktionale Anforderungen die Leistungs- und Qualitätsanforderungen sowie Rahmenbedingungen wieder.\autocite[Vgl.][S. 10]{nl-robertson2012mastering}\autocite[Vgl.][S. 3 ff]{nl-braun2016nicht}

\begin{table}
  \setlength\extrarowheight{3pt}
\centering
  \begin{tabular}{|C{1cm}|C{1cm}|L{12cm}|}
    \hline
    \textbf{Nr.} & \textbf{Art} & \textbf{Beschreibung} \\
    \hline
    {\label{Anf:A1}A1} & f & Endgeräteunabhängige Nutzung \\
    \hline
    {\label{Anf:A2}A2} & f & Registrieren mit Registrierungsschlüssel \\
    \hline
    {\label{Anf:A3}A3} & f & Manuelles Registrieren neuer Nutzer \\
    \hline
    {\label{Anf:A4}A4} & f & Möglichkeit zur Passwortänderung \\
    \hline
    {\label{Anf:A5}A5} & f & Pflicht zur Passwortänderung bei manueller Registrierung \\
    \hline
    {\label{Anf:A6}A6} & f & Pflicht zur Passwortänderung bei zurückgesetztem Passwort \\
    \hline
    {\label{Anf:A7}A7} & f & Erstellen von Umfragen \\
    \hline
    {\label{Anf:A8}A8} & f & Verwalten von Umfragen \\
    \hline
    {\label{Anf:A9}A9} & f & Verwendung verschiedener Fragetypen \\
    \hline
    {\label{Anf:A10}A10} & f & Manuelle Veröffentlichung der Umfragen \\
    \hline
    {\label{Anf:A11}A11} & f & Bearbeitbarkeit unveröffentlichter Umfragen \\
    \hline
    {\label{Anf:A12}A12} & f & Wiederholbarkeit von Umfragen \\
    \hline
    {\label{Anf:A13}A13} & f & Grafische Darstellung der Umfrageergebnisse \\
    \hline
    {\label{Anf:A14}A14} & f & Anonyme Teilnahme an Umfragen \\
    \hline
    {\label{Anf:A15}A15} & f & Möglichkeit zur einfachen Teilnahme an Umfragen \\
    \hline
    {\label{Anf:A16}A16} & nf & Farbschema im Corporate-Design \\
    \hline
  \end{tabular}
  \caption{Übersicht der Anforderungen}
  \label{tab:Anforderungen}
\end{table}
