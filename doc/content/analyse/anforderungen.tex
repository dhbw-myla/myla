% !TEX root =  ../../master.tex
\section{Anforderungen}

Eine Vielzahl von Anforderungen bestimmte das Vorgehen im Projekt.

Ein wesentlicher Grundsatz war, dass diese App eine Lernapp darstellt, welche für Umfragen genutzt wird.
Um diese Applikation auch einsetzen zu können, galten bestimmte Vorgaben.

Zum einen sollten neue Accounts über einen Anmeldeschlüssel erstellt werden können.

Eine weitere Anforderung war, dass Dozenten nach dem ersten Login ein neues Passwort für ihren Account festlegen sollten.
Zusätzlich ist es für sie möglich, Umfragen zu erstellen und zu gruppieren sowie Umfrageergebnisse einzusehen.
Bei der Erstellung der Umfragen soll es ermöglicht werden, dass diese über mehrere Zeiträume und Vorlesungen wiederverwendet werden können.
Dabei sollen diese Umfragen einige Fragearten, namentlich offene Fragen, Fragen mit festgelegten Antwortmöglichkeiten, also Multiple-Choice- beziehungsweise Single-Choice-Fragen, sowie Fragen mit einer Skala oder Bewertungen, beinhalten können.
Die erstellte Umfrage kann nachfolgend verfügbar gemacht werden.

Teilnehmer der Umfrage sollen anonym an den Umfragen partizipieren können. Dabei kann via Link die entsprechende Umfrageseite aufgerufen werden. Am Ende einer Umfrage sollte den Teilnehmern die Möglichkeit gegeben werden, anonym darunter kommentieren zu können.

Nach einer Umfrage soll es einem Dozenten ermöglicht werden, die Ergebnisse der Umfrage direkt einzusehen, wobei es ihm dabei ermöglicht wird, die Ergebnisse im zeitlichen Verlauf zu betrachten, um den Lernfortschritt der Studenten überblicken zu können.
Außerdem soll er verschiedene Ansichten nutzen können und es soll ihm ermöglicht werden, die Ergebnisse grafisch anschaulich in einem Diagramm zu betrachten.


% VORSCHLAG - Tabularx ist irgndwie broke im repo hier - bst durch andere anpassungen in der config.. sollte man nochmal drüber schauen! falls fragen sind, einfach mich fragn - Rene

\begin{table}[!htbp]
    \centering
    \begin{tabularx}{\textwidth}{l|k|r}
      \toprule
      {Bezeichnung} & {Art} & {Beschreibung} \\
      \midrule
      {\label{Anf:A1}A1} & fachlich & Lernapp entwickeln, welche für Umfragen genutzt wird \\
      \hline
      {\label{Anf:A2}A2} & technisch & Erstellen neuer Accounts über einen Anmeldeschlüssel \\
      \hline
      {\label{Anf:A3}A3} & fachlich und technisch & Dozenten sollen nach dem ersten Login ein neues Passwort für ihren Account festlegen \\
      \hline
      {\label{Anf:A4}A4} & fachlich und technisch & Dozenten sollen Umfragen erstellen können\\
      \hline
      {\label{Anf:A5}A5} & fachlich und technisch & Dozenten sollen Umfragen gruppieren können\\
      \hline
      {\label{Anf:A6}A6} & fachlich und technisch & Dozenten sollen Umfrageergebnisse einsehen können\\
      \hline
      {\label{Anf:A6a}A6a} & fachlich und technisch & Dozenten sollen Umfrageergebnisse im zeitlichen Verlauf miteinander vergleichen können\\
      \hline
      {\label{Anf:A6b}A6b} & fachlich und technisch & Dozenten sollen Umfrageergebnisse in verschiedenen Ansichten betrachten können\\
      \hline
      {\label{Anf:A6c}A6c} & fachlich und technisch & Dozenten sollen Umfrageergebnisse grafisch, in Form von Diagrammen, einsehen können\\
      \hline
      {\label{Anf:A7a}A7a} & technisch & Umfragen sollen über mehrere Zeiträume verfügbar gemacht werden können \\
      \hline
      {\label{Anf:A7b}A7b} & technisch & Umfragen sollen für mehrere Vorlesungen verfügbar gemacht werden können \\
      \hline
      {\label{Anf:A7c}A7c} & technisch & Umfragen sollen für mehrere Kurse verfügbar gemacht werden können \\
      \hline
      {\label{Anf:A8a}A8a} & technisch & offene Fragen sollen in Umfragen verwendet werden können \\
      \hline
      {\label{Anf:A8b}A8b} & technisch & Single-Choice-Fragen sollen in Umfragen verwendet werden können \\
      \hline
      {\label{Anf:A8c}A8c} & technisch & Multiple-Choice-Fragen sollen in Umfragen verwendet werden können \\
      \hline
      {\label{Anf:A8d}A8d} & technisch & Fragen mit Bewertung sollen in Umfragen verwendet werden können \\
      \hline
      {\label{Anf:A8e}A8e} & technisch & Fragen mit Skala sollen in Umfragen verwendet werden können \\
      \hline
      {\label{Anf:A9}A9} & fachlich und technisch & erstellte Umfragen können veröffentlicht werden \\
      \hline
      {\label{Anf:A10}A10} & fachlich und technisch & Teilnehmer sollen anonym an den Umfragen partizipieren können \\
      \hline
      {\label{Anf:A11}A11} & fachlich und technisch & Die Umfrageseite soll via Link aufgerufen werden können \\
      \hline
      {\label{Anf:A12}A12} & fachlich und technisch & Teilnehmer sollen am Ende einer Umfrage anonym kommentieren können \\
      \hline
      \bottomrule
    %   ... weiter für weitere Anforderungen
    \end{tabularx}
    \caption{Anforderungen}
    \label{tab:Anforderungen}
\end{table}