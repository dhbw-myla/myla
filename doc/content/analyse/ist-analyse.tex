% !TEX root =  ../../master.tex
\section{Ist-Analyse}
\label{sec:IstAnalyse}

Zum Zeitpunkt dieser Arbeit besitzt die \acs{DHBW} kein eigenes Tool, um Online-Umfragen jedem Dozierenden oder Studierenden der Hochschule zugänglich zu machen.
Jedoch wurde im Rahmen einer ähnlichen, zuvor durchgeführten Seminararbeit bereits versucht, eine ähnliche Anwendung zu entwickeln.
Bei dieser lag der Fokus auf der Erhebung von Daten im Verlauf einer Vorlesung, was ebenfalls über Umfragen realisiert werden sollte.
Nach einer näheren Analyse dieser bereits existierenden Anwendung konnte festgestellt werden, dass diese nicht betriebsfähig ist.

Das Vorgängermodells wurde als Webanwendung prototypisch realisiert und basiert auf dem bekannten Framework \emph{Angular 6+}.
Zudem wurde zur Entwicklung die von Google bereitgestellte Entwicklungs-Plattform \emph{Firebase} verwendet.
Bereits vorgefertigte native Funktionalitäten wie ein Chat-Tool wurden dabei eingebunden.

Das Vorgängermodell hat dabei äußerlich den Anschein einer funktionsfähigen Software erweckt, jedoch sind viele Bestandteile nur zu Demonstrationszwecken entwickelt worden.
Von diesen Bestandteile wurden viele Funktionalitäten bzw. Ergebnisse dieser fest einprogrammiert, unabhängig von den Eingaben des Benutzers.
Dies wird zum Beispiel daran deutlich, dass ein Student zwar Bewertungen vornehmen kann und diese selbst sieht, allerdings keine Persistenz stattfindet, sodass der Dozent diese nicht einsehen kann.
Der Dozent sieht lediglich eine fest eingestellte Auswertung.
Somit kann der angestrebte Nutzen, die Verwendung der Daten zur Unterstützung der Learning Analytics, nicht gewährleistet werden.
Das Ziel der Arbeit wurde dementsprechend verfehlt.

Eine Weiterentwicklung der alten Software ist ebenfalls nicht ratsam, da neben der kommerziellen Lösung Firebase auch weitere Pakete verwendet wurden, deren Benutzungsrechte unklar sind.
Eine rechtliche Absicherung und somit die Nutzungslizenzen aller verwendeten Module ist deshalb nicht vorhanden.
Somit ist die Weiterentwicklung der nicht funktionsfähigen Software zunächst mit der Aufschlüsselung aller Module und deren Lizenzen sowie Verhandlungen über etwaige Nutzungsrechte verbunden.
Der hierfür nötige Aufwand überschreitet den einer funktionsfähigen Neuentwicklung bei Weitem.

Dies wird auch durch die Dokumentation der Software verdeutlicht, da diese praktisch nicht vorhanden ist.
Zwar wurde eine Seminararbeit verfasst, eine detaillierte oder zumindest grundlegende Beschreibung auf technischer Ebene liegt allerdings nicht vor.
