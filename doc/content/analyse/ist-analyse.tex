% !TEX root =  ../../master.tex
\section{Ist-Analyse}
\label{sec:IstAnalyse}

Zum Zeitpunkt dieser Arbeit besitzt die \acs{DHBW} kein eigenes Tool, um Online-Umfragen jedem Dozierenden oder Studierenden der Hochschule zugänglich zu machen.
Jedoch wurde im Rahmen einer ähnlichen zuvor durchgeführten Seminararbeit 

eine solche Anwendung zu entwickeln.

Bevor die Anforderungen an die Anwendung formuliert sind, wurde die vorhergehende Umsetzung einer Learning-Analytics-Software analysiert.

Hierbei konnte zum einen ermittelt werden, dass die Seite via Google Firebase umgesetzt und zur Verfügung gestellt wurde.
Dabei ist jedoch festzuhalten, dass der vorhandene Code einen sehr niedrigen Umfang hatte, sowie unsauber war.
Dadurch wurde eine Weiterentwicklung oder eine Umsetzung, die auf dem vorangegangenen Projekt aufbaut, deutlich erschwert.

Hinzu kommt, dass die Visualisierung sehr generisch war und keine individuellen Anpassungen an die Gegebenheiten aufwies. Wodurch diese ebenfalls nicht weiter verwendet werden konnte.

Die erstellten und aufrufbaren Seiten boten keinerlei Funktionalitäten, da sämtliche Eingaben nicht persistent gespeichert, sondern bei dem nächsten Start der Anwendung überschrieben wurden.
Dies hat zur Folge, dass kein Bestandteil der vorherrschenden Anwendung für die Umsetzung der neuen Applikation verwendet werden konnte.
Somit mussten für diese Anwendung vollständig neue Anforderungen formuliert, sowie sämtliche Entwicklungsschritte wie beispielsweise die Konzeption und Verwirklichung vollständig neu begonnen werden.
