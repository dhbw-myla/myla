% !TEX root =  master.tex
\section{Sequenzdiagramm}
\label{sec:sequenzdiagram}
\authorsection{\authorSG}

Im Anhang \vref{fig:Anhang_seq_reservieren} befindet sich ein Sequenzdiagramm des Reservierungsvorganges.

Im Folgenden werden die einzelnen Schritte kurz erläutert.

Nachdem der Benutzer die gewünschten Sitzplätze ausgewählt hat, drückt er im Front-End auf den Button \enquote{Reservieren}.
Im Front-End wird daraufhin ein \acs{JSON} mit den Informationen der gewünschten Sitzplätze erstellt (siehe \vref{lst:json_book}).
Vor dem \acs{JSON} wird noch ein Formparameter book angeschlossen, um eine genauere Identifizierung in der \acs{REST}-Schnittstelle \jinline|reservation/book| zu realisieren.

Das Front-End ruft über einen \acs{AJAX}-Call den Webserver mit der \acs{URI}  \url{http://localhost:8080/cinemasystem-system/reservation/book} auf und übergibt dabei das \acs{JSON}.
Diese Ressource ruft ihren eigenen Service auf.
Dieser ruf den Webserver für die Datenschicht über die \acs{URI} \url{http://localhost:8080/cinemasystem-data/reservation/book} auf und übergibt das \acs{JSON}.
Hier wird das \acs{JSON} mittels eines \acs{JSON}-Converters in ein \acs{DTO} konvertiert.
Da aus dem Front-End lediglich die Vorstellungs-ID übermittelt wird, müssen die Informationen der Vorstellung aus der Datenbank nachgeladen werden. \\
Hier wird die Methode \jinline|find(String id)| im \jinline|ShowService|, der ebenfalls in der Ressource implementiert ist, aufgerufen.
Diese greift auf die Datenbank zu und holt die Informationen der Vorstellung ab und wandelt es anschließend in ein Show-\acs{DTO} um.

%Anschließend kann eine Verifizierung erfolgen, ob überhaupt eine Reservierung derzeit möglich ist.
%Ein möglicher Ausschlussgrund könnte sein, dass zwischen Beginn der ausgewählten Vorstellung und aktueller Zeit weniger als 30 Minuten liegen. \\
%Ist diese erfolgreich gewesen, wird überprüft, ob es bereits den Kunden in der Datenbank gibt.
%Dies wird durch die vom Front-End übermittelte E-Mail-Adresse des aktuellen Benutzers durchgeführt.
%Somit kann später der Reservierung und somit den Tickets eindeutig der Kunde zugewiesen werden. \\

Anschließend erfolgen verschiedene Verifizierungen, ob das Anlegen der Reservierung möglich ist.
Ausschlussgründe könnten etwa folgende sein:
\begin{enumerate}
	\item Zwischen Vorstellung und aktueller Uhrzeit liegen weniger als 30 Minuten.
	\item Der ausgewählte Sitzplatz ist schon von einem anderen Benutzer blockiert.
	\item Der ausgewählte Sitzplatz ist bereits vergeben.
\end{enumerate}
War die Reservierung erfolgreich, wird das Ergebnis an den \jinline|ReservationService| des Webserver System im \acs{JSON} zurück übermittelt. % TODO: Webserver? Webservice?
Ggf. kann es hier noch zu weiteren Berechnungen kommen, weshalb das \acs{JSON} in ein Reservation\acs{DTO} umgewandelt wird.

Ist die Methode \jinline|postTickets| abgeschlossen, wird das Ergebnis in ein \acs{JSON} umgewandelt und an das Front-End gesendet, welches dem Benutzer die Reservierung als \acs{QR-Code} darstellt (siehe Kapitel \vref{fig:vorstellung04}).

Eine nähere Beschreibung der Herausforderungen für das Reservieren und den Lösungsansätzen befindet sich in Kapitel \vref{ssec:herausforderung_reservieren} und Kapitel \vref{ssec:loesung_reservieren}.
