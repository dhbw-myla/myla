% !TEX root =  ../../master.tex
\section{Client}
\label{sec:ClientKonzept}
\authorsection{\authorSG}

Wie bereits in Kap. \vref{ssec:React} beschrieben, wird in diesem Projekt React verwendet. 
Für die Unterteilung der verschiedenen Komponenten soll auf eine spezielle \acfp{UX} und \acfp{UI} wert gelegt werden. 
Da die Applikation im Rahmen eines Projektes der DHBW Mannheim konzeptioniert wird, sollen die Komponenten an dieses Farbschema angelehnt werden. 
Im folgenden soll auf den Aufbau der Benutzeroberfläche eingegangen werden. 

% !TeX root = ../../../master.tex

\subsection{Participate}
Im Bezug auf Verwendbarkeit der Anwendung soll die Startseite die \emph{Participate}-Seite sein.
Also die Seite, an dem ein Teilnehmer an einer Umfrage teilnehmen kann (siehe \vref{fig:ParticipateImplement}). \newline
In dieser soll der Benutzer bzw. Student der an einer Umfrage teilnimmt, einen \emph{Surveycode} wie \zb \texttt{W3VFG5NHY} eingeben, um an der Umfrage zu partizipieren.

Anschließend wird der Benutzer auf die Umfrageseite weitergeleitet, auf der er die benötigten Felder ausfüllt (siehe Kap. \vref{ssec:UmfrageImplement}).

\begin{figure}[h]
	\centering
	\includegraphics[width=0.95\textwidth, keepaspectratio]{img/client/Participate.png}
	\captionsetup{justification=centering, format=plain}
	\caption[\acl{UI}: Teilnahme Umfrage]{\acl{UI}: Teilnahme Umfrage \\ \quelleScreenshot}
	\label{fig:ParticipateImplement}
\end{figure}


\subsection{Umfrage}
\label{ssec:konzept:client:umfrage}
Wie in Abb. \vref{fig:MockUmfrage} dargestellt, soll der Teilnehmer hier auf den zuvor erstellten Frageboge geleitet werden. 
Hier kann dieser diesen ausfüllen und nach Beendigung über einen Button abschicken. 
Der Teilnehmer soll ein Feedback erhalten, ob seine Teilnahme erfolgreich war. 

\begin{figure}[h]
	\centering
	\includegraphics[width=0.7\textwidth]{img/konzeption/client/umfrage_teilnehmer}
	\captionsetup{justification=centering, format=plain}
	\caption[Mock-Up der Teilnahmeseite]{Mock-Up der Teilnahmeseite \\\figma}
	\label{fig:MockUmfrageTeilnehmer}
\end{figure}

% !TeX root = ../../../master.tex

\subsection{Signup}
\label{ssec:Signup}

Um an einer Umfrage partizipieren zu können, muss sich ein Benutzer zuvor ein Nutzerkonto erstellen.  
Hierfür wählt der Benutzer wie in \abb \ref{fig:SignupImplement} dargestellt das Formfeld mit seinem Benutzernamen wie \zb \emph{\texttt{Sascha}}, den Registerkey, der vom Administrator der Software festgelegt ist. 
Dieser könnte \emph{\texttt{DemoKey}} sein.
Anschließend wählt der Benutzer ein sicheres Passwort, welches er nochmals darunter eingibt.  
Darauf hin startet er den Registrierungsprozess durch das Drücken des Buttons \jinline|Sign Up|.
Ist das gewählte Passwort \emph{konkludent}, so soll der Benutzer auf die \emph{Result-Seite} weitergeleitet, da diese im späteren Verlauf \ua das Kernstück darstellt (siehe Kap. \vref{ssec:ResultDashboardImplement}). 
Stimmt das Passwort nicht überein, so erhält der Benutzer ein visuelles Feedback mit der Aufforderung, die Passwortwahl erneut zu treffen. 

\begin{figure}[hp]
	\centering
	\includegraphics[width=0.95\textwidth, keepaspectratio]{img/client/Signup.png}
	\captionsetup{justification=centering, format=plain}
	\caption[\acf{UI}: Registrierung]{\acf{UI}: Registrierung \\ \quelleScreenshot}
	\label{fig:SignupImplement}
\end{figure}

% !TeX root = ../../../master.tex

\subsection{Result-Dashboard}
\label{ssec:ResultDashboardImplement}

Wie bereits der User-Journey aus Kapitel \myRefGeneral{sec:UserJourney} dargestellt wurde, ist das Result-Dashboard eine wichtige Komponente dieser Anwendung. 
Deshalb wird nach dem \emph{Login} der Benutzer auf diese Seite geleitet. 

Wie der Name schon sagt, sollen hier die Resultate der zuvor erstellten Umfragen einsehbar sein. 
Um eine Ansprechende \acsu*{UI} zu generieren, soll hier auf ein \emph{Card-Design} verwendet werden. 
Über eine Button auf der Karte ist es möglich auf ein detaillierte Auswertung zu gelangen. \newline
\abb \vref{fig:SurveyResultDashboardImplement} zeigt drei erstelle Umfragen des Benutzers: 
% 
\begin{itemize}
	\item Kurzes Beispiel - WWI19SEC
	\item Projektmanagement - WWI19SEC
	\item Projektmanagement - WWI17SEC
\end{itemize}
% 
Jede Umfrage besitzt einen individuellen einzigartigen \emph{Sourveycode} wie z.B. \emph{\texttt{QSDQO6EP0T}}. 
Dieser lässt sich über das Icon \faClipboard\xspace den \emph{Sourveycode} in die \emph{Zwischenablage} kopieren. 
Ferner hat der Benutzer eine Übersicht, wie viele Teilnehmer \engl{Participations} bzgl. diese Umfrage getätigt haben. 
Über den Button \jinline|Show Survey Results| kann der Benutzer die Ergebnisse dieser Umfrage ansehen. 

\begin{figure}[!htb]
	\centering
	\includegraphics[width=0.95\textwidth, keepaspectratio]{img/client/SurveyResultDashboard.png}
	\captionsetup{justification=centering, format=plain}
	\caption[\acf{UI}: Registrierung]{\acf{UI}: Registrierung \\ \quelleScreenshot}
	\label{fig:SurveyResultDashboardImplement}
\end{figure}

\subsubsection*{Detaillierte Auswertung}
Hat der Benutzer die detaillierte Auswertung ausgwählt, soll je nach Frageart ein bestimmtes Diagramm erstellen werden. 
Die Diagramme werden mit Hilfe von \emph{react-chartjs-2} generiert.\footnote{\url{https://www.npmjs.com/package/react-chartjs-2}} 

\abb \myRefGeneral{fig:SurveyResultDetailImplement} zeigt einen Ausschnitt der Auswertung der Umfrage zur Projektmanagement-Vorlesung. 
Hier wird exemplarisch eine \emph{Tortendiagramm} ausgegeben. 
Die verwendeten Farben wurden zuvor definiert. 
Der Benutzer hat die Möglichkeit über das \emph{Tortendiagramm} mit seiner Maus zu fahren (hovern).
Hier erhält er den jeweiligen Wert über einen \emph{Tooltip} angezeigt.

\begin{figure}[!htb]
	\centering
	\includegraphics[width=0.95\textwidth, keepaspectratio]{img/client/SurveyResultDetail2.png}
	\captionsetup{justification=centering, format=plain}
	\caption[\acf{UI}: Auswertung der Umfrage]{\acf{UI}: Auswertung der Umfrage aus Abb. \vref{fig:SurveyResultDashboardImplement} \\ \quelleScreenshot}
	\label{fig:SurveyResultDetailImplement}
\end{figure}


\subsection{Umfrage erstellen}
\label{ssec:konzept:client:umfrage_erstellen}

Um Umfragen auswerten zu können, müssen diese zunächst erstellt werden.
Dazu soll die Anwendung die Möglichkeit bieten Fragen verschiedener Arten, wie sie durch Anforderung~\hyperref[Anf:A9]{A9} definiert wurden, in Umfragen hinzuzufügen.
Mock-Ups, welche die Erstellung der einzelnen Fragekategorien zeigen, sind in den Abbildungen \ref{fig:MockUmfrageSingleChoice}, \ref{fig:MockUmfrageOffeneFragen} sowie in den Abbildungen im Anhang \ref{fig:MockUmfrageMultipleChoice}, \ref{fig:MockUmfrageRating} zu sehen.
Die Art der Frage kann in den Mock-Ups über ein Dropdown-Feld bestimmt werden.
Anschließend kann die Fragestellung in ein dafür vorgesehenes Feld eingetragen werden, während mögliche Antworten, welche bei Single- oder Multiple-Choice-Fragen benötigt werden, hinzugefügt werden können.
Fragen, welche Intervall- oder Ratingskalen verwenden, können nach belieben angepasst werden.
Dies wird beispielhaft in Abbildung~\ref{fig:MockUmfrageRating} durch die \emph{von} und \emph{bis}-Felder repräsentiert.

\begin{figure}[H]
	\centering
	\includegraphics[width=0.7\textwidth]{img/konzeption/client/umfrage_erstellen_single_choice}
	\captionsetup{justification=centering, format=plain}
	\caption[Mock-Up der Umfrageerstellung von Single-Choice-Fragen]{Mock-Up der Umfrageerstellung von Single-Choice-Fragen\\\figma}
	\label{fig:MockUmfrageSingleChoice}
\end{figure}

\begin{figure}[H]
	\centering
	\includegraphics[width=0.7\textwidth]{img/konzeption/client/umfrage_erstellen_offene_frage}
	\captionsetup{justification=centering, format=plain}
	\caption[Mock-Up der Umfrageerstellung von offenen Fragen]{Mock-Up der Umfrageerstellung von offenen Fragen\\\figma}
	\label{fig:MockUmfrageOffeneFragen}
\end{figure}


\subsection{Unterteilung der Gliederungsansichten}

\begin{itemize}
	\item Beschreibung der Mockups 
	 \begin{itemize}
		 \item Was haben wir uns bei den einzelnen Seiten gedacht
		 \item Wie sollten die Seiten grundsätzlich aussehen. (Brainstorming)
		 \item evtl. bissel auf Designthinking eingehen
		 \item Welches tool haben wir dazu genutzt
	 \end{itemize}
	 \item Recherche zu vorhandenen Umfragetools --> (https://www.polly.ai/slack-poll, https://strawpoll.de/, https://www.limesurvey.org/de/, https://www.surveymonkey.de/, https://pingo.coactum.de/)
	 --> Ideensammlung und Marktrecherche
	 \item Grundlegende Idee der Einfachheit (evtl Gesamtkonzept)
	 \item Man könnte auf den User Journey eingehen
	 \item Anlehnung an DHBW farben sind vorgesehen
	 \item Orientierung an modernen Websites mit header und footer
	 \item Hinblick auf mobile usage
\end{itemize}

\subsection{Bestimmung von Darstellungsformen}