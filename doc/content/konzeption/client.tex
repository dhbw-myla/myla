% !TEX root =  ../../master.tex
\section{Client}
\authorsection{\authorSG}

Wie bereits in Kap. \vref{ssec:React} beschrieben, wird in diesem Projekt React verwendet. 
Für die Unterteilung der verschiedenen Komponenten soll auf eine spezielle \acfp{UX} und \acfp{UI} gelegt werden. 
Im folgenden soll auf den Aufbau der Benutzeroberfläche eingegangen werden. 

\subsection{Participate}
Im Bezug auf Verwendbarkeit wie in Kap. \vref{sec:UserJourney} beschrieben ist die Startseite die \emph{Participate}-Seite. 
In dieser soll der Benutzer bzw. Student der an einer Umfrage teilnimmt, einen \emph{Surveycode} wie z.B. \emph{\texttt{OYZQGGXOF9}} eingeben, um an der Umfrage zu partizipieren. 

Anschließend wird der Benutzer auf die Umfrageseite weitergeleitet, auf der er die benötigten Felder ausfüllt (siehe \vref{ssec:Umfrage}).

\subsection{Umfrage}
\label{ssec:konzept:client:umfrage}
Wie in Abb. \vref{fig:MockUmfrage} dargestellt, soll der Teilnehmer hier auf den zuvor erstellten Frageboge geleitet werden. 
Hier kann dieser diesen ausfüllen und nach Beendigung über einen Button abschicken. 
Der Teilnehmer soll ein Feedback erhalten, ob seine Teilnahme erfolgreich war. 

\begin{figure}[h]
	\centering
	\includegraphics[width=0.7\textwidth]{img/konzeption/client/umfrage_teilnehmer}
	\captionsetup{justification=centering, format=plain}
	\caption[Mock-Up der Teilnahmeseite]{Mock-Up der Teilnahmeseite \\\figma}
	\label{fig:MockUmfrageTeilnehmer}
\end{figure}

\subsection{Result-Dashboard}
\label{ssec:ResultDashboard}





\subsection{Unterteilung der Gliederungsansichten}

\begin{itemize}
	\item Beschreibung der Mockups 
	 \begin{itemize}
		 \item Was haben wir uns bei den einzelnen Seiten gedacht
		 \item Wie sollten die Seiten grundsätzlich aussehen. (Brainstorming)
		 \item evtl. bissel auf Designthinking eingehen
		 \item Welches tool haben wir dazu genutzt
	 \end{itemize}
	 \item Recherche zu vorhandenen Umfragetools --> (https://www.polly.ai/slack-poll, https://strawpoll.de/, https://www.limesurvey.org/de/, https://www.surveymonkey.de/, https://pingo.coactum.de/)
	 --> Ideensammlung und Marktrecherche
	 \item Grundlegende Idee der Einfachheit (evtl Gesamtkonzept)
	 \item Man könnte auf den User Journey eingehen
	 \item Anlehnung an DHBW farben sind vorgesehen
	 \item Orientierung an modernen Websites mit header und footer
	 \item Hinblick auf mobile usage
\end{itemize}
\subsection{Bestimmung von Darstellungsformen}