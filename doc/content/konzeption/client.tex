% !TEX root =  ../../master.tex
\section{Client}
\label{sec:ClientKonzept}

Im Rahmen des Designs der Webseite hat sich eine vereinfachte Marktrecherche angeboten. 
Auch wenn im Rahmen der gewünschten Anwendung bereits eine Marktrecherche nicht direkt notwendig ist, da das Tool für eine abgegrenzte Anwendergruppe geplant ist, wird diese trotzdessen durchgeführt. 
Dadurch wird sich erhofft, dass das Risko für Fehlentscheidungen im Rahmen der Produktentwicklungen reduziert werden kann.\autocite{Pioch2019}
Die Zielgruppe des Tools sind Studenten und Dozenten, welche eine Umfrage durchführen möchten. 
Hierbei ist eine dedizierte Definition durch die Anforderungen und den Verwendungsrahmen geschehen. 

Um einen Einblick in mögliche Implementierungen zu gewinnen, sind unterschiedliche Umfragetools evaluiert worden. 
Damit das Beste aus den unterschiedlichen Anwendungen herausgenommen wird, werden diese miteinander verglichen.
So hat die Einfachheit der Anwendung von \emph{Polly} überzeugt.
Die Auswertungsmöglichkeiten und Darstellungen mit der Verbindung zu der direkten Umfrage von \emph{strawpoll}. 
Sowie der Verwendung eines Teiles der kompletten Breite, um den Fokus auf die wichtigen Inhalte der Webseite zu legen. 
Der Aufbau der Seitenelemente von \emph{limesurvey} in Verbindung mit der Möglichkeit des Long-Scrolling.
Sowie Schlussendlich dem Design des Umfragetools \emph{Pingo}.  

In den nachfolgenden Kapiteln soll auf den grundlegende Aufbau bzw. die grundlegende Idee der Benutzeroberfläche beim Start des Projektes eingegangen werden. 
Die gezeigten Mockups erheben dabei keinen Anspruch vollständig zu sein, da diese lediglich zur groben Orientierung und Ausrichtung benötigt werden.
Für die Unterteilung der verschiedenen Komponenten soll auf eine spezielle \acfp{UX} und \acfp{UI} wert gelegt werden. 
Da die Applikation im Rahmen eines Projektes der DHBW Mannheim konzeptioniert wird, sollen die Komponenten an dieses Farbschema angelehnt werden. 
Das verwendete Tool zum Erstellen von Mockups ist Figma.
Figma\footnote{https://www.figma.com/} bietet eine Webanwendung an, in welcher einzelne Elemente per \emph{Drag\&Drop} zu einer Arbeitsfläche hinzugefügt werden können.
Die Auswahl der Elemente umfasst dabei Schrift, Formen und Farben, sowie weitere nützliche Eigenschaften, welche für den Anwendungsfall jedoch nicht relevant sind.
Durch die große Auswahl an Strukturierungsmöglichkeiten, sind alle wichtigen Aspekte eines Mockup-Tools vorhanden.

% !TEX root =  ../../master.tex
\subsection{Grundgerüst}

Um auf den einzelnen Unterseiten der Anwendung einen Anhaltspunkt zu besitzen, ist die Abbildung \ref{fig:MockGrundgeruest} da. 
In dieser wird eine grundlegende Struktur festgelegt, welche auf den Unterseiten der Webanwendung verwendet werden soll.
Dabei wird die Seite in unterschiedliche Sektoren untergliedert.
Die einzelnen Komponenten einer Seite sind dabei der \emph{Header}, \emph{Inhalt der Seite}, sowie der \emph{Footer}. 
Wie schon die Namen der Sektoren indizieren, ist die Anordnung von \emph{Header} über den \emph{Inhalt der Seite} zu dem abschließenden \emph{Footer}, strukturiert.
Jedoch besitzen diese Komponenten nicht die maximale Breite.
Daher gibt es an den rechten und linken Rand der Seite jeweils ungenutzte Fläche. 
Diese freie Fläche ist jedoch gewollt, dadurch soll der Benutzer nicht mit zu viel Inhalt auf einmal konfrontiert werden.
Somit kann seine Aufmerksamkeit auf den \emph{Inhalt der Seite} gelenkt werden.
Als Vorteil besitzt ein konsistenter Aufbau, dass sich der Benutzer an eine Struktur gewöhnen kann. 

\begin{figure}[h]
	\centering
	\includegraphics[width=0.7\textwidth]{img/konzeption/client/grundgeruest}
	\captionsetup{justification=centering, format=plain}
	\caption[Mock-Up vom Grundgerüst der Anwendung]{Mock-Up vom Grundgerüst der Anwendung \\\figma}
	\label{fig:MockGrundgeruest}
\end{figure}


\subsection{Teilnahme und Login}

Beim ersten Laden der Anwendung soll eine Startseite aufgebaut werden, wie sie in Abbildung~\vref{fig:MockSignin} zu sehen ist.
Hierbei sollen Nutzer einfach an einer Umfrage teilnehmen und sich anmelden oder ein Registrierungsvorgang starten können.
Für die Teilnahme an einer Umfrage müssen die Studierenden lediglich einen \emph{Surveycode} wie \zb \texttt{OYZQGGXOF9} eingeben.
Eine Anmeldung ist nicht erforderlich, um die Anonymität gemäß Anforderung~\hyperref[Anf:A14]{A14} zu wahren.
Anschließend wird der Benutzer auf die Umfrageseite weitergeleitet, auf der er die benötigten Felder ausfüllt (siehe Kapitel~\ref{ssec:konzept:client:umfrage}).
Beim Beantworten von Umfragen ist davon auszugehen, dass dies zu einem großen Teil von mobilen Endgeräten erfolgt (siehe Anforderung~\hyperref[Anf:A1]{A1}).
Diese Seiten müssen daher für die mobile Nutzung optimiert werden, dies ist jedoch nicht als Teil der Mock-Ups dargestellt.
Für angemeldete Benutzer, die Umfragen erstellen und verwalten, ist jedoch die mobile Nutzung unwahrscheinlich, da etwa Dozenten in der Regel die Vorlesungsvorbereitung an einem Desktop-Computer durchführen.
% Hierbei soll der späteren Implementierung dieser Seite der Aspekt der mobilen Benutzung beachtet werden, da viele Benutzer eine Umfrage auf ihren mobilen Endgeräten durchführen.
% Bei allen anderen Seiten soll dieser Aspekt jedoch zunächst vernachlässigt werden.

\begin{figure}[H]
	\centering
	\includegraphics[width=0.7\textwidth]{img/konzeption/client/signin}
	\captionsetup{justification=centering, format=plain}
	\caption[Mock-Up der Startseite]{Mock-Up der Startseite \\\figma}
	\label{fig:MockSignin}
\end{figure}

Nutzer erhalten, wie in Abbildung~\ref{fig:MockSignin} auf der rechten Seiten sichtbar, ebenfalls die bereits angesprochene Anmelde-Möglichkeit.
Hierzu soll der \emph{Benutzername} und das \emph{Passwort} eines existierenden Benutzers eingegeben werden.
Nach einer erfolgreichen Anmeldung soll der Benutzer auf die Dashboard-Seite weitergeleitet werden (siehe Kapitel~\ref{ssec:konzept:client:dashboard}).
Bei der ersten Nutzung der Anwendung besteht die Möglichkeit über den Knopf \emph{Sign Up} ein Registrierungsvorgang gestartet werden, was zunächst eine Weiterleitung auf die Registrierungsseite (siehe Kapitel~\ref{ssec:konzept:client:dashboard}) vorsieht.
Dadurch wird Anforderung~\hyperref[Anf:A10]{A10}, die manuelle Veröffentlichung einer Umfrage, sowie Anforderung~\hyperref[Anf:A15]{A15}, die Möglichkeit zur einfachen Teilnahme an Umfragen, erfüllt.


% !TeX root = ../../../master.tex

\subsection{Signup}
\label{ssec:Signup}

Um an einer Umfrage partizipieren zu können, muss sich ein Benutzer zuvor ein Nutzerkonto erstellen.  
Hierfür wählt der Benutzer wie in \abb \ref{fig:SignupImplement} dargestellt das Formfeld mit seinem Benutzernamen wie \zb \emph{\texttt{Sascha}}, den Registerkey, der vom Administrator der Software festgelegt ist. 
Dieser könnte \emph{\texttt{DemoKey}} sein.
Anschließend wählt der Benutzer ein sicheres Passwort, welches er nochmals darunter eingibt.  
Darauf hin startet er den Registrierungsprozess durch das Drücken des Buttons \jinline|Sign Up|.
Ist das gewählte Passwort \emph{konkludent}, so soll der Benutzer auf die \emph{Result-Seite} weitergeleitet, da diese im späteren Verlauf \ua das Kernstück darstellt (siehe Kap. \vref{ssec:ResultDashboardImplement}). 
Stimmt das Passwort nicht überein, so erhält der Benutzer ein visuelles Feedback mit der Aufforderung, die Passwortwahl erneut zu treffen. 

\begin{figure}[hp]
	\centering
	\includegraphics[width=0.95\textwidth, keepaspectratio]{img/client/Signup.png}
	\captionsetup{justification=centering, format=plain}
	\caption[\acf{UI}: Registrierung]{\acf{UI}: Registrierung \\ \quelleScreenshot}
	\label{fig:SignupImplement}
\end{figure}

\subsection{Umfrage}
\label{ssec:konzept:client:umfrage}
Wie in Abb. \vref{fig:MockUmfrage} dargestellt, soll der Teilnehmer hier auf den zuvor erstellten Frageboge geleitet werden. 
Hier kann dieser diesen ausfüllen und nach Beendigung über einen Button abschicken. 
Der Teilnehmer soll ein Feedback erhalten, ob seine Teilnahme erfolgreich war. 

\begin{figure}[h]
	\centering
	\includegraphics[width=0.7\textwidth]{img/konzeption/client/umfrage_teilnehmer}
	\captionsetup{justification=centering, format=plain}
	\caption[Mock-Up der Teilnahmeseite]{Mock-Up der Teilnahmeseite \\\figma}
	\label{fig:MockUmfrageTeilnehmer}
\end{figure}

% !TeX root = ../../../master.tex

\subsection{Result-Dashboard}
\label{ssec:ResultDashboardImplement}

Wie bereits der User-Journey aus Kapitel \myRefGeneral{sec:UserJourney} dargestellt wurde, ist das Result-Dashboard eine wichtige Komponente dieser Anwendung. 
Deshalb wird nach dem \emph{Login} der Benutzer auf diese Seite geleitet. 

Wie der Name schon sagt, sollen hier die Resultate der zuvor erstellten Umfragen einsehbar sein. 
Um eine Ansprechende \acsu*{UI} zu generieren, soll hier auf ein \emph{Card-Design} verwendet werden. 
Über eine Button auf der Karte ist es möglich auf ein detaillierte Auswertung zu gelangen. \newline
\abb \vref{fig:SurveyResultDashboardImplement} zeigt drei erstelle Umfragen des Benutzers: 
% 
\begin{itemize}
	\item Kurzes Beispiel - WWI19SEC
	\item Projektmanagement - WWI19SEC
	\item Projektmanagement - WWI17SEC
\end{itemize}
% 
Jede Umfrage besitzt einen individuellen einzigartigen \emph{Sourveycode} wie z.B. \emph{\texttt{QSDQO6EP0T}}. 
Dieser lässt sich über das Icon \faClipboard\xspace den \emph{Sourveycode} in die \emph{Zwischenablage} kopieren. 
Ferner hat der Benutzer eine Übersicht, wie viele Teilnehmer \engl{Participations} bzgl. diese Umfrage getätigt haben. 
Über den Button \jinline|Show Survey Results| kann der Benutzer die Ergebnisse dieser Umfrage ansehen. 

\begin{figure}[!htb]
	\centering
	\includegraphics[width=0.95\textwidth, keepaspectratio]{img/client/SurveyResultDashboard.png}
	\captionsetup{justification=centering, format=plain}
	\caption[\acf{UI}: Registrierung]{\acf{UI}: Registrierung \\ \quelleScreenshot}
	\label{fig:SurveyResultDashboardImplement}
\end{figure}

\subsubsection*{Detaillierte Auswertung}
Hat der Benutzer die detaillierte Auswertung ausgwählt, soll je nach Frageart ein bestimmtes Diagramm erstellen werden. 
Die Diagramme werden mit Hilfe von \emph{react-chartjs-2} generiert.\footnote{\url{https://www.npmjs.com/package/react-chartjs-2}} 

\abb \myRefGeneral{fig:SurveyResultDetailImplement} zeigt einen Ausschnitt der Auswertung der Umfrage zur Projektmanagement-Vorlesung. 
Hier wird exemplarisch eine \emph{Tortendiagramm} ausgegeben. 
Die verwendeten Farben wurden zuvor definiert. 
Der Benutzer hat die Möglichkeit über das \emph{Tortendiagramm} mit seiner Maus zu fahren (hovern).
Hier erhält er den jeweiligen Wert über einen \emph{Tooltip} angezeigt.

\begin{figure}[!htb]
	\centering
	\includegraphics[width=0.95\textwidth, keepaspectratio]{img/client/SurveyResultDetail2.png}
	\captionsetup{justification=centering, format=plain}
	\caption[\acf{UI}: Auswertung der Umfrage]{\acf{UI}: Auswertung der Umfrage aus Abb. \vref{fig:SurveyResultDashboardImplement} \\ \quelleScreenshot}
	\label{fig:SurveyResultDetailImplement}
\end{figure}


\subsection{Umfrage erstellen}
\label{ssec:konzept:client:umfrage_erstellen}

Um Umfragen auswerten zu können, müssen diese zunächst erstellt werden.
Dazu soll die Anwendung die Möglichkeit bieten Fragen verschiedener Arten, wie sie durch Anforderung~\hyperref[Anf:A9]{A9} definiert wurden, in Umfragen hinzuzufügen.
Mock-Ups, welche die Erstellung der einzelnen Fragekategorien zeigen, sind in den Abbildungen \ref{fig:MockUmfrageSingleChoice}, \ref{fig:MockUmfrageOffeneFragen} sowie in den Abbildungen im Anhang \ref{fig:MockUmfrageMultipleChoice}, \ref{fig:MockUmfrageRating} zu sehen.
Die Art der Frage kann in den Mock-Ups über ein Dropdown-Feld bestimmt werden.
Anschließend kann die Fragestellung in ein dafür vorgesehenes Feld eingetragen werden, während mögliche Antworten, welche bei Single- oder Multiple-Choice-Fragen benötigt werden, hinzugefügt werden können.
Fragen, welche Intervall- oder Ratingskalen verwenden, können nach belieben angepasst werden.
Dies wird beispielhaft in Abbildung~\ref{fig:MockUmfrageRating} durch die \emph{von} und \emph{bis}-Felder repräsentiert.

\begin{figure}[H]
	\centering
	\includegraphics[width=0.7\textwidth]{img/konzeption/client/umfrage_erstellen_single_choice}
	\captionsetup{justification=centering, format=plain}
	\caption[Mock-Up der Umfrageerstellung von Single-Choice-Fragen]{Mock-Up der Umfrageerstellung von Single-Choice-Fragen\\\figma}
	\label{fig:MockUmfrageSingleChoice}
\end{figure}

\begin{figure}[H]
	\centering
	\includegraphics[width=0.7\textwidth]{img/konzeption/client/umfrage_erstellen_offene_frage}
	\captionsetup{justification=centering, format=plain}
	\caption[Mock-Up der Umfrageerstellung von offenen Fragen]{Mock-Up der Umfrageerstellung von offenen Fragen\\\figma}
	\label{fig:MockUmfrageOffeneFragen}
\end{figure}


\begin{itemize}
	\item Beschreibung der Mockups 
	 \begin{itemize}
		 \item Was haben wir uns bei den einzelnen Seiten gedacht -> mit den unterseiten erledigt [x]
		 \item Wie sollten die Seiten grundsätzlich aussehen. (Brainstorming) -> im Grundgerüst [x]
		 \item evtl. bissel auf Designthinking eingehen 
		 \item Welches tool haben wir dazu genutzt ->  FIGMA [x]
	 \end{itemize}
	 \item Recherche zu vorhandenen Umfragetools --> (https://www.polly.ai/slack-poll, https://strawpoll.de/, https://www.limesurvey.org/de/, https://www.surveymonkey.de/, https://pingo.coactum.de/)
	 --> Ideensammlung und Marktrecherche [x]
	 \item Grundlegende Idee der Einfachheit (evtl Gesamtkonzept) 
	 \item Man könnte auf den User Journey eingehen 
	 \item Anlehnung an DHBW farben sind vorgesehen [x]
	 \item Orientierung an modernen Websites mit header und footer [x]
	 \item Hinblick auf mobile usage
\end{itemize}
