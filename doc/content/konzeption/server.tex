% !TEX root =  ../../master.tex
\section{Server}

% VORSCHLAG: -Rene

\subsection{\acl{API}}
% PLAGIAT
Die \acf{API} dient zur Kommunikation zwischen Client und Server, dabei soll dem Client ermöglichen werden auf Ressourcen des Servers zuzugreifen.
Dazu müssen zunächst Routen definiert werden, welche nach dem Programmierparadigma \ac{REST} aufgebaut werden.
Derartige Schnittstellen stellen über die \ac{HTTP}-Methoden die \ac{CRUD}-Operationen zur Verfügung, da in der Regel der Server vorwiegend zur Datenhaltung dient.

% Abschnitt zur vollständigkeit drin. geht mir mehr um das gesamtpaket, dass sowas da is
% In der Anwendung sollen jedoch Datensätze nicht durch den Mitarbeiter verändert werden können, sondern lediglich Änderungen im \ac{ERP}-System eine Änderung der Daten hervorrufen.
% Dies kann über eine Schnittstelle gelöst werden, jedoch müsste dazu das \ac{ERP}-System angepasst werden, sodass Änderungen direkt weitergegeben werden.
% Da eine solche Veränderung jedoch nicht möglich ist, greift die Anwendung direkt auf die Datenbank des \ac{ERP}-Systems zu, um Änderungen festzustellen und in eine eigene Datenbank, wie in Kapitel \ref{chapter:Daten-Transfer} beschrieben, zu übertragen.

% Aus diesem Grund müssen die Routen lediglich lesend zur Verfügung stehen, sodass der Client alle geplanten Funktionen nutzen kann.
% Dabei stellt die \ac{API} grundlegend für jede Berechnung des Servers eine Route zur Verfügung, sodass der Client modular aufgebaut werden kann und eventuelle Einbindungen in bereits bestehende Systeme möglich sind.
%Dennoch ist es dem Nutzer möglich, die Ergebnisse der Berechnung durch eine Sortierung oder Beschränkung, u. a. auf das betrachtete Geschäftsjahr, über in der Anfrage vermerkte Parameter anzupassen.

Zur Authentifizierung existiert ebenso eine Route, welche den Nutzer überprüft und ein Token ausstellt.
Das Token muss anschließend bei jeder Anfrage übergeben werden, um die jeweilige Route nutzen zu können.

