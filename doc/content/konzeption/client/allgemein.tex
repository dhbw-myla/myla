Im Rahmen des Designs der Webseite hat sich eine vereinfachte Marktrecherche angeboten.
Dadurch soll das Risiko für Fehlentscheidungen im Rahmen der Produktentwicklungen reduziert werden.\autocite{Pioch2019}
Die Zielgruppe des Tools stellen dabei vorwiegend Dozenten oder etwaige Studenten dar, die eine Umfrage durchführen oder beantworten möchten.
Hierbei ist eine dedizierte Definition durch die Anforderungen und den Verwendungsrahmen geschehen.

Um Ideen für mögliche Design-Entscheidungen zu gewinnen, sind unterschiedliche Umfragetools evaluiert worden.
So wurde sich am simplen Aufbau der Anwendung \emph{Polly} für Teilnehmer auch bei der eigenen Konzeption orientiert.
\emph{Strawpoll} ist vor allem durch die inkrementell-hochzählenden Surveycodes aufgefallen, die es auf einfache Weise ermöglichen, an fremden Umfragen teilzunehmen, was sicherheitskritisch sein könnte.
Das Umfragetool \emph{LimeSurvey} hebt sich vor allem durch zahlreiche Fragetypen hervor.
Auch das Umfragetool \emph{Pingo} ist durch sein schlichtes und ergebnisorientiertes Design aufgefallen, was in die Konzeption der Anwendung eingeflossen ist.

In den nachfolgenden Kapiteln soll auf den grundlegende Aufbau bzw. die grundlegende Idee der Benutzeroberfläche beim Start des Projektes eingegangen werden.
Die gezeigten Mock-Ups wurden lediglich zum Aufbau und zur Orientierung entworfen, weshalb sie nicht das vollständige System abbilden.
Für die Unterteilung der verschiedenen Komponenten wurde der Wert besonders auf eine angenehme \acf{UX} und ein benutzerfreundliches \acf{UI} gelegt.
Da die Anwendung im Rahmen eines Projektes der \acs{DHBW} Mannheim konzeptioniert wird, sollen die Komponenten gemäß Anforderung~\hyperref[Anf:A16]{A16} das entsprechende Farbschema nutzen.
Das verwendete Tool zum Erstellen von Mock-Ups ist Figma\footnote{https://www.figma.com/}.
Es handelt sich hierbei um eine Webanwendung, bei der einzelne Elemente per \emph{Drag\&Drop} zu einer Arbeitsfläche hinzugefügt werden können.
Die Auswahl der Elemente umfasst dabei Schrift, Formen und Farben sowie weitere nützliche Eigenschaften, welche für den Anwendungsfall jedoch nicht relevant sind.
Durch die große Auswahl an Strukturierungsmöglichkeiten, sind alle wichtigen Aspekte eines Mock-Up-Tools vorhanden.
