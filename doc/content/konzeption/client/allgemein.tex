Im Rahmen des Designs der Webseite hat sich eine vereinfachte Marktrecherche angeboten. 
Auch wenn im Rahmen der gewünschten Anwendung bereits eine Marktrecherche nicht direkt notwendig ist, da das Tool für eine abgegrenzte Anwendergruppe geplant ist, wird diese trotzdessen durchgeführt. 
Dadurch wird sich erhofft, dass das Risko für Fehlentscheidungen im Rahmen der Produktentwicklungen reduziert werden kann.\autocite{Pioch2019}
Die Zielgruppe des Tools sind Studenten und Dozenten, welche eine Umfrage durchführen möchten. 
Hierbei ist eine dedizierte Definition durch die Anforderungen und den Verwendungsrahmen geschehen. 

Um einen Einblick in mögliche Implementierungen zu gewinnen, sind unterschiedliche Umfragetools evaluiert worden. 
Damit das Beste aus den unterschiedlichen Anwendungen herausgenommen wird, werden diese miteinander verglichen.
So hat die Einfachheit der Anwendung von \emph{Polly} überzeugt.
Die Auswertungsmöglichkeiten und Darstellungen mit der Verbindung zu der direkten Umfrage von \emph{strawpoll}. 
Sowie der Verwendung eines Teiles der kompletten Breite, um den Fokus auf die wichtigen Inhalte der Webseite zu legen. 
Der Aufbau der Seitenelemente von \emph{limesurvey} in Verbindung mit der Möglichkeit des Long-Scrolling.
Sowie Schlussendlich dem Design des Umfragetools \emph{Pingo}.  

In den nachfolgenden Kapiteln soll auf den grundlegende Aufbau bzw. die grundlegende Idee der Benutzeroberfläche beim Start des Projektes eingegangen werden. 
Die gezeigten Mockups erheben dabei keinen Anspruch vollständig zu sein, da diese lediglich zur groben Orientierung und Ausrichtung benötigt werden.
Für die Unterteilung der verschiedenen Komponenten soll auf eine spezielle \acfp{UX} und \acfp{UI} wert gelegt werden. 
Da die Applikation im Rahmen eines Projektes der DHBW Mannheim konzeptioniert wird, sollen die Komponenten an dieses Farbschema angelehnt werden. 
Das verwendete Tool zum Erstellen von Mockups ist Figma.
Figma\footnote{https://www.figma.com/} bietet eine Webanwendung an, in welcher einzelne Elemente per \emph{Drag\&Drop} zu einer Arbeitsfläche hinzugefügt werden können.
Die Auswahl der Elemente umfasst dabei Schrift, Formen und Farben, sowie weitere nützliche Eigenschaften, welche für den Anwendungsfall jedoch nicht relevant sind.
Durch die große Auswahl an Strukturierungsmöglichkeiten, sind alle wichtigen Aspekte eines Mockup-Tools vorhanden.